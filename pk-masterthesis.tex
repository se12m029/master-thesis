% !TEX encoding = IsoLatin2
% notwendige Zeile für Mac-Benutzer (muss als Kommentar stehen);
% Windows-Benutzer können die Zeile löschen.

% LaTeX-Vorlage Version 3.1,  Juli 2011
% erstellt von Dr. Andreas Drauschke (andreas.drauschke@technikum-wien.at) und
% Dr. Susanne Teschl (susanne.teschl@technikum-wien.at)
% geringfügig adaptiert von Harald Stockinger
% (harald.stockinger@technikum-wien.at)

\documentclass[a4paper,bibtotoc,oneside]{scrbook}
% Für kurze Arbeiten wäre auch die Dokumentklasse "scrartcl" ausreichend. In
% diesem Fall ist "section" die höchste Ebene ("chapter" gibt es dann nicht).
% \documentclass[a4paper,bibtotoc,oneside]{scrartcl}

% listings
\usepackage{listings}

\lstdefinelanguage{HTML5} {
  morekeywords={
    html, head, title, base, link, meta, style, script, noscript,
    body, section, nav, article, aside, h1, h2, h3, h4, h5, h6,
    header, footer, address, main, p, hr, pre, blockquote, ol, ul,
    li, dl, dt, dd, figure, figcaption, div, a, em, strong, small,
    s, cite, q, dfn, abbr, data, time, code, var, samp, kbd, sub,
    sup, i, b, u, mark, ruby, rt, rp, bdi, bdo, span, br, wbr,
    ins, del, img, iframe, embed, object, param, video, audio,
    source, track, canvas, map, area, svg, math, table, caption,
    colgroup, col, tbody, thead, tfoot, tr, td, th, form,
    fieldset, legend, label, input, button, select, datalist,
    optgroup, option, textarea, keygen, output, progress, meter,
    details, summary, menuitem, menu
  },
  sensitive=false,
  alsoletter={<>=-+},
  morecomment=[s]{<!-}{-->},
  tag=[s],
  morestring=[d]'
}

\lstdefinelanguage{JavaScript} {
  morekeywords={
    break,const,continue,delete,do,while,export,for,in,function,
    if,else,import,in,instanceOf,label,let,new,return,switch,this,
    throw,try,catch,typeof,var,void,with,yield
  },
  sensitive=false,
  morecomment=[l]{//},
  morecomment=[s]{/*}{*/},
  morestring=[b]",
  morestring=[d]'
}

\lstdefinelanguage{CSS} {
  morekeywords={
    accelerator,azimuth,background,background-attachment,
    background-color,background-image,background-position,
    background-position-x,background-position-y,background-repeat,
    behavior,border,border-bottom,border-bottom-color,
    border-bottom-style,border-bottom-width,border-collapse,
    border-color,border-left,border-left-color,border-left-style,
    border-left-width,border-right,border-right-color,
    border-right-style,border-right-width,border-spacing,
    border-style,border-top,border-top-color,border-top-style,
    border-top-width,border-width,bottom,caption-side,clear,
    clip,color,content,counter-increment,counter-reset,cue,
    cue-after,cue-before,cursor,direction,display,elevation,
    empty-cells,filter,float,font,font-family,font-size,
    font-size-adjust,font-stretch,font-style,font-variant,
    font-weight,height,ime-mode,include-source,
    layer-background-color,layer-background-image,layout-flow,
    layout-grid,layout-grid-char,layout-grid-char-spacing,
    layout-grid-line,layout-grid-mode,layout-grid-type,left,
    letter-spacing,line-break,line-height,list-style,
    list-style-image,list-style-position,list-style-type,margin,
    margin-bottom,margin-left,margin-right,margin-top,
    marker-offset,marks,max-height,max-width,min-height,
    min-width,-moz-binding,-moz-border-radius,
    -moz-border-radius-topleft,-moz-border-radius-topright,
    -moz-border-radius-bottomright,-moz-border-radius-bottomleft,
    -moz-border-top-colors,-moz-border-right-colors,
    -moz-border-bottom-colors,-moz-border-left-colors,-moz-opacity,
    -moz-outline,-moz-outline-color,-moz-outline-style,
    -moz-outline-width,-moz-user-focus,-moz-user-input,
    -moz-user-modify,-moz-user-select,orphans,outline,
    outline-color,outline-style,outline-width,overflow,
    overflow-X,overflow-Y,padding,padding-bottom,padding-left,
    padding-right,padding-top,page,page-break-after,
    page-break-before,page-break-inside,pause,pause-after,
    pause-before,pitch,pitch-range,play-during,position,quotes,
    -replace,richness,right,ruby-align,ruby-overhang,
    ruby-position,-set-link-source,size,speak,speak-header,
    speak-numeral,speak-punctuation,speech-rate,stress,
    scrollbar-arrow-color,scrollbar-base-color,
    scrollbar-dark-shadow-color,scrollbar-face-color,
    scrollbar-highlight-color,scrollbar-shadow-color,
    scrollbar-3d-light-color,scrollbar-track-color,table-layout,
    text-align,text-align-last,text-decoration,text-indent,
    text-justify,text-overflow,text-shadow,text-transform,
    text-autospace,text-kashida-space,text-underline-position,top,
    unicode-bidi,-use-link-source,vertical-align,visibility,
    voice-family,volume,white-space,widows,width,word-break,
    word-spacing,word-wrap,writing-mode,z-index,zoom
  },
  sensitive=false,
  morecomment=[l]{//},
  morecomment=[s]{/*}{*/},
  morestring=[b]",
  morestring=[d]'
}

\lstset{
  language=JavaScript,                               % Language of Code
  basicstyle=\footnotesize\ttfamily,                 % Standardfont for Code
  showstringspaces=false,                            % Underline spaces in St.
  keywordstyle=\ttfamily\bfseries\color{CadetBlue},  % Keyword Style
  identifierstyle=\ttfamily,                         % Identifier Style
  stringstyle=\ttfamily\color{OliveGreen},           % String Style
  commentstyle=\color{GrayBlue}                      % Comment Style
  xleftmargin=17pt,                                  % Left Margin
  aboveskip=\bigskipamount,
  belowskip=\bigskipamount,
  tabsize=2                                          % Tabsize in Code
}

% captions
\usepackage{caption}
\DeclareCaptionFont{white}{\color{white}}
\DeclareCaptionFormat{listing}{\colorbox{gray}{\parbox{\textwidth}{#1#2#3}}}
\captionsetup[lstlisting]{format=listing,labelfont=white,textfont=white}

% verlinkte Querverweise im pdf
\usepackage{hyperref}

% deutsche Anpassungen
\usepackage[utf8]{inputenc}
\usepackage[T1]{fontenc}
\usepackage[ngerman]{babel}

% mathematische Symbole
\usepackage{amsmath,amssymb,amsfonts,amstext}

% Kopfzeilen frei gestaltbar
\usepackage{fancyhdr}
\lfoot[\fancyplain{}{}]{\fancyplain{}{}}
\rfoot[\fancyplain{}{}]{\fancyplain{}{}}
\cfoot[\fancyplain{}{\footnotesize\thepage}]{\fancyplain{}{\footnotesize\thepage}}
\lhead[\fancyplain{}{\footnotesize\nouppercase\leftmark}]{\fancyplain{}{}}
\chead{}
\rhead[\fancyplain{}{}]{\fancyplain{}{\footnotesize\nouppercase\sc\leftmark}}

% Farben im Dokument möglich
\usepackage[usenames,dvipsnames]{color}

% enumerate styling
\usepackage{enumitem}

% Schriftart Helvetica
\usepackage{helvet}
\renewcommand{\familydefault}{cmss}

% Graphiken einbinden: hier für pdflatex
\usepackage[pdftex]{graphicx}

\usepackage{array}

% Höhe und Breite des Textkörpers etwas grösser definieren
\setlength{\textheight}{225mm}
\setlength{\textwidth}{1.05\textwidth}

% weniger Warnungen wegen überfüllter Boxen
\tolerance = 9999
\sloppy

% Anpassung einiger Überschriften
\renewcommand\figurename{Abbildung}
\renewcommand\tablename{Tabelle}

\begin{document}

% Kopf- und Fusszeilen initiieren
\pagestyle{fancy}

% Deckblatt:
\thispagestyle{empty}
\begin{picture}(0,0)
\color{white}\sffamily
\put(-101,-749){\includegraphics[width=1.002\paperwidth,
height=\paperheight]{BM_2011.pdf}}
\put(220,-670){\includegraphics[width=0.5\textwidth]{FHTW_Logo_4c.pdf}}
\put(-30, -20){\bfseries\huge MASTER THESIS}
\put(-30,-50){\Large zur Erlangung des akademischen Grades}
\put(-30,-70){\Large \glqq Master of Science in Engineering\grqq}
% Titel des Studienganges einfügen:
\put(-30,-90){\Large im Studiengang Multimedia und Softwareentwicklung}
% Titel der Arbeit einfügen:
% Die Minipage wird gesetzt, damit auch mehrzeilige Titel möglich werden.
\put(-32,-180){
\begin{minipage}{14cm}
  \bfseries\huge HTML5 Canvas vs. Adobe Flash
\end{minipage}
}
% Name der Autorin/des Autors eingeben:
\put(-30,-270){\large Ausgeführt von: Peter Kerschner, BSc.}
% Personenkennzeichen der Autorin/des Autors eingeben:
\put(-30,-290){\large Personenkennzeichen: 1210299029}
% Name der Begutachterinnen/der Begutachter eingeben:
\put(-30,-330){\large 1. BegutachterIn: Dipl.-Ing. (FH) Arthur Michael Zaczek}
\put(-30,-350){\large 2. BegutachterIn: Dipl.-Ing. Mag. Dr. Michael Tesar}
\put(-30,-390){\large Wien, \today} % das Datum des letzten Kompilierens wird
% automatisch eingesetzt
\color{black}
\end{picture}

\newpage


\section*{Eidesstattliche Erklärung}\thispagestyle{empty}
\glqq Ich erkläre hiermit an Eides statt, dass ich die vorliegende Arbeit
selbständig angefertigt habe.
Die aus fremden Quellen direkt oder indirekt übernommenen Gedanken sind als
solche kenntlich gemacht.
Die Arbeit wurde bisher weder in gleicher noch in ähnlicher Form einer anderen
Prüfungsbehörde vorgelegt
und auch noch nicht veröffentlicht. Ich versichere, dass die abgegebene Version
jener im Uploadtool entspricht.\grqq\\[5\baselineskip]
\rule{5cm}{0.2pt}\hfill\rule{5cm}{0.2pt}\\
\phantom{Datum }Ort, Datum\hfill Unterschrift\hspace{15mm}

\newpage

%%%----------------------------------------------------------
%%% Preface
%%%----------------------------------------------------------
\section*{Kurzfassung}\thispagestyle{empty}
Das HTML5 Canvas-Element und die Skriptsprache JavaScript ermöglichen es, den
Browser für Multimedia-Anwendungen und Spiele zu nutzen und die
Internet-Entwicklung unabhängig von Flash zu machen. In Zukunft könnte das
bedeuten, dass Flash-Anwendungen obsolet werden könnten. Im Zuge der Master
Thesis sollen Faktoren wie unter anderem die Möglichkeiten, Performance,
Cross-Plattformkompatibilität und mobile Verwendbarkeit von HTML5 mit Flash
verglichen werden. Die Analysen sollen anhand praktischer Umsetzungen im Bereich
Multimedia- und Spiele-Entwicklung durchgeführt werden. Als Ergebnis soll
ersichtlich sein ob das Canvas-Element und JavaScript die Flash-Entwicklung
ersetzen kann.
\\ \vfill
% Bitte 3-5 deutsche Schlagwörter eingeben, die die Arbeit charakterisieren:
\paragraph*{Schlagwörter:} Webapplikation, Web-Entwicklung, Web Standards,
HTML5, Adobe Flash
\section*{Abstract}\thispagestyle{empty}
The new features of HTML5 offer new possibilities for developers. The new
canvas element in combination with JavaScript can be used to develop
multimedia applications and games for the browser. The result is the
progress of the internet independent from Adobe Flash, which possibly means
that Adobe Flash can be replaced by HTML5.
\newline\newline
This paper compares the core technologies of HTML5 and
Adobe Flash, pointing out how HTML5 and Adobe Flash can
be used at present and which of the two technologies has a greater potential
for the future. On the basis of practical implementations, the advantages and
disadvantages of both competitors are shown and rough guidelines are be
given for the choice of which technology can be preferably used for which
project.
\newline\newline
Analyses based on practical implementations in the field of multimedia and
games include a discussion of the results and future perspectives of the two
technologies.
\\ \vfill
% Bitte 3-5 englische Keywords eingeben, die die Arbeit charakterisieren:
\paragraph*{Keywords:} Web-Entwicklung, Web Standards, HTML5,
Adobe Flash, CSS, JavaScript, ActionScript
\section*{Danksagung}\thispagestyle{empty}
--------DANKSAGUNG-------

\tableofcontents\thispagestyle{empty}
\newpage

\setcounter{page}{1}

%%%----------------------------------------------------------
%%% Hauptteil
%%%----------------------------------------------------------

% =Einleitung
\chapter{Einleitung}
Wahrscheinlich gibt es keine andere Erfindung, abgesehen von der
Telekommunikation und des Fernsehens, die die Gesellschaft der Industrie-
und Schwellenländer entscheidender prägt als das Internet.
Das World Wide Web, der bekannteste Dienst des Internets, erlebt einen
stetigen Anstieg an Nutzerzahlen. Immer mehr und vor allem jüngere
Nutzergruppen lassen sich durch soziale Netzwerke in ihren Bann ziehen und
gerade diese Netzwerke haben dabei Einfluss auf die Art der Kommunikation von
Generationen.
\newline\newline
Für die Verbraucher beschäftigt sich ein kompletter Industriezweig damit,
ihnen das Leben zu vereinfachen, in dem im Internet sämtliche Informationen
zur Verfügung gestellt und Dienstleitungen wie diverse Einkaufsmöglichkeiten
verfügbar gemacht werden. Dabei zeigt die Tendenz, dass Nutzer ihre Daten
nicht mehr auf physikalischen Datenträgern speichern und transportieren,
sondern diese in der ``Cloud'' ablegen und somit unter der Vorraussetzung
eines Internetzugangs der Zugriff von überall ermöglicht wird.
\newline\newline
Ohne der Weiterentwicklung des Internets würde es Begriffe wie "`Soziale
Netzwerke"', "`Videokonferenz"', "`Onlinebanking"' oder "`Onlineversandhaus"'
nicht geben - Begriffe, die vor allem in den vergangenen Jahren stetig an
Bedeutung gewonnen haben. Wohin die Entwicklung des Internets dabei führen
wird, lässt sich kaum vorhersagen. Fest steht, dass das Internet ein
wesentlicher Teil im Leben der aktuellen Generation einnimmt und für
zukünftige Generationen einnehmen wird.
\newline\newline
Seit der Gründung des World Wide Web Consortiums (W3C) am 1. Dezember 1994
wird bemüht an der Umsetzung eines einheitlichen Standards und dem Festlegen
der dazu notwendigen Rahmenbedingungen gearbeitet. Seit Jahren wird weltweit
solch ein neuer Standard von Webentwicklern herbeigewünscht, um unnötigen
Programmieraufwand aufgrund unterschiedlicher Browservarianten und -versionen
zu beseitigen.
\newline\newline
Die vollständige Erarbeitung und Etablierung dieses Standards wird allerdings
noch einige Jahre in Anspruch nehmen. Dabei liegt die Problematik darin, dass
der neue Standard für die Zukunft gewappnet und die Bereitstellung älterer
Webseiten kompromisslos möglich sein muss. Dieser doppelte Anspruch verzögert
die Fertigstellung ungemein.
\vfill
% -Problemstellung
\section{Problemstellung}
Die Entwickler von Flash (früher Macromedia, jetzt Adobe) haben schon früh das
Potential von ihrem Flash Player erkannt, um plattformübergreifend Video- und
Audioinhalte im Web bereitzustellen. Ein großer Vorteil war, dass Inhalte für
Flash lediglich einmal kodiert werden mussten. Tausende von Seiten die sich für
Flash als Streamingplattform von Multimediainhalten entschieden haben,
bestätigten, bis heute, das Potential.
\newline\newline
Mit dem Erscheinen von Apples iPhone und iPod touch im Jahr 2007 und der
folgenden Entscheidung, Flash auf diesen Geräten nicht zu unterstützen, mussten
Webseitenbetreiber darauf reagieren. Viele boten Video/Audio-Streams an, die
direkt im mobilen Safari Browser wiedergegeben werden konnten. Durch die
Verwendung der H.264 Kodierung konnten die Inhalte auch über den Flash-Player
(sofern vom verwendeten Gerät unterstützt) abgespielt werden. Dadurch mussten
Inhalte weiterhin nur einmal kodiert werden, um mit möglichst vielen
Plattformen kompatibel zu sein.
\newline\newline
Die Entwickler der HTML5 Spezifikation (und u.a. Apple) sind der Meinung, dass
Browser Audio und Video nativ unterstützen sollten ohne dabei auf etwaige
Plugins zurückgreifen zu müssen. Dies hat den Vorteil, dass es keine
herstellerspezifischen Einschränkungen (wie es bei Flash der Fall ist) für den
Entwickler gibt, was er mit den Inhalten anstellt, nachdem sie in die Website
eingebettet wurden. Mittels CSS und JavaScript ist es jederzeit direkt möglich
dargestellte HTML Elemente zu manipulieren.
\newline\newline
Um ein Bild für eine Webanwendung zu erstellen wird üblicherweise eine
Grafiksoftware genutzt und anschließend in die Anwendung eingebettet. Für
Animationen wird vorwiegend Flash verwendet. Mit dem in HTML5 verfügbaren
Canvas Element können Entwickler Bilder und Animationen direkt im Browser
mittels JavaScript generieren. Mit dem Canvas Element ist es möglich, einfache
bis komplexe Formen, Graphen und auch Diagramme zu erstellen, ohne auf diverse
Bibliotheken, Flash oder ein anderes Plugin zurückgreifen zu müssen.
\newline\newline
Mit der Veröffentlichung und der Implementierung von HTML5 Features in die
neuesten Brower eröffneten sich neue Möglichkeiten um Animationen, Spiele und
Entertainment-Applikationen für das Web zu erstellen. Aufgrund vieler
auftretender Fragen ist sich kein Entwickler wirklich klar, ob HTML5 den
Platzhirschen Flash verdrängen kann:
\begin{itemize}
	\item Wo liegen die Vorteile und Nachteile beider Technologien?
	\item Ist HTML5 die Zukunft und löst Flash ab?
	\item Welche Technologie ist einfacher zu erlernen?
	\item Was kann HTML5 was Flash nicht kann (und visa-versa)?
	\item Lohnt es sich noch Flash zu erlernen?
	\item Gibt es Entwicklungsumgebungen, die das Programmieren erleichtern?
\end{itemize}
% -Motivation
\section{Motivation}
Das Internet stellt den weltweit größten Netzverbund und eine Zusammenfassung
von verschiedensten Dienstleistungen dar. Die dabei am häufigsten genutzte
Dienstleistung - das World Wide Web - hat einen sehr großen Anteil an der in
den 90er Jahren enstandenen und stetig wachsenden Beliebtheit des Internets.
Seit dem hat sich das Internet vom reinen Wissenschaftsnetz zu einem
kommerziell genutzten Netz, mit einer Vielzahl an Diensten und multimedialen
Anwendungen, entwickelt.
\newline\newline
Für viele Nutzer ist das Internet fast nicht mehr aus ihrem Leben wegzudenken.
Zu verlockend sind die Möglichkeiten der flexibleren Gestaltung des
Privatlebens, Zeitersparnisse, Kontaktpflege mit Familie und Bekannten,
Arbereitserleichterung und vielem mehr. Die Möglichkeiten, die das Internet
bietet, sind fast grenzenlos.
\newline\newline
Für eine Menge Menschen bilden soziale Plattformen, wie Facebook oder Google+,
die Grundlage ihrer sozialen Interaktion und dementsprechend hoch sind die
Ansprüche. Generell steigen mit der fortgeschrittenen Entwicklung des
Internets und dessen Technologien die Erwartungen und Ansprüche der Nutzer.
\newline\newline
Aus diesem Grund muss schon vor dem Beginn eines Projektes sehr gut
abgewogen werden, wie es umgesetzt werden soll.
Eine entscheidende Rolle spielt dabei die zu nutzende Technologie aus der
Sicht der möglichen Nutzer und deren Systemvorraussetzungen (Plattform, CPU-
Leistung, Browser-Hersteller und -version). Andererseits stellen auch
Entwicklungs- und wartungsaufwand eine nicht zu unterschätzende Größe dar.

% -Zielsetzung
\section{Zielsetzung}
Das Ziel dieser Arbeit ist es, zwei Technologien zu vergleichen, die in einem
direkten Konkurrenzkampf stehen.
\newline\newline
Adobe's Flash bietet einem Entwickler die Möglichkeit, mit einmaligen
Entwicklungsaufwand viele Nutzer zu ereichen. Einzige Vorraussetzung ist ein
installiertes Flash-Player-Plugin auf dem Computer des Seitenaufrufers.
Allerdings wird genau dieses Manko bei der Entwicklergemeinde als größtes
Hindernis angesehen. Wieso dem Nutzer die Installation eines Plugins
aufzwingen, wenn ähnliche Funktionalitäten auch ohne angeboten werden können?
\newline\newline
Hingegen verlangt die Verwendung bestehender Web-Standards dem Entwickler die
Kenntnisse mehrerer Programmiersprachen und deren Verknüpfung ab, um annähernd
vergleichbare Ergebnisse zu erzielen. Dieser Umstand und die Abhängigkeit
von verschiedensten Browsern und -versionen macht es zu einer komplexen
Aufgabe ein Produkt zu entwickeln, dass in allen Browsern dieselbe
Funktionalität zur Verfügung stellt. HTML5 steht als Weiterentwicklung der
bestehenden HTML 4.01 Spezifikation in den Startlöchern um das Web
zukunftssicher zu machen. Es soll besagten Aufwand minimieren und gleichzeitig
Möglichkeiten bieten, die Flash seit geraumer Zeit gewährleistet.
\newline\newline
Deshalb gilt es in den Vergleichen herauszufinden, ob die kommenden
Webstandards HTML5 und CSS3, im speziellen das Canvas Element, in Zukunft das
proprietäre Webformat Flash samt ActionScript ablösen können. Mittels der
Evaluierung der Ergebnisse sollen Richtlinien erstellt werden, anhand derer
Web-Entwickler leicht ermitteln können, in welchen Fällen die eine Technologie
der anderen vorzuziehen ist und aus welchen Gründen.

% -Aufbau der Arbeit
\section{Aufbau der Arbeit}
Der Aufbau dieser Arbeit kann grob in vier Bereiche unterteilt werden.
Zu Beginn bietet der Theorieteil einen Überblick und eine Einführung in die
verwendeten Technologien, die für das Verständnis der Inhalte dieser Arbeit
notwendig sind. Anschließend folgt ein theoretischer Vergleich beider
Technologien anhand verschiedener Kriterien. Im dritten Teil folgt die
praktische Analyse der jeweiligen Technologien, die Beschreibungen der
umgesetzten Prototypen und deren Implementierung. Abschließend folgt eine
Diskussion und Zusammenfassung der erhaltenen Ergebnisse sowie ein Ausblick
auf die mögliche Zukunft des Webs und dessen Technologien.
% =Überblick
\chapter{Überblick}
% -HTML5 Geschichte
\section{Die Geschichte von HTML5}
HTML ist die allgegenwärtige Auszeichnungssprache des World Wide Webs. Durch den geschickten Einsatz der paar Elemente (fortan: Tags) die die Sprache unterstützt, konnten erstaunlich viele unterschiedliche Netzwerke von verlinkten Dokumente erstellt werden. Von bekannten Seiten wie Amazon, Ebay und Wikipedia bis hin zu personalisierten Blogs und Webseiten die sich auf Katzenbilder spezialisiert haben.
HTML5 ist die aktuellste Version dieser Auszeichnungssprache. Obwohl diese
Version die bisher umfangreichsten Änderungen mit sich bringt, ist es nicht die
erste Aktualisierung von HTML.
Sir Tim Berners-Lee zeichnet sich für die Entwicklung von HTML und damit dem
Beginn des Internet verantwortlich. 1991 veröffentlichte er ein Dokument mit dem
Titel ``HTML Tags'', in dem er weniger als zwei Dutzend Elemente, die für das
Schreiben von Webseiten genutzt werden konnten, vorschlug.
Die Verwendung von Tags (Wörter umgeben von eckigen Klammern: z.B. <html>) war
nicht Berners-Lees eigene Errungenschaft. Tags wurden bereits in der SGML
(Standard Generalized Markup Language) verwendet.
Anstatt einen komplett neuen Standard zu erfinden, erkannte Berners-Lee die
Vorteile, bereits existierende Standards weiterzuentwickeln - ein Trend der auch
in der Entwicklung der neuen HTML5 Spezifikation erkennbar ist.

\subsection{Von der IETF zur W3C: Der Weg zu HTML4}
Die erste offizielle Spezifikation war HTML 2.0, veröffentlicht durch die IETF,
die ``Internet Engineering Task Force''. Viele der neuen Features dieser
Spezifikation basierten dabei auf bereits veröffentlichte Implementierungen. So
bot der im Jahr 1994 marktführende Web Browser Mosaic Autoren von Webseiten
bereits die Möglichkeit Bilder in Dokumente mittels eines <img>-Tags
einzubinden. Das <img>-Tag wurde in die HTML 2.0 Spezifikation inkludiert.
Die IETF wurde allmählich durch die W3C, dem World Wide Web Consortium, ersetzt.
Folgende Aktualisierungen des HTML Standards wurden auf \url{http://www.w3.org}
veröffentlicht. In der zweiten Hälfte der Neunziger wurde der Standard mehrmals
überarbeitet bis 1999 HTML 4.01 veröffentlicht wurde.

\subsection{XHTML 1: Die Vermischung von HTML und XML}
Das nächste Update von HTML 4.01 trug den Namen XHTML 1.0. Das X steht dabei für
``eXtensible''. Die Spezifikation von XHTML 1.0 war ident zu der von HTML 4.01,
somit wurden keine neuen Elemente und Attribute eingeführt. Die Spezifikationen
unterschieden sich nur durch die zu verwendende Syntax zum Schreiben von
Dokumenten. Während Autoren von HTML Dokumenten kaum Einschränkungen im
Schreibstil von Elementen und Attributen hatten, setzte XHTML es voraus, dass
die Regeln von XML, einer strikteren Auszeichnungssprache auf der viele
Technologien des W3C basierten, eingehalten werden.
Aufgrund der strikteren Regeln einigten sich Autoren auf einen gängigen
Schreibstil, der auch unter dem HTML 4.01 Standard Verwendung fand. Während Tags
früher in Kleinbuchstaben, Großbuchstaben oder einer Mischung aus beidem
geschrieben werden konnten, verlangte ein valides XHTML Dokument, dass alle
Elemente und Attribute klein geschrieben werden.
Während XHTML 1.0 noch auf HTML basierte und lediglich die strikteren Regeln von
XML verwendete, war die XHTML 1.1 Spezifikation reines XML. Dieser Umstand
führte zu schwerwiegenden Problemen. XHTML 1.1 Dokumente konnten nicht mehr
unter dem Mime-Type \textit{text/html} definiert werden. Der bis dato
bekannteste Webbrowser Internet Explorer konnte jedoch Dokumente die mit einem
XML Mime-Type publiziert wurden, nicht darstellen.

\subsection{XHTML 2}
Das W3C war mit der vierten Version von HTML der Meinung, dass der HTML basierte
Ansatz seinen Zenith erreicht hat und setzten für die zukünftigen Version
vollständig auf XML. Trotz der fast identen Namen von XHTML 1 und XHTML 2,
konnten die Unterschiede zwischen den beiden Spezifikationen nicht größer sein.
Anders als XHTML 1, war XHTML 2 nicht abwärtskompatibel mit bereits
existierenden Webinhalten oder vorangegangenen HTML Versionen. Es sollte ein
vollkommen neuer Standard werden und es zeigte sich, dass dieser Weg nicht mit
Erfolg gekrönt sein würde.

\subsection{Die Spaltung: WHATWG}
Innerhalb des W3C bildete sich eine Gruppierung die gegen die Ansätze
rebellierten. Namenhafte Vertreter von Opera, Apple und Mozilla waren mit der
eingeschlagenen Entwicklungrichtung des W3Cs nicht zufrieden. Ihnen war es
wichtiger mehr Aufmerksamkeit auf Formate zur Entwicklung von Webapplikationen
zu richten.
Bei einem Workshop in 2004 schlug Ian Hickson, der zu der Zeit bei Opera
arbeitete, eine Weiterentwicklung von HTML vor, mit der es möglich sein soll
Anwendungen zu entwickeln. Der Vorschlag wurde durch die W3C abgelehnt.
Unzufrieden mit der Entscheidung bildeten die Rebellen eine eigene Gruppe: Die
Web Hypertext Application Technology Working Group, oder kurz WHATWG.

\subsection{Der Weg von Web Apps 1.0 zu HTML5}
Von Beginn an operierte die WHATWG anders als das W3C. Während das W3C einen
kosensorientierten Ansatz anwendet, Themen werden vorgetragen, es wird
diskutiert und abgestimmt, wird auch bei der WHATWG diskutiert und abgestimmt,
allerdings liegt die finale Entscheidung, was in die Spezifikation komm und was
nicht, beim Editor. Der Editor ist Ian Hickson.
Der W3C Ansatz klingt demokratisch und fair, allerdings führt dieser auch dazu,
dass der Prozess stark verlangsamt wird. Bei der WHATWG hat jeder die
Möglichkeit mitzuwirken, da die letzte Entscheidung aber beim Editor liegt,
entwickelt sich alles schneller.
Schon zu Beginn der Gründung von WHATWG wurden die Projekte in zwei große
Spezifikationen aufgeteilt: Web Forms 2.0 und Web Apps 1.0. Beide
Spezifikationen sollen die bisherige HTML Spezifikation erweitern. Mit
fortschreitender Entwicklung wurden alle Spezifikationen in eine einzige
implementiert und in HTML5 umbenannt.

\subsection{Die Wiedervereinigung}
Während HTML5 von der WHATWG weiter entwickelt wurde, setzte das W3C die
Entwicklung der XHTML 2 Spezifikation fort. Erst im Oktober 2006, schrieb Sir
Tim Berners-Lee einen Blog-Post in dem er zugab, dass der Versuch, das Internet
von HTML auf XML zu übertragen, nicht funktionieren wird.
Nur wenige Monate später entschied sich die W3C eine neue HTML Working Group zu
bilden. Anstatt komplett von Anfang an zu beginnen, nutzen sie glücklicherweise
die bisherige Arbeit der WHATWG als Basis für zukünftige HTML Spezifikationen.
Dieses hin- und her führte zu einer schwer überschaubaren Situation. Das W3C
arbeitete parallel an den beiden unterschiedlichen, nicht kompatiblen
Spezifikationen: XHTML 2 und HTML 5 (mit Leerzeichen). Währenddessen arbeitete
die WHATWG, an der Spezifikation für HTML5 (ohne Leerzeichen) auf der die Arbeit
der W3C aufbaut.

\subsection{HTML5: 2012 und 2022}
Heute ist der aktuelle Stand der HTML5 Spezifikation nicht mehr so
undurchsichtig wie er früher war, allerdings gibt es noch immer offene Fragen.

Es gibt noch immer zwei Gruppen die an HTML5 arbeiten.
Die wohl wichtigste Frage für Webentwickler ist "`Wann können wir es nutzen?"'.
In einem Interview gab Ian Hickson an, dass HTML5 frühestens 2022 den ``proposed
recommendation'' Status erreichen wird. Klingt nach einer langen Wartezeit,
allerdings bedeutet ``proposed recommendation'', dass die HTML5 Spezifikation
zwei mal komplett implementiert werden muss. Als Vergleich: HTML 4 existiert nun
seit über einem Jahrzehnt und hat noch nicht die gesetzten Features erreicht.
Wenn man den Umfang der Spezifikation betrachtet, klingt dieses Datum hoch
gesteckt. Browserhersteller sind nach wie vor nicht dafür bekannt, dass
existierende Standards so schnell wie möglich implementiert werden. Der Internet
Explorer benötigte mehr als ein Jahrzehnt um das \textit{abbr}-Element richtig
darstellen zu können.

Für Webentwickler war das Jahr 2012 wesentlich wichtiger. 2012 erreichte die
HTML5 Spezifikation den ``candidate recommendation'' Status, der gleichbedeutend
ist mit "`fertig und abgeschlossen"'.
Allerdings reicht das alleine leider auch nicht aus. Wirklich entscheident ist
es wann Webbrowser den neuen Standard unterstützen. Schon die Veröffentlichung
des CSS 2.1 Standards zeigte, dass man nicht auf die Fertigstellung der
Spezifikationen warten sollte, sondern, wenn möglich, die Features nutzen sollte
sobald es möglich ist. Das selbe gilt auch für HTML5. Sobald Webbrowser
bestimmte Features der Spezifikation unterstützen können diese auch jederzeit
verwendet werden.
Man darf nicht vergessen, dass HTML5 keine komplett neu entwickelte Sprache ist.
Im Sinne der HTML Spezifikation ist es eher eine Evolution als eine Revolution.
Da HTML5 auf den früheren Versionen aufsetzt und irgendeine Version des HTML
Standards zur Erstellung von Webseiten genutzt wird, wird bereits HTML5 genutzt.


% \section{Geschichte}
% \subsection{Der Begründer des World Wide Web}
% Die Geschichte des Internets geht zurück bis in die frühen neuziger als Tim
%Berners-Lee das World Wide Web entwickelte mit der Hyper Text Markup Language
%(HTML) als Auszeichnungssprache. Während die Elemente (auch Tags genannt) zur
%Auszeichnung von Inhalten in strukturierte Teile (Paragraphen, Überschriften,
%Listen, etc.) dabei auf der Standard Generalized Markup Language (SGML), eine
%international anerkannte Methode um Texte zu strukturieren, basierte, war die
%Idee des Hypertext Links von Tim Berners-Lee.
%Ein Jahr später veröffentlichte Berners-Lee die erste Version seines Browsers.
%Während dieser Zeit waren die Möglichkeiten von HTML stark eingeschränkt und
%Entwickler konnten lediglich einfache Textinhalte im Web veröffentlichen. Mit
%dem Erscheinen von HTML+, welche von Dave Ragget von Hewlett-Packards Labs
%entwickelt wurde, gab es die Möglichkeit mittels IMG Element auch Bilder in
%Webseiten zu inkludieren

%Versuche um weitere Verbesserungen zur HTML Spezifikation hinzuzufügen, führte
%zur Veröffentlichung einer neuen Version, HTML 2.0. Sie basierte vollständig auf
%der alten Version wurde jedoch mit weiteren Funktionen und der Document Type
%Definition (DTD) erweitert. Die Funktionen von HTML 2.0 wurden zum Standard für
%alle Webbrowser. Der Wunsch nach Personalisierung, gestalterischer Freiheit und
%einem zeitgemäßen Look and Feel führte zu HTML 3.0 und der Einführung der
%Cascading Style Sheets.
%
%Probleme entstanden erst als Netscape sich entschied eigene Tags und Attribute
%in ihre Browser zu implementieren. Entwickler mussten Tags speziell für Netscape
%anpassen. Netscape sah jedoch ein, dass ihr Versuch nicht zielführend ist und
%gab ihe Erweiterungen auf. Mit dem steigenden Interesse am Internet
%veröffentlichte Microsoft die erste Version des Internet Explorers.
%
%Mit dem schnellen Fortschreiten der Entwicklung des HTML Standards wurde HTML
%4.0 veröffentlicht. Diese Version enthielt weitere Verbesserungen und
%Möglichkeiten in der Entwicklung von Webseiten. Es gab mehr Möglichkeiten um
%Multimedia in Webseiten einzusetzen, nutzen von Skiptsprachen, verbesserte
%Druckeigenschaften, Style Sheets und Optimierungen für Menschen mit
%Behinderungen.

%IMAGE Timeline of Web Technologies
% http://www.instantshift.com/2012/07/20/the-evolution-of-html5-infographic/
% publishing multimedia on the web.pdf

\section{Rich Media}
\subsection{Audio}
MP3 stellt das allgegenwärtige Format zur Kodierung von Audio-Dateien dar. Um sich diese
Audio-Dateien auch anhören zu können waren propritäre Technologien notwendig. Damit
wurde der Flash Player allgegenwärtig.
\newline\newline
Mit HTML5 bietet sich eine neue Technologie an die versucht den Platzhirschen Flash zu
verdrängen.
Das Einbinden einer Audio-Datei in ein HTML5-Dokument ist äußerst simpel:
\begin{verbatim}
<audio src="sample.mp3">
</audio>
\end{verbatim}
Das Audio Element bietet mehrer Attribute zum Steuern der Audioausgabe an.
Das Attribut ``autoplay'' sorgt dafür, dass die Audio-Datei nach vollständigem Laden
sofort gestartet wird. ``loop'' hingegen, kümmert sich darum, dass das die Audio-Datei
in einer Endlosschleife abgespielt wird.
\begin{verbatim}
<audio src="sample.mp3" autoplay loop>
</audio>
\end{verbatim}
Eine Besonderheit dieser Attribute ist, dass sie im Gegensatz zu den herkömmlichen HTML-Attributen,
keinen Wert zugewiesen bekommen. Das liegt daran, dass es sich bei diesen Attributen bereits um
``Boolean''-Attribute handelt. Das zuweisen eines Wertes, z.B. autoplay="no", wird an der Ausführung
des Attributes nichts ändern - entweder die Attribute werden angegeben und ausgeführt oder
eben nicht.

\subsubsection{Audio API}
Das Boolean-Attribut controls sorgt dafür, dass der Browser die nativen Kontrollmöglichkeiten
anzeigt und den Usern zu Verfügung stellt. Somit ist es sehr schnell möglich die Audiowiedergabe
zu Starten/Stoppen, die Position und die Lautstärke zu verändern.
%Image Audioplayer mit Controls
Über JavaScript ist es möglich die Audio API zuzugreifen und somit Methoden für Play, Pause und
Eigenschaften wie Lautstärke zu nutzen. Ein einfaches Beispiel mit Button Elementen und
Inline Event Handler:
\begin{verbatim}
<audio id="player" src="sample.mp3">
</audio>
<div>
<button onclick="document.getElementById('player').play()">
Play
</button>
<button onclick="document.getElementById('player').pause()">
Pause
</button>
<button onclick="document.getElementById('player').volume += 0.1">
Volume Up
</button>
<button onclick="document.getElementById('player').volume -= 0.1">
Volume Down
</button>
</div>
\end{verbatim}

\subsubsection{Audioformate}
Obwohl das Audio Element einen sehr guten Eindruck macht gibt es einen Wehrmutstropfen,
und dieser liegt nicht in der Spezifikation des Elements. Das Problem ist die breite Fragmentierung
von Audiocodecs. Leider gibt es bezüglich der zu verwendeten Codecs unterschiedliche Meinungen
unter den Browserherstellern. Während heutzutage das MP3 Format allgegenwärtig ist,
ist es nach wie vor kein offenes Format. Die Folge ist, dass MP3 Dateien von Applikationen nur
dann dekodiert werden können, wenn für die entsprechenden Patentrechte bezahlt wird.
Für Giganten wie Apple oder Adobe stellt das kein gröberes Problem dar, allerdings erschwert es die
Arbeit von kleineren Unternehmen und Open-Source Organisationen. Infolgedessen gibt
Apples Safari ohne Probleme MP3 Datei wieder während Mozillas Firefox daran scheitert.
\newline\newline
Natürlich existieren weitere Audioformate. Der Vobis Codec - auch bekannt unter der Datei-Endung
.ogg - ist zum Beispiel nicht patentiert. Firefox unterstützt den Codec aber Safari nicht.
\newline\newline
Glücklicherweise ist es nicht notwendig eine fixe Entscheidung bei der Auswahl des Codecs zu treffen.
Anstatt das src Attribut im sich öffnenden <audio> Tag zu nutzen, können mehrere Dateiformate
über source Elemente festgelegt werden:
\begin{verbatim}
<audio controls>
<source src="sample.ogg">
<source src="sample.mp3">
</audio>
\end{verbatim}
Ein Browser der Ogg Vorbis Datei wiedergeben kann wird sich für die weiteren source Elemente
nicht mehr interessieren. Ein Browser der MP3 wiedergeben kann aber Ogg Vorbis nicht wird
einfach das erste Source Element überspringen und die Datei im zweiten wiedergeben.
Zusätzliche Hilfe bei der Entscheidung bietet die Deklarierung des entsprechenden
Mime Types der Audio Datei:
\begin{verbatim}
<audio controls>
<source src="sample.ogg" type="audio/ogg">
<source src="sample.mp3" type="audio/mpeg">
</audio>
\end{verbatim}
Um die Möglichkeiten vom Audio Element vollständig ausnutzen zu können wird angeraten
MP3 und Ogg Vorbis als Codecs zu verwenden.

\subsubsection{Fallback Lösungen}
Auch wenn das festlegen von mehreren source Elementen sehr nützlich ist, muss
bedacht werden, dass Browser existieren die das Audio Element nicht unterstützen.
Internet Explorer und Konsorten müssen auf die altmodische Art auf ein Flashumsetzung
zurückgreifen. Glücklicherweise unterstützt das Audie Element die Nutzung von Flash.
Alles was zwischen den sich öffnenden und schließenden <audio> Tags befindet und
kein source Element ist wird nur dem Browser, der das audio Element nicht unterstützt,
angezeigt:
\begin{verbatim}
<audio controls>
<source src="sample.ogg" type="audio/ogg">
<source src="sample.mp3" type="audio/mpeg">
<object type="application/x-shockwave-flash"
<param name="movie" value="player.swf?soundFile=sample.mp3">
</object>
</audio>
\end{verbatim}
Das object Element bietet zusätzlich eine Möglichkeit an um Fallback Inhalte anzubieten.
So kann zum Beispiel im schlimmsten Fall ein gewöhnlicher Downloadlink angezeigt werden.
\begin{verbatim}
<audio controls>
<source src="sample.ogg" type="audio/ogg">
<source src="sample.mp3" type="audio/mpeg">
<object type="application/x-shockwave-flash"
<param name="movie" value="player.swf?soundFile=sample.mp3">
<a href="sample.mp3">Audio Datei herunterladen</a>
</object>
</audio>
\end{verbatim}
Mit diesem Code werden bereits vier Fallback Ebenen angeboten:
\begin{itemize}
\item{Der Browser unterstützt das Audio Element und den Ogg Vorbis Codec.}
\item{Der Browser unterstützt das Audio Element und den MP3 Codec.}
\item{Der Browser unterstützt das Audio Element nicht, hat aber das Flash Plug-in installiert.}
\item{Der Browser unterstützt das Audio Element nicht und hat kein Flash Plug-in installiert.}
\end{itemize}

\subsection{Video}
Mit dem anstieg der verfügbaren Bandbreite stieg auch das Interesse an Video Inhalten an.
Das Flash-Plugin ist derzeit noch die erste Wahl wenn Videos im Web angeboten werden sollen.
Mit HTML5 könnte sich das ändern.
\newline\newline
Das Video-Element funktioniert genauso wie das Audio-Element. Es unterstützt die
selben optionalen ``autoplay'', ``loop'' und ``preload'' Attribute. Der Speicherort des Videos
kann entweder mittels ``src'' Attribut im Video-Element oder mittels ``source'' Elementen, die
sich verschachtelt zwischen den sich öffnenden und schließenden <video> Tags befinden,
festgelegt werden. Zur Darstellung eines geeigneten User Interfaces kann entweder mittels
``controls'' Attribut dem der Browser die Darstellung übernehmen lassen oder es wird mit
entsprechenden HTML-Elementen, CSS Befehlen und JavaScript ein benutzerdefiniertes erstellt.
\newline\newline
Einer der wesentlichen Unterschiede zwischen dem Audio und Video Element ist, dass Videos
natürlicherweise einen fixen Bereich der Webseite einnehmen werden. Um diesen
Bereich festzulegen müssen im Video Element die entsprechenden Dimensionen definiert werden:
\begin{verbatim}
<video src="sample.mp4" controls width="360" height="240">
</video
\end{verbatim}
Um beim Laden des Videos kann mittels ``poster'' Attribute dem Browser mitgeteilt werden,
währenddessen ein representatives Bild anzuzeigen:
\begin{verbatim}
<video src="sample.mp4" controls width="360" height="240"
poster="sampleimage.jpg">
</video
\end{verbatim}
% Image Placeholder and Dimensions
Der Kampf der rivalisierenden Videoformate stellt sich noch extremer als bei den Audio Formaten dar.
Die am stärksten vertretensten Formate sind das patentierte MP4 und das freie Theora Video. Wie beim
Audio Element muss das Video in mehreren Kodierungen verfügbar und Notfalls eine Fallback Lösung
vorhanden sein.
\begin{verbatim}
<video controls width="360" height="240"
poster="sampleimage.jpg">
<source src="sample.ogv" type="video/ogg">
<source src="sample.mp4" type="video/mp4">
<object type="application/x-shockwave-flash" width="360" height="240"
data="player.swf?file=movie.mp4">
<a href="movie.mp4">Video herunterladen</a>
</object>
</video
\end{verbatim}
Die Autoren der HTML5 Spezifikation hofften auf eine Standardisierung des Videoformates. Allerdings
konnten sich die Browserhersteller, bis heute, nicht auf ein einziges Format einigen.

\subsubsection{Native Unterstützung von Videos}
Die Fähigkeit Videos nativ in Webseiten einzubinden stellt womöglich eine der aufregensten Erweiterung
von HTML dar, seit der Einführung des ``img'' Elements. Giganten wie Google zögern nicht lange und
zeigen bereits ihren Enthusiasmus mit einer auf HTML5 basierenden YouTube-Version:
\url{http://youtube.com/HTML5}
Eines der Probleme von Plug-ins zur Darstellung von Webinhalten ist, dass der Inhalt des Plug-ins
von dem restlichen Inhalt der Webseite geschützt ist (``sandboxed''). Nativen Rich Media Elementen
in HTML haben zur Folge, dass sie ohne Probleme mit den anderen Web-Technologien,
CSS und JavaScript, zusammen arbeitet.
\newline\newline
Das Video Element ist somit nicht nur programmierbar sondern auch stylebar.
% Image Skin & Style
Ein Plug-in bietet derartige Möglichkeiten nicht an.

\subsection{Canvas}

Das Canvas Element ist eine Umgebung zur Erstellung von dynamischen Bildern.
Das Element ist genauso einfach wie das Audio oder Video Element zu verwenden.
Als Attribute werden lediglich Breiten- und Höhenangaben des Canvas angeboten:
\begin{verbatim}
<canvas id="canvas" width="360" height="240">
</canvas>
\end{verbatim}
Alles was sich zwischen den sich öffnenden und schließenden <canvas> Tags
befindet, wird nur Browsern angezeigt, die das canvas Element nicht
unterstützen.
\begin{lstlisting}[language=html]
<canvas id="canvas" width="360" height="240">
<p>Canvas Element wird von ihrem Browser nicht unterstuetzt.</p>
</canvas>
\end{lstlisting}

JavaScript wird verwendet um das Canvas produktiv nutzen zu können.
Um mit dem Canvas arbeite zu können muss immer das entsprechende
Element über ihre ID ausgewählt und der Kontext festgelegt werden.
Kontext bedeutet in diesem Fall welche API genutzt werden soll:
\begin{verbatim}
var canvas = document.getElementById('canvas');
var context = canvas.getContext('2d');
\end{verbatim}
Aktuell stehen nur der 2D und WebGL Kontexts zur Verfügung.
Mit der Auswahl des Kontextes ist das Canvas Element bereit
um für das Zeichnen von Bildern genutzt zu werden.
Die 2D API verfügt über so ziemlich alle Tools die man auch bei einem
Grafik Programm wie Illustrator erwartet: Es können Linien, Konturen, gefüllte
Flächen, Verläufe, Schatten, Formen und Bézier Kurven gezeichnet werden.
Der wesentliche Unterschied besteht allerdings darin, dass kein
grafischen User Interface genutzt wird, sondern alles mit JavaScript
definiert werden muss.

\subsubsection{Mit Code zeichnen}
Um die Farbe einer Kontur bzw. einer Linie zu definieren ist folgender
Code notwendig:
\begin{verbatim}
context.strokeStyle = "\#990000";
\end{verbatim}
Alles was nun auf dem Canvas gezeichnet wird, hat eine rote Kontur.

Die Syntax für die strokeRect Methode sieht dabei folgendermaßen aus:
\begin{verbatim}
strokeRect(left, top, width, height);
\end{verbatim}
Um nun ein rotes Rechteck zeichnen, das 20 Pixel vom linken Rand und
30 Pixel vom oberen Rand des Canvas entfernt ist und 100 Pixel breit und
50 Pixel hoch ist, ist folgender Code notwendig:
\begin{lstlisting}
context.strokeRect(20, 30, 100, 50);
\end{lstlisting}
% Image Rectangle Canvas
Dabei handelt es sich noch um ein sehr einfaches Beispiel. Die 2D API
bietet eine sehr umfangreiche Auswahl an Methoden wie
fillStyle, fillRect, lineWidth, shadowColor und viele mehr.
\newline\newline


\section{Verfügbarkeit}
Seit der offiziellen Präsentation der ersten Entwürfe der erneuerten
Webtechnologie, ist die Begeisterung in der IT Branche groß. Diese Begeisterung
ist besonders an dem Enthusiasmus der Web-Giganten Apple, Google und Mozilla und
der stetigen Implementierung von HTML5 und CSS3 in ihre Browser erkennbar.

\subsection{HTML5, CSS3 und JavaScript: Das Web von morgen}
Graph

\subsection{Verfügbarkeit auf Computer}
Ob und wie weit die Implementierung der neuen Webfeatures in Browsern
fortgeschritten ist, hängt vollständig von den jeweiligen Browserherstellern ab.
Um die Verfügbarkeit auf Computern festzustellen, muss der Marktanteil der
unterschiedlichen Browser mit einbezogen werden.

IMAGE Browser Jahr

Es ist notwendig die kommenden Trends zu bedenken, da die weitere
Implementierung von HTML5 stark von der Popularität des jeweiligen Web-Browsers
abhängt. Unbekanntere Browserhersteßller werden naturgemäß mehr Zeit für die zur
Verfügungstellung von HTML5-Features benötigen als bekannte und beliebte Größen
wie Google oder Mozilla.

IMAGE Browser Statistik Monat zu Monat

Die angeführten Statistiken zeigen auf, dass nicht nur der gegenwärtige
Prozentsatz an Nutzern sondern auch der Trendverlauf darauf schließen lässt,
dass der Internet Explorer tatsächlich nicht mehr zu den marktführenden Browsern
zählt, wodurch auch dessen Bedeutung in der Web-Entwicklung schwindet.
Aufgrund der sich reduzierenden Nutzeranzahl und dem Umstand, dass der Internet
Explorer der Browser mit der am geringsten fortgeschrittensten Implementierung
des HTML5 Spezifikation am Markt ist, ermöglicht vielen Web-Entwicklern, schon
vor der Fertigstellung der Spezifikation, die Nutzung einiger neuen Features.

Microsofts Marktstrategie ihr eigenes Betriebssystem (Windows) ausschließlich
mit dem eigenen Browser auszuliefern ändert kaum etwas an der Statistik. Viele
Benutzer installieren daher zum Beispiel Mozilla Firefox oder Google Chrome
selbstständig.

\subsection{Verfügbarkeit auf mobilen Geräten}

\section{Möglichkeiten}

Folglich werden einige neue Elemente der HTML5 Spezifikation genauer
vorgestellt. Diese wurden auch in der Umsetzung der praktischen Ausarbeitungen
verwendet.


\subsection{Spiele}
\subsection{Animationen}
Mit HTML5 Canvas und den CSS3 Neuerungen gibt es für Entwickler neue APIs um
Grafiken im Web zu zeichnen und diese auch zu animieren. Die APIs bieten
einfache anzuwendente Funktionen an um vordefinierte Formen zu zeichnen, Bilder
zu importieren, die Darstellung von bereits gezeichneten Bildern zu verändern
oder diese auch zu animieren. Im Gegensatz dazu bietet Flash eine komplette IDE
um schnell komplexe Formen zu erstellen, diese anzuzeigen, deren Verhalten zu
steuern und zu animieren.

Ein Charakter lässt sich mit einer Zeichenapplikation um ein vielfaches
einfacher gestalten als mit mehreren Zeilen Code. Flash bietet eine
vollständiges Tool zur Erstellung und Animierung komplexer Grafiken und ist
damit wesentlich effizienter zu nutzen, solange es kein equivalentes Tool für
HTML5 und CSS3 gibt.

Dieses Kapitel soll aufzeigen, ob HTML5 und CSS3 effizient für die Erstellung
von Animationen verwendet werden kann. Anhand bestehender Animationen ...
Anschließend werden die Fragen behandelt wie eine Animation mittels Code
erstellt wird und ob sich der Zeitaufwand für Entwickler auszahlt. Abschließend
wird analysiert ob Webbrowser die Animationen flüssig wiedergeben können.

\subsubsection{CSS3 Spiderman}
Ein kurzer Animationsfilm der mittels HTML5, CSS3 und jQuery erstellt wurde. Die
Animationen sind komplex. So bewegt sich der Hintergrund, es gibt mehrere
Szenen, unterschiedliche Betrachtungswinkel, Bewegungen und Gesichtsausdrücke.
Der Film beinhaltet auch Musik mittels dem HTML5 Audio Element.

Bisher gibt es viele Experimente die mittels CSS3 umgesetzt wurden. Im Vergleich
gibt es nur wenige die auf das Canvas Element und JavaScript setzen.

CSS3 bietet einfache Transformationen (Rotation, Translation, Skalierung) und
reichen für simple Animationen. Flash bietet zusätzlich Formtransformationen,
als Beispiel die allmähliche Veränderung eines Kreises in ein Quadrat. Bisher
ist das mit CSS3 nicht möglich.

CSS3 zeigt im Vergleich zu HTML5 ein größeres Potential um Animationen zu
erstellen. Allerdings ist es schwierig mittels Code komplexere Animationen zu
erstellen. Solange es kein Tool für HTML5 und CSS3 Animationen existiert wird
Flash weiterhin die führende Software in diesem Bereich bleiben.

Performancetechnisch wurde diese Animation ohne jegliche Probleme flüssig in
jedem modernen Browser angezeigt.

\subsection{Entertainment}

\subsubsection{Audio \& Video}
Flash ist bis heut noch das bekannteste propritäte Plugin um Audio- und
Videoinhalte im Internet zur Verfügung zu stellen. Jedoch bietet nun HTML5
eigene Features die diesem Bereich gewidmet und sind ernst zu nehmende
Alternativen zu Flash. Es werden die Audio und Video Spezifikation von HTML5
genauer betrachtet.

\subsubsection{Audio}
Das Audio Element steht bisher in allen aktuellen Webbrowsern zur Verfügung und
ist ähnlich einfach zu nutzen wie das Image Element. Folgender Code zeigt wie
ein Audiofile in eine Webseite eingebunden werden kann:
\begin{verbatim}
<audio src="audio.ogg" controls>
<p>Your browser does not support the audio element.</p>
</audio>
\end{verbatim}
Leider gibt es bezüglich der zu verwendeten Codecs unterschiedliche Meinungen
unter den Browserherstellern. Um die Möglichkeiten vom Audio Element vollständig
ausnutzen zu können wird angeraten MP3 und Ogg Vorbis als Codecs zu verwenden.

\subsubsection{Video}
Ein Video kann mittels folgendem Code auf einer Webseite angezeigt werden:
\begin{verbatim}
<video src="movie.ogg" width="640" height="360" controls>
<p>Your browser does not support the video element.</p>
</video>
\end{verbatim}
Auch beim Video Element ist die Frage nach dem Codec nicht einfach zu
beantworten. Das Video Element kann mit allen modernen Webbrowsern genutzt
werden, allerdings gibt es andere Probleme

\begin{itemize}
\item Apple Geräte können den Ogg Theora Codec aufgrund von Hardware Problemen
nicht wiedergeben
\item Opera und Firefox unterstützen den H.264 Codec aufgrund von Lizenz
Problemen nicht.
\end{itemize}

Erst vor kurzem brachte Google eine mögliche Lösung ist Spiel: Die Nutzung von
WebM. Dabei handelt es sich um einen lizenzfreie Videocodec. Firefox, Opera,
Chrome und auch IE haben bereits bestätigt, dass dieser Codec in Zukunft
unterstützt wird.
Einige der bekanntesten Video Streaming Anbieter sind bereits dabei HTML5 für
Videoinhalte zu nutzen, darunter unter anderem YouTube und Vimeo.

In Kombination mit einem Canvas Element können die Inhalte des Video Elements
auf verschiedenste Art manipuliert werden. Unter anderem könnte das Bild in
mehere Teile zerschnitten, Explosionen eingefügt und Filter angewendet werden.
Allerdings, benötigen komplexe Effekte und Manipulationen eine
Hardwarebeschleunigung. Flash nutzt zur effizienten Darstellung von Animationen
und Videos die Grafikkarte des Anwenders. Die Möglichkeiten von HTML5 sind noch
nicht soweit entwickelt und bietet keine native Möglichkeit um die Hardware des
Anwenders zu nutzen.

Um auch die Video Element effizient nutzen zu können, sollten die Quelldateien
mit mindestens zwei verschiedene Codecs zur Verfügung gestellt werden.

\section{Vor- und Nachteile}
Jeffrey Zeldman war schon im Jahr 2010 der Meinung, dass HTML, CSS und
JavaScript in Zukunft die Entwicklung von Rich Media Applikationen vorantreiben
wird. Einige der Vorteile werden hier aufgezählt.

\subsection{Plattformübergreifende Web-Technologien}
Web-Entwickler mussten vor der Umsetzung einer Webanwendung oder Webseite
entscheiden für welche Webbrowser entwickelt werden soll. Dieser Umstand
entstand aufgrund der ungeeigneten Konzeption früherer HTML und CSS Versionen.
Mit den neuen Webstandards (HTML5, CSS3 und JavaScript) ist es nun möglich
plattformübergreifende Anwendungen zu erstellen, die nicht nur auf allen
Desktopbrowsern sonder auch auf mobilen Browsern genutzt werden können.

\subsection{Optimiertes Mobile-Web}
Bei der Entwicklung von Anwendungen für mobile Geräte spielt die begrenzte
Bandbreite des Internets eine entscheidente Rolle. Mittels CSS3 können viele
Bilder durch Gradienten, Schatten und andere Effekte ersetzt werden.
Überdimensionierte Hintergrund- und Füllbilder werden somit überflüssig und
führen zu einem geringerem Verbrauch der zur Verfügung stehenden Bandbreite.
Einige HTML5 Features wie Local oder Session Storage erlauben es Daten
clientseitig zu speichern und Applikationen auch ohne Internet verfügbar zu
machen. Zusätzlich müssen Daten die bereits runtergeladen wurden nicht noch
einmal gealden werden.

\subsection{Rich Media ohne Plugins}
Die Audio, Video und Canvas APIs von HTML5 erlauben es Rich Media ohne
Verwendung von Plugins im Internet zu nutzen. Propritäte Technologien die in
diesem Bereich hauptsächlich verwendet werden haben mit HTML5 einen ernst zu
nehmenden Konkurrenten bekommen.

\subsection{Rich Media Applikationen}
Der Großteil der neuen HTML5 Features sind speziell für die Erstellung von Rich
Media Applikationen vorgesehen.

% =HTML5 Syntax
\section{Die Syntax von HTML5}

% =HTML5 Rich Media
\section{Rich Media}

% =HTML5 Web Forms 2.0
\section{Web Forms 2.0}

% =HTML5 Semantik
\section{Semantik}

% =HTML5 Schon Heute
\section{HTML5 schon heute}
% =MOBILE
\chapter{Mobile Geräte}

\section{Die Entwicklung des mobilen Internets}
Die Optimierung von Applikationen und Webseiten für mobile Endgeräte gewinnt
immer mehr an Wichtigkeit. Wollte ein Nutzer vor der aktuellen Smartphonegeneration 
(iPhone und Android) unkompliziert mobil im Internet surfen oder E-Mails schreiben 
wollte, kam er nicht um ein Blackberry von RIM herum. RIM legte bei ihren Telefon
den Hauptfokus auf die E-Mail Funktion. Das Surfen auf einem mobilen Telefon
war meist durch zu kleine Bildschirme, langsame Verbindungsgeschwindigkeiten,
umständlicher Bedienung, zu hohen Kosten und Einschränkungen in der Darstellung
und Nutzung von Webinhalten stark eingeschränkt. Die Einführung der 
GSM-Erweiterungen GPRS und EDGE sowie die noch schnelleren, auf dem 
Mobilfunkstandard basierenden UMTS-Datenübertragungsverfahren HSDPA und
HSUPA sorgte, insbesondere mit dem Erscheinen des iPhone von Apple, für einen
langsamen Umschwung in der mobilen Internetnutzung. Das Bankhaus Morgan
Stanley hat im April 2010 eine 87-seitige Präsentation zum Thema "`Internet Trends"'
veröffentlicht, welche zum einen dem mobilen Internet ein schnelleres Wachstum 
gegenüber dem desktop-basierten Internet verspricht, und zum anderen, dass im
Jahr 2014 dieses in der Nutzung überholen wird.

\section{iOS von Apple}
\section{Android von Google}
% =FLASH
\chapter{Adobe Flash}

\section{Was ist Flash?}
Während die Popularität von HTML5 immer weiter steigt, werden immer mehr
Vergleiche mit "`Flash"' durchgeführt. Hierbei handelt es sich um ein Produkt
der Firma Adobe, die sich auch unter anderem für Software wie Photoshop oder
Illustrator verantwortlich zeigen. Allerdings bezieht sich das Wort Flash in
den meisten Fällen nicht auf die gleichnamige Entwicklungsumgebung der Firma,
sondern auf den Flash Player, der das Abspielen von proprietären SWF-Dateien
im Browser ermöglicht. Anfänglich war Flash ein Programm um Illustrationen und
einfache Animationen zu erstellen und wurde 1997 von der Firma Macromedia,
welche im Jahr 2005 von Adobe aufgekauft wurde, entwickelt.

\section{Die Geschichte von Flash und ActionScript}
Seit der Veröffentlichung von Flash im Jahr 1996 wurde Flash und ActionScript
gemeinsam weiter entwickelt. Mittlerweile stellt die Kombination aus Design-
und Animations-Tools in Flash und den interaktiven Möglichkeiten von
ActionScript eine der mächtigsten, vielseitigsten und beliebtesten
Entwicklungsumgebung dar. Allerdings war ActionScript zu Beginn eine sehr
bescheidene Programmiersprache.
\newline\newline
Die ersten drei Versionen von Flash boten noch keine Programmiertools an und
interaktive Abläufe wurden durch Drag-and-Drop Optionen als Actions festgelegt.
Lediglich die Navigation durch die Zeitleiste und das Erstellen von Links waren
über diese Actions möglich.
\newline\newline
Flash 4 war die erste Version, die die Möglichkeit bot, Code mittels einer
einfachen Skriptsprache zu schreiben, welche inoffiziell bereits ActionScript
genannt wurde. Mit Flash 5 entwickelte sich ActionScript weiter und wurde zur
offiziellen Skriptsprache. Mit jeder folgenden Version von Flash wurden auch
die Möglichkeiten von ActionScript erweitert. ActionScript ermöglichte die
interaktive Kontrolle von Text, Animation, Sound, Video, Data und mehr. 2003
wurde ActionScript 2.0 vorgestellt. Die Möglichkeiten dieses Produktes waren
vergleichbar mit objekt-orientierten Programmiersprachen wie Java oder C\#.
\newline\newline
Professionelle Programmierer interessierten sich immer mehr für ActionScript
als Entwicklungstool, jedoch zeigte sich, dass trotz der vergleichbaren
Möglichkeiten ActionScript in der Performance noch nicht mit seiner Konkurrenz
mithalten konnte. Der Grund dafür war, dass jede Version von ActionScript auf
der vorherigen aufbaute. Der Flash Player war ursprünglich nicht für
aufwändige Anwendungen und komplexe Spiele konzipiert, dennoch begannen
Entwickler den Flash Player dafür zu nutzen. Damit wurde es klar, dass eine
neue Version von ActionScript von Beginn an neu entwickelt werden musste um
den Erwartungen gerecht zu werden.
\newline\newline
2006 wurde von Adobe ActionScript 3.0 vorgestellt, welches bedeutend mehr neue
Funktionalitäten und eine starke Performancesteigerung mit sich brachte. Flash
CS3 war die erste Version von Flash, die ActionScript 3.0 unterstützte. Flash
CS4 fügte neue Funktionen zu ActionScript 3.0 hinzu, unter anderem neue 3D
Möglichkeiten, Kontrollfunktionen für Animationen und ActionScript Klassen für
Adobe AIR. Flash CS5 und Flash CS6 erweiterten die Funktionalitäten unter
anderem mit der Unterstützung von Controller und anderen Geräten, inklusive
Multitouch-Funktionen und Touch-Screen Geräten.
\newline\newline
Die aktuelle Version des Adobe Flash Players (Version 11) wurde im
Oktober 2011 veröffentlicht.
% http://helpx.adobe.com/en/flash-player/release-note/fp_119_air_39_release_notes.html
% http://www.adobe.com/support/documentation/en/flashplayer/releasenotes.html

%alexander benz - begin

\section{Features}
Adobe Flash bietet für Entwickler eine Fülle an Funktionen und Features.
An dieser Stelle soll ein Überblick über die bedeutsamsten Funktionalitäten
gegeben werden.

\subsection{Grafiken und Animation}
\begin{itemize}
	\item{2D Zeichentools}
	\item{Animationstools}
	\item{3D Unterstützung}
	\item{3D Transformationen}
	\item{SVG (Scaleable Vector Graphics)}
	\item{Skalierbare Inhalte}
	\item{Text-Engine}
	\item{Filtereffekte (z.B. Weichzeichnen)}
	\item{Präsentationen}
\end{itemize}
Adobe Flash bietet Entwicklern eine große Auswahl an Werkzeugen zur
Erstellung von Grafiken und Animationen. Die umfangreiche
Entwicklungsumgebung ermöglicht mittels integrierter Tools wie den
Zeichentools, der Zeitleiste oder der inversen Kinematik die einfache
Erstellung von Grafiken und Animationen, ohne dabei ActionScript Code verwenden
zu müssen. Auch Texte lassen sich innerhalb der Entwicklungsumgebung
bearbeiten und modifizieren und mühelos in die fertige Anwendung integrieren.
Vordefinierte Effekte können auf eine Anwendung bequem per Knopfdruck
angewendet werden. Allerdings ist es auch möglich alle vordefinierten
Anwendungen mittels ActionScript anzusprechen, selbst zu programmieren oder
auch mit weiteren Funktionalitäten zu erweitern. Durch die Nutzung von
ActionScript wird auch die Programmierung von komplexeren Anwendungen
ermöglicht. Adobe legt großen Wert auf die bisherigen Möglichkeiten im 3D-
Bereich und möchte diese noch weiter ausbauen.

\subsection{Schnittstellen (APIs)}
\begin{itemize}
	\item{Zugriff auf das Filesystem}
	\item{Web Sockets}
	\item{Geolocation}
	\item{Datenaustausch}
\end{itemize}
Adobe Flash bietet eine Vielzahl an Schnittstellen an, mit denen Systeme und
Daten angesprochen werden können. Mit dem Zugriff auf das Filesystem des
Computers ist es möglich, Daten zu laden, zu bearbeiten und zu speichern. Der
Zugriff auf die Hardware des Computers wird ebenfalls durch Schnittstellen
ermöglicht. Dadurch kann zum Beispiel das Mikrofon, die Kamera oder die
Sensoren zur Standortbestimmung per Geolocation des Endgeräts genutzt werden.
Für den Austausch von Daten zwischen Anwendungen oder dem Computersystem bietet
Adobe Flash den Entwicklern weitere Schnittstellen für XML und andere
Dateiformate an.

\subsection{Multimedia}
\begin{itemize}
	\item{Videounterstützung}
	\item{Audiounterstützung}
	\item{Streaming}
\end{itemize}
Adobe Flash bot als eine der ersten Technologien die Möglichkeit, Video- und
Audio-Elemente in Anwendungen zu integrieren. Die Unterstützung von
verschiedenen Kodierungsverfahren und Optionen ermöglichen die einfache und
vollständige Kontrolle über die Qualität und Datenmenge. Als große Stärke
von Adobe Flash zeichnet sich die einfachen Realisierung von direkten
Live-Streams ab.

\subsection{Sonstiges}
\begin{itemize}
	\item{
	Kompatibilität und Unterstützung auf unterschiedlichen Plattformen und
	Endgeräten
	}
\end{itemize}
Adobe Flash unterstützt alle gängigen Desktop-Betriebssysteme wie Apples
MacOS, Microsoft Windows und Linux. Lediglich das mobile Betriebssystem
Android kann eine vereinfachte Version des Adobe Flash Players nutzen und
damit Inhalte vollständig wiedergeben. Für die nicht unterstützten mobilen
Betriebssysteme wie z.B. iOS können Flash Inhalte in HTML5 umgewandelt werden.
Allerdings ist das Ergebnis aufgrund der begrenzten technischen Möglichkeiten
nicht ident mit der originalen Flash Version.

\subsection{Flash Spezifisch}
\begin{itemize}
	\item{ActionScript}
	\item{Eigene Entwicklungsumgebungen (Adobe Flash, Flex)}
\end{itemize}
Die in Adobe Flash integrierte objektorientierte Programmiersprache
ActionScript ermöglicht die Realisierung umfangreicher und komplexer
Anwendungen. Adobe bietet mit Adobe Flash eine umfangreiche aber teure
Entwicklungsumgebung mit grafischer Oberfläche und integrierten Tools sowie
Syntaxhervorhebungen und Programmierhilfen für ActionScript Code.
Mit Flex gibt es auch eine kostenfreie Entwicklungsumgebung mit Tools und
Hilfen für die Entwicklung von Anwendungen mittels ActionScript, allerdings
ohne grafische Oberfläche und Werkzeuge zur Erstellung und Bearbeitung
von Grafiken und Animationen.

\section{Aktueller Einsatz}
% Adobe 2010 99% - noch vor dem Wandel zu HTML5
Trotz propritärer Technologien hat Adobe es geschafft, ihren Flash Player auf
99\% Prozent aller Systeme unterzubringen. Gründe dafür waren die Fähigkeiten
von Flash, mit denen Ziele erreicht werden konnten, die mit Hilfe von offenen
Standards nicht möglich waren. In den Anfangstagen von Flash wurde es vor
allem für Introanimationen, welche auf Startseiten eingesetzt wurden,
interaktive Navigationen, die auf die Aktionen des Users mit z.B. Animationen
reagierten, oder für die allseits bekannten Werbebanner genutzt. Die
Erstellung ähnlicher Funktionen mit standardkonformen Technologien wie
HTML 4.01, JavaScript oder CSS, war im Gegensatz zu Flash, nur schwer oder
überhaupt nicht realisierbar. Selbst heute werden komplette Webseiten, Spiele
und vor allem Werbebanner mittels Flash realisiert. Ein weiterer Vorteil von
Flash ist das einfache Erlernen der Programmiersprache und Bedienen der
Entwicklungsumgebung, wodurch auch Neueinsteiger sehr schnell zu ansehnlichen
Ergebnissen kommen.
\newline\newline
Für sich betrachtet vermitteln die Verbreitung des Adobe Flash Player und die
genannten Fakten den Eindruck, dass Flash die unangefochtene Nummer Eins unter
den Webtechnologien sein muss. Jedoch gilt es zu beachten, dass diese
Informationen aus einer Zeit stammen, in der der Wandel zu HTML5 gerade erst
begonnen hat, was zur Folge hat, dass aktuelle Zahlen anders aussehen würden.
Dennoch verdeutlichen die Verbreitung und die genannten Fakten was für eine
starke Plattform Adobe Flash bietet und in Zunkunft weiterhin bieten wird.
\newline\newline
HTML5 hat es schwer, Flash in sämtlichen Bereichen zu verdrängen. Einige
Anbieter, unter anderem YouTube, haben bereits mit der Umstellung ihrer
Inhalte begonnen. In Bereichen in denen die Features der HTML5 Spezifikation
bereits in einem fortgeschrittenen Entwicklungsstadium sind, können dafür
bereits genutzt werden. In anderen Bereichen wird Flash auch in Zukunft
unumgänglich bleiben, da es Features enthält, die mittels HTML5 nur
umständlich oder überhaupt nicht umsetzbar sind. Dies sind zum Beispiel die
Bereiche des gesicherten Online-Streamings oder die Erstellung von Rich
Internet Applications (RIA).
\newline\newline
Ein großer Vorteil von Flash stellt die hohe Verbreitung des Adobe Flash
Player dar. Wie es aus dem ersten Absatz des Kapitels ersichtlich ist,
kann dieser auf fast jedem Computer gefunden werden. Anders als HTML5 ist man
dabei nicht von den Browserherstellern abhängig, um neue Funktionen einzubauen
und zu unterstützen. Mit der Installation des Adobe Flash Player, sind
lediglich eigene Updates der Anwendungen für Neuerungen und Inhalte notwendig.
Bei Betriebssystemen ohne Unterstützungen des Flash Players sieht das wiederum
anders aus. Als Beispiel verzichtet Apple vollständig auf die Unterstützung
von Flash auf sämtlichen mobilen Endgeräten.
\newline\newline
Veröffentlicht als proprietäre Software, wurden seitdem große Teile von
ActionScript 3.0 durch Adobe selbst, Drittanbietern oder OpenSource Projekte
öffentlich zugänglich gemacht, wodurch es mittlerweile möglich ist,
unabhängig von Adobe Produkten Flash Inhalte zu erstellen. Mit
Entwicklungsumgebungen wie Flex und Add-Ons für Eclipse wird versucht, für
Entwickler immer bessere Tools zur Erstellung von Flash Anwendungen anzubieten.
\newline\newline
Trotz möglicher aufkommender Konkurrenten lässt sich,
aufgrund der hohen Nutzerzahl und der vielen Vorteile, mit Sicherheit sagen,
dass Flash weiterhin ein wichtiger Bestandteil des World Wide Webs bleiben
wird.

\section{ActionScript 3.0}
ActionScript ist die von Adobe für Flash entwickelte Programmiersprache.
Die aktuelle Version 3.0 ist eine vollwertige objektorientierte
Programmiersprache.
\newline\newline
Erstmals erschien die Sprache mit dem Flash Player 4 im Jahr 1999. Die
volle Unterstützung kam erst im Jahr 2000. Die erste Version beinhaltete
einfache Befehle, mit denen die Steuerung von Flash-Elementen möglich war.
Nutzer hatten damit erstmals die Möglichkeit frei über die Interaktionen
der Elemente zu entscheiden.
\newline\newline
Version 2.0 von ActionScript erschien im Jahr 2003 mit dem Flash Player 7
und erweiterte die Steuerungsmöglichkeiten von Flash Inhalten deutlich.
Auch die ersten Ansätze von objektorientierter Programmierung und Vererbung
von Klassen wurden implementiert. Im Vergleich zu anderen Skriptsprachen wie
JavaScript war der Funktionsmöglichkeiten von ActionScript zu diesem Zeitpunkt
bereits umfangreicher. So konnten Entwickler bereits komplexe multimediale
Inhalte erstellen, die vom Nutzer, dank hoher Interaktivität, welche mittels
ActionScript ermöglicht wurde, gesteuert werden konnten.
\newline\newline
2006 erschien die bis heute aktuelle Version von ActionScript: 3.0. Mit der
Version 3.0 wurden auch die meisten Neuerungen eingeführt. So unterstützt
ActionScript 3.0 die von anderen Programmiersprachen, wie z.B. Java oder C,
bekannten Paradigmen der klassenbasierten Objektorientierung und Typisierung
zur Laufzeit vollständig. Mittels einer virtuellen Maschine, in der
ActionScript Code nun ausgeführt wird, wird garantiert, dass der Code auf
jeder Plattform gleich ausgeführt wird. Im Hinblick auf mobile Plattformen
und deren starkem Wachstum ein zukunftssicherer Schritt.
\newline\newline
Entwicklern werden verschiedene Bibliotheken von Adobe mit den
Funktionalitäten von Flash angeboten, sodass auch außerhalb ohne der Nutzung
der offiziellen Entwicklungsumgebung Code generiert werden kann. So kann auch
mit wenig Mitteln komplexere Projekte auf der Basis von ActionScript
realisieren, die weit über den Einsatz im Flash Player als Browser Plug-In
hinaus gehen. So hat das Aufkommen von ActionScript 3.0 besonders in den
letzten Jahren dafür gesorgt, dass sich Flash im Markt der Rich Internet
Applications eine starke Position sichern konnte.

% alexander benz - end

%\section{Flash Verfügbarkeit auf mobilen Geräten}

%\section{Möglichkeiten}
%\subsection{Spiele}
%\subsection{Animationen}
%\subsection{Entertainment}

%\section{Die Einflüsse von Flash auf das Web}
%\subsection{Die Wurzeln und Entwicklung von Flash}
%\subsection{Stärken von Flash}
%\subsection{Wichtiger Beiträger für die Entwicklung des Webs}
\section{Apples Argumente gegen Flash}
Eine oft verwendete Aussage und immer wiederkehrendes Streitthema in vielen
Diskussionen ist sicherlich, ob HTML5 langfristig Flash als
Webentwicklungswerkzeug ablöst. Apple spielt hierbei eine große Rolle. Das
Unternehmen verzichtete seit der Einführung ihrer iDevices (iPhone, iPod
touch, iPad) auf die Nutzung von Flash. Vor allem die fehlende Möglichkeit,
Flash auf dem iPad zu verwenden, führte vermehrt zu Kritik. Der User sei
eingeschränkt und ohne Flash handelt es sich nicht um das "`echte Internet"',
welches Apple mit ihren Produkten anpreist. Der verstorbene Steve Jobs, CEO
von Apple, hat daraufhin einen offenen Brief verfasst, der den Nutzern
erklärt, warum die Firma auf Flash verzichtet hat und auch weiterhin darauf
verzichten wird. Steve Jobs Hauptargumente waren und gelten noch bis heute:
\begin{itemize}
	\item[1]{
		Flash ist ein 100 Prozent proprietäres System von Adobe, welches trotz der
		allgemeinen Verfügbarkeit des Flash Players die Zügel in der Hand hat.
		Adobe könnte ganz alleine über die Richtung entscheiden, in die Flash
		geht oder die Preisgestaltung ändern. Apple möchte das Internetstandards
		offen sein sollen, selbst wenn es auf die eigenen Geräte zutrifft.
	}
	\item[3]{
		Flashprodukte sind laut einer Analyse des Softwarehauses Symantec mit
		am anfälligsten für Sicherheitsprobleme. Sie seien einer der Hauptgründe
		für Abstürze auf dem Mac, auch die Performance auf Mobiltelefonen würde
		zu wünschen übrig lassen.
	}
	\item[4]{
		Flash beansprucht die Akkulaufzeit auf mobilen Geräten, da das Dekodieren
		von Videomaterial über die Software laufen muss. H.264 würde über die
		Hardware dekodiert werden.
	}
	\item[5]{
		Flash wurde nicht für Touchscreens geschaffen. Viele Webseiten müssten
		ihre Seite komplett neu konzipieren und entwickeln. Entwickler sollten
		aber dann doch auf modernere Technologien wie HTML5, CSS3 und
		Javascript zurückgreifen.
	}
\end{itemize}
Ein kommerzieller Hersteller wirft somit einem anderen kommerziellen Hersteller
vor, zu proprietär zu sein - das mag durchaus merkwürdig erscheinen. Vor allem
weil die Kunden von Apple auch im eigenen System eingezäunt und der Kontakt zu
anderen Systemen über Schnittstellen so gering wie möglich gehalten wird und
bei Adobe nur von der theoretischen, aber unwahrscheinlichen Möglichkeite
ausgegangen wird, dass diese die Vormachtstellung ihres Browserplugins
missbrauchen könnten.
\newline\newline
Auch die Performanceprobleme scheinen bei Googles Smartphone Betriebssystem
Android seit der Version 2.2 gelöst worden zu sein. Denn seitdem läuft
Flash stabil und schnell, ohne den Akku zu sehr zu beanspruchen.
\newline\newline
Die Behauptung das Flash nicht für Touchscreen ausgelegt worden ist, ist an
und für sich wahr, das gleiche gilt allerdings auch für HTML. Auch hier gibt
es eben so viele mausbasierende Aktionen, wie z.B. Rollover Animationen.
% =VERGLEICH

% Links
% http://venturebeat.com/2012/01/31/html5-versus-flash-infographic/
% http://www.periscopic.com/#/news/2011/05/our-research-into-flash-and-html5-which-one-is-right-for-your-project/
% http://www.actionscript.org/resources/blogs/53/A-good-comparison-between-HTML5-and-Flash.html
% http://blog.accusoft.com/posts/2012/october/html5-vs-flash-what-do-you-need-to-know-part-1.html
% http://blog.accusoft.com/posts/2013/january/html5-vs-flash-infographic.html
% http://blog.accusoft.com/posts/2012/october/html5-vs-flash-what-do-you-need-to-know-part-2.aspx
% http://readwrite.com/2010/03/09/does_html5_really_beat_flash_surprising_results_of_new_tests#awesm=~obelLBPH3UgwgC

\chapter{HTML5 Canvas und Adobe Flash: Vergleich und mögliche Auswirkungen}

\section{Vergleich von HTML5 Canvas und Adobe Flash}
Proprietär vs Standard
\subsection{Verfügbarkeit}
\subsection{Audio und Video}
\subsection{Animation}
\subsection{Spiele}
\subsection{Werbung}
\subsection{Web-Applikationen}

\section{Mögliche Auswirkungen}
\subsection{Der Wandel von dem allgegenwärtigen Flash zu den neuen Webstandards}
\subsection{Die Zukunft von HTML5 und Flash}
\subsection{Unternehmen und ihre Einstellung zu neuen Ideen}

\chapter{Auswirkungen}
\section{Browser}
\subsection{Chrome und Safari (WebKit)}
Chrome von Google und Safari von Apple bauen beide auf der HTML-Rendering-Engine
WebKit zur Darstellung von Webinhalten auf. Google Chrome konnte zunehmend mit
dem bis dato meist genutzen Webbrowser Internet Explorer konkurrieren, bis es
nach den Angaben des globalen Statistiksunternehmens StatCounter erstmals im Mai
2012, weltweit die Spitzenposition einnehmen konnte. Mit einem
durchschnittlichen Anteil von 35\% für Google Chrome und 7\% für Apple Safari
stellen diese die beliebtesten Webbrowser dar.
% http://gs.statcounter.com/#browser-ww-monthly-201204-201304-bar
Im Vergleich decken beide Browser den größten Teil der HTML5 Unterstützung ab.
% Vergleiche HTML5 Unterstützung - http://html5readiness.com/

Die WebKit Engine wurde auf der Grundlage des KDE-Projekts KHTML als Open Source
Projekt von Apple entwickelt. Apple ist sehr daran interessiert, dass HTML5 so
schnell wie möglich auf allen Geräten verfügbar ist, um den Benutzern ein
uneingeschränktes Internet anbieten zu können. Auch die Mobile Safari Variante
auf den iDivices basieren auf der WebKit-Engine.

\subsection{Firefox (Gecko)}
Der Open Source Webbrowser Firefox von Mozilla stellt im deutschsprachigen Raum
den meist genutzten Webbrowser dar. Weltweit betrachtet befindet sich Firefox
mit einem durchschnittlichen Anteil von 23\% hinter dem Internet Explorer von
Microsoft und Google Chrome.
Mozilla zeichnete sich schon früh als Unterstützer des HTML5 Standards ab. So
war es schon mit sehr frühen Versionen von Firefox möglich, das Video-Element zu
nutzen.

\subsection{Internet Explorer (Trident)}
Der wohl bekannteste Webbrowser Internet Explorer von Microsoft ist bereits in
der zehnten Version veröffentlicht worden. Schon mit dem Internet Explorer 9 war
Microsoft darauf bedacht die Geschwindigkeit ihres Browsers zu verbessern und es
wurde darauf verzichtet, eigene Standards durchzusetzen.
Erstmals orientierte man sich gänzlich an den Spezifikationen und unterstütze
bereits einige HTML5 Features.
So wurde das Audio/Video-Element unterstützt, wobei nur auf proprietäre Codecs
wie H.264 und MP3/AAC zurückgegriffen wurde. Seit dem Internet Explorer 10 wird
auch der freie Codec WebM unterstützt.

Microsoft hatte sich aus seinem größten Manko einen Vorteil verschafft: Dadurch,
das der Internet Explorer nur auf Windows Geräten läuft, kann er die ganze
Plattform und Hardware nutzen. Im Gegensatz zur Browsersoftware der Konkurrenz,
die auf den kleinsten gemeinsamen Nenner zurückgreifen müssen.
So kann Internet Explorers JavaScript Engine Chakra, die Leistung von Multicore
Prozessoren ausnutzen.

Mit dem Internet Explorer 10 hat Microsoft einiges aus der Vergangenheit
aufgeholt, allerdings sollten diese neuen Features nicht überbewertet werden,
denn die Einsatzfähigkeit von HTML5 zeichnet sich weniger durch die neuen
Browser, sonder viel mehr durch die Anzahl an veralteten Browser aus. Solange
die alten Versionen des Internet Explorer im Gebrauch sind können viele
Neuheiten von HTML5 nicht sinnvoll und ohne Fallbacklösung verwendet werden.
\chapter{Zukunftsaussicht}
% !!!!!!! komplett abgeschrieben von report_final !!!!!!!
\section{Wird HTML5 Flash ersetzen?}

Steve Jobs deutet in seinem offenen Brief immer wieder an, dass
Flash veraltet und HTML5 die Zunkunft ist. Dies ist insofern falsch,
da HTML5 mit seiner bisherigen Browserunterstützung noch zu weit
entfernt ist, um mit dem Flash Player mitzuhalten. Unrecht hat 
Steve Jobs mit der Aussage der fehlenden Offenheit allerdings nicht.
Flash soll eingesetzt werden, wenn das Ziel mit einem offenen 
Standard nicht erreicht werden kann. Mit Flash kann qualitativ
hochwertige Leistung erbracht werden. Aufgrund seiner
Verbreitung ist es ein inoffizieller Standard.
\newline\newline
HTML5 steckt noch in den Kinderschuhen und hat es schwer, mit Flash
zu konkurrieren. Denn auch wenn Flash geschlossen ist, ist die Verbreitung
des Flashplayer immer noch höher als die Verbreitung von HTML5-fähigen
Browsern. Außerdem hat es ein Unternehmen wie Adobe wesentlich
einfacher, ein neues Feature in ihren Player einzuführen, als Konsortien wie
die W3C oder WHATWG, welche jedes Feature auf ihre Art demokratisch 
einführen und dann auf Browsersupport hoffen.
\newline\newline
Im Videbereich kann sich HTML5 zwar schon präsentieren, doch auch auf
YouTube wird der Flash Player weiterhin eine wichtige Rolle spielen, da es
noch an wichtigen Funktionen in HTML5 Video fehlt, wie z.B. an einem
nativen Vollbild, einem Schutz vor Download zur Wahrung der Urheberrechte
oder der Kommunikation mit dem Nutzer. YouTube bietet dem Nutzer die 
Möglichkeit, direkt über die Webcam ein Video hochzuladen, was ohne
Flash nicht möglich wäre.
\newline\newline
Entscheidend dafür, was demnächst in HTML5 und was weiterhin in Flash
umgesetzt wird, werden sicher die Interessen der Webentwickler sein. 
Während der Anwender nämlich nur die Technik nutzt und es ihm
egak sein wird wie sein Video oder seine Webseite läuft, liegt es an dem
Entwickler, ob er überhaupt in relativ funktionsarmen Javascript statt
des sehr umfangreichen ActionScript 3 programmiert. Große Unternehmen
wie Yahoo, Facebook oder Google könnten ihre Vorteile wiederum
aus Webstandards ziehen, da sie selbst eigene Anforderungen an eine
neue Spezifikation einbringen können, ohne dass sie von einem 
kommerziellen Anbieter wie Adobe abhängig sind.
\newline\newline
Die Frage, ob HTML5 den Flash Player in Zukunft ablösen wird, bleibt also offen.
Flash wird uns sicherlich noch eine Weile erhalten bleiben, da mit HTML5
in seiner jetztigen Form und Verbreitung noch nicht diesselbe Masse
erreicht werden kann. Sicher ist auch, dass keines der beiden Programme
das andere vollständig ersetzen kann, weshalb man auch nicht von
Konkurrenz sprechen sollte, sondern von gegenseitiger Ergänzung.
\newline\newline
So eignet sich Flash in Zukunft vo allem für Elemente, die eine hohe Performance
benötigen und auf eine umfrangreiche Scriptbibliothek (ActionScript 3)
zugreifen können, wie z.B. Spiele, Rich Internet Applications, komplexe
3D-Animationen, Audio/Videoplayer, Simulationen oder umfangreiche
Präsentationen mit hohem Audio- und Video-Anteil. Die HTML5/JavaScript
Funktionalitäten eigenen sich hingegen für interaktive Webseitenelemente
(Accordions, Tooltips, Dropdown-Menüs, Tabs, uvm), Formularvalidierung,
Chats oder einfache Präsentationen.

\section{HTML6}

Die Unterstützung von HTML5 ist noch nicht gewährleistet - hier und da wird
an allen Ecken noch gebastelt. Aber trotzdem hat im Januar der 
Google-Mitarbeiter und das WHATWG-Mitglied Mark Pilgrim im WHATWG-Blog
bereits einen Einblick in HTML6 gewährt. Unter anderem wurde ein neues
Element mit dem Namen \em{<device>} vorgestellt, welches z.B.
Webcam-Konferenzen ohne den Umweg über Flash ermöglichen würde.
Aber ob diese Spezifikation auch wirklich "`HTML6"' heißen wird, ist unklar.
Weiter schreibt Pilgrim nämlich:

\begin{quote}
	The next version of HTML doesn't have a name yet. In fact,
	it may never have a name, because the working group is
	switching to an unversioned development model. Various
	parts of the specification will be at varying degrees of 
	stability, as noted in each section. But if all goes according 
	to plan, there will never be One Big Cutoff that is frozen in
	time and dubbed "`HTML6"'. HTML is an unbroken line
	stretching back almost two decades, and version numbers
	are a vestige of an older development model for standards
	that never really matched reality very well anyway.
	HTML5 is so last week. Let's talk about what's next.
\end{quote}

Pilgrim sagt also, dass die Vergabe von Versionsnummern veraltet ist und 
besonders für den Veröffentlichungsprozess von HTML-Standards nicht funktioniert.
Ebenfalls interessant ist, dass der HTML5 WHATWG-Entwurf seit diesem
Eintrag seinen Namen immer wieder von "`WHATWG HTML (including HTML5)"'
zu "`HTML5 (including next generation addition still in development)"' ändert.
\newline\newline
All dies könnte darauf hindeuten, dass HTML5 wirklich die letzte versionierte
HTML-Spezifikation ist. Die Zukunft heißt also nicht "`HTML6"', sondern
einfach nur "`HTML"' oder "`HTML5 mit zusätzlichen Elementen"'. Das
würde die Entwicklung vorantreiben. Es muss nicht mehr ein ganzes
Paket an Neuerungen herausgebracht werden, auch einzelne Elemente
können eingeführt werden. Schließlich sind es die Browserhersteller, die
im Endeffekt HTML5 entwerfen. Dieses könnte eventuell die Existenz
der W3C in Zukunft überfküssig machen. Die beiden Spezifikationen sind
beinahe identisch und alles, was sich in der W3C-Version findet, steht
auch in der WHATWG-Version. Iam Hickson hat allerdings in einer E-Mail
kurz erwähnt, dass geplant ist, die WHATWG-Spezifikation in Zukunft ohne
engere Absprachen mit dem W3C zu erweitern. Das neue \em{<device>}
Element ist also nur ein Anfang.

\begin{quote}
	I've given up trying to keep a WHATWG copy of the
	HTML5 spec that matches what the W3C publish [...]
\end{quote}

Auch Buchautor Peter Kröner prophezeit, dass es eventuell nicht einmal zu
einer W3C-Version von HTML5 kommt.

\begin{quote}
	Die spannende Frage ist, ob es den "`One Big Cutoff"', sprich den
	Recommendation-Status für HTML5 zu einem Zeitpunkt, an dem
	es noch jemanden interessiert, zumindest für HTML5 geben wird.
	Ich glaube nicht.
\end{quote}

Dies begründet Kröner damit, dass die Ausarbeitung von HTML5 seit
dem Beginn der Entwicklung im Jahr 2004 bis heute (2013) neun
Jahre gedauert hat und inzwischen zum großen Teil implementiert
und benutzbar ist. Trotzdem wird aber von einem Recommendation-Status
gesprochen, der erst im Jahr 2022 angesetzt wird. Wenn im Jahr 2013
nun aber bereits von einem HTML6 \em{<device>} Element geredet wird, 
könnte bereits im Jahr 2016 ein Standard namens HTML6 entwickelt und
großteilig in den Browsern implementiert sein und infolgedessen den
Stand von HTML5 im Jahr 2013 haben. Also ein Datum, was neun Jahre vor
dem Recommendation-Status von HTML5 liegt und im Jahr 2022 wahrscheinlich 
niemanden mehr interessieren wird. Auch aus dieser Sicht scheint ein
versionsloses HTML-Modell mehr Sinn zu machen und ist laut
Kröner bereits auf dem Weg:

\begin{quote}
	Wenn die WHATWG obendrein plant, mit HTML6 ein neues,
	versionsloses Entwicklungsmodell einzuführen, betrifft dies
	bereits HTML5. Die Spezifikationen der WHATWG sind schon
	jetzt ein Mischmasch aus alten HTML-Features, Neuerungen,
	die sich auf einen breiten Konsens stützen hoch kontroversen
	Ideen wie HTML5-Microformats und dem \em{<device>}
	Element. Der unversionierte Ausbau ist also bereits im Gange."
\end{quote}
\include{tex/fazit}


% Literaturverzeichnis
% Das Literaturverzeichnis kann auch nach einem allfälligen Anhang positiioniert
% werden (siehe "`Leitfaden für Bachelor- und Diplomarbeiten"', Version 2.0,
% Abschnitt 2.9).

% Möglichkeit 1: Erzeugung des Literaturverzeichnisses mit BibTeX:
% Die Quellen sind in der Datei *.bib (hier Literatur.bib) einzugeben. Danach
% muss diese Vorlage einmal geTeXt werden, dann BibTeX angewendet werden und
% anschliessend nochmals zweimal geTeXt werden.
% Im Text erfolgt die Zitierung mit dem Anker-Schlüsselwort, z.B. \cite{kop05}.
\bibliographystyle{IEEEtran}
\nocite{*}
\bibliography{Literatur}


% Möglichkeit 2: Erzeugung eines Literaturverzeichnisses ohne BibTeX:
%\begin{thebibliography}{99}
%\bibitem[kop05]{kop05}
%H.~Kopka, {\em LaTeX, Band 1: Einführung}, Pearson Studium, München,
%3.~Auflage, 2005.
%\bibitem[knu98]{knu98}
%F.~Mittelbach, M.~Goossens, J.~Braams, D.~Carlisle, and Ch. Rowley, {\em The
% LaTeX Companion},
%Addison-Wesley, 2nd edition, 2004.
%\end{thebibliography}

% Abbildungsverzeichnis
\listoffigures
\addcontentsline{toc}{chapter}{Abbildungsverzeichnis} % fügt den Eintrag
% "Abbildungsverzeichnis" im Inhaltsverzeichnis hinzu
\newpage

% Tabellenverzeichnis
\listoftables
\addcontentsline{toc}{chapter}{Tabellenverzeichnis} % fügt den Eintrag
% "Tabellenverzeichnis" im Inhaltsverzeichnis hinzu
\newpage

% Abkürzungsverzeichnis
% Bei Verwendung der Dokumentklasse "scrartcl" ist der Befehlt
% \addchap{Abkürzungsverzeichnis} durch
% \addsec{Abkürzungsverzeichnis} zu ersetzen
\addchap{Abkürzungsverzeichnis}
\hspace{-17mm}\begin{tabular}{>{\raggedleft}p{0.2\linewidth} p{0.75\linewidth}
p{0.1\linewidth}}
%#
3G & 3rd Generation Mobile Telecommunications \\
%a
Ajax & Asynchronous JavaScript and XML \\
AS3 & ActionScript3 \\
API & Application Programming Interface (Programmierschnittstelle) \\
%b
%c
CSS & Cascading Style Sheet \\
CSS3 & Cascading Style Sheet Level 3 \\
%d
DOM & Document Object Model \\
%e
%f
%g
%h
HTML & HyperText Markup Language \\
HTML5 & HyperText Markup Language Version 5 \\
%i
%j
JS & JavaScript \\
JSON & JavaScript Object Notation \\
%k
%l
%m
%n
%o
%p
%q
%r
RIA & Rich Media Application \\
%s
SGML & Standard Generalized Markup Language \\
SQL & Structured Query Language \\
SVG & Scaleable Vector Graphics \\
%t
%u
UA & User Agent \\
URL & Uniform Resource Locator \\
%v
%w
W3C & World Wide Web Consortium \\

\end{tabular}


% Anhänge
\begin{appendix}
\chapter[Erster Anhang]{Überschrift des ersten Anhangs}

Text Text Text Text Text Text Text Text Text Text Text Text Text Text Text Text
Text Text Text Text Text Text Text Text ...


\chapter[Zweiter Anhang]{Überschrift des zweiten Anhangs}

Text Text Text Text Text Text Text Text Text Text Text Text Text Text Text Text
Text Text Text Text Text Text Text Text ...

\end{appendix}


\end{document}
