\chapter{Zukunftsaussicht}

\section{Wird HTML5 Flash ersetzen?}

Steve Jobs deutet in seinem offenen Brief immer wieder an, dass
Flash veraltet und HTML5 die Zunkunft ist. Dies ist insofern falsch,
da HTML5 mit seiner bisherigen Browserunterstützung noch zu weit
entfernt ist, um mit dem Flash Player mitzuhalten. Unrecht hat
Steve Jobs mit der Aussage der fehlenden Offenheit allerdings nicht.
Flash soll eingesetzt werden, wenn das Ziel mit einem offenen
Standard nicht erreicht werden kann. Mit Flash kann qualitativ
hochwertige Leistung erbracht werden. Aufgrund seiner
Verbreitung ist es ein inoffizieller Standard.
\newline\newline
HTML5 steckt noch in den Kinderschuhen und hat es schwer, mit Flash
zu konkurrieren. Denn auch wenn Flash geschlossen ist, ist die Verbreitung
des Flashplayer immer noch höher als die Verbreitung von HTML5-fähigen
Browsern. Außerdem hat es ein Unternehmen wie Adobe wesentlich
einfacher, ein neues Feature in ihren Player einzuführen, als Konsortien wie
die W3C oder WHATWG, welche jedes Feature auf ihre Art demokratisch
einführen und dann auf Browsersupport hoffen.
\newline\newline
Im Videbereich kann sich HTML5 zwar schon präsentieren, doch auch auf
YouTube wird der Flash Player weiterhin eine wichtige Rolle spielen, da es
noch an wichtigen Funktionen in HTML5 Video fehlt, wie z.B. an einem
nativen Vollbild, einem Schutz vor Download zur Wahrung der Urheberrechte
oder der Kommunikation mit dem Nutzer. YouTube bietet dem Nutzer die
Möglichkeit, direkt über die Webcam ein Video hochzuladen, was ohne
Flash nicht möglich wäre.
\newline\newline
Entscheidend dafür, was demnächst in HTML5 und was weiterhin in Flash
umgesetzt wird, werden sicher die Interessen der Webentwickler sein.
Während der Anwender nämlich nur die Technik nutzt und es ihm
egak sein wird wie sein Video oder seine Webseite läuft, liegt es an dem
Entwickler, ob er überhaupt in relativ funktionsarmen Javascript statt
des sehr umfangreichen ActionScript 3 programmiert. Große Unternehmen
wie Yahoo, Facebook oder Google könnten ihre Vorteile wiederum
aus Webstandards ziehen, da sie selbst eigene Anforderungen an eine
neue Spezifikation einbringen können, ohne dass sie von einem
kommerziellen Anbieter wie Adobe abhängig sind.
\newline\newline
Die Frage, ob HTML5 den Flash Player in Zukunft ablösen wird, bleibt also offen.
Flash wird uns sicherlich noch eine Weile erhalten bleiben, da mit HTML5
in seiner jetztigen Form und Verbreitung noch nicht diesselbe Masse
erreicht werden kann. Sicher ist auch, dass keines der beiden Programme
das andere vollständig ersetzen kann, weshalb man auch nicht von
Konkurrenz sprechen sollte, sondern von gegenseitiger Ergänzung.
\newline\newline
So eignet sich Flash in Zukunft vo allem für Elemente, die eine hohe Performance
benötigen und auf eine umfrangreiche Scriptbibliothek (ActionScript 3)
zugreifen können, wie z.B. Spiele, Rich Internet Applications, komplexe
3D-Animationen, Audio/Videoplayer, Simulationen oder umfangreiche
Präsentationen mit hohem Audio- und Video-Anteil. Die HTML5/JavaScript
Funktionalitäten eigenen sich hingegen für interaktive Webseitenelemente
(Accordions, Tooltips, Dropdown-Menüs, Tabs, uvm), Formularvalidierung,
Chats oder einfache Präsentationen.

\section{HTML6}

Die Unterstützung von HTML5 ist noch nicht gewährleistet - hier und da wird
an allen Ecken noch gebastelt. Aber trotzdem hat im Januar der
Google-Mitarbeiter und das WHATWG-Mitglied Mark Pilgrim im WHATWG-Blog
bereits einen Einblick in HTML6 gewährt. Unter anderem wurde ein neues
Element mit dem Namen \em{<device>} vorgestellt, welches z.B.
Webcam-Konferenzen ohne den Umweg über Flash ermöglichen würde.
Aber ob diese Spezifikation auch wirklich "`HTML6"' heißen wird, ist unklar.
Weiter schreibt Pilgrim nämlich:

\begin{quote}
	The next version of HTML doesn't have a name yet. In fact,
	it may never have a name, because the working group is
	switching to an unversioned development model. Various
	parts of the specification will be at varying degrees of
	stability, as noted in each section. But if all goes according
	to plan, there will never be One Big Cutoff that is frozen in
	time and dubbed "`HTML6"'. HTML is an unbroken line
	stretching back almost two decades, and version numbers
	are a vestige of an older development model for standards
	that never really matched reality very well anyway.
	HTML5 is so last week. Let's talk about what's next.
\end{quote}

Pilgrim sagt also, dass die Vergabe von Versionsnummern veraltet ist und
besonders für den Veröffentlichungsprozess von HTML-Standards nicht funktioniert.
Ebenfalls interessant ist, dass der HTML5 WHATWG-Entwurf seit diesem
Eintrag seinen Namen immer wieder von "`WHATWG HTML (including HTML5)"'
zu "`HTML5 (including next generation addition still in development)"' ändert.
\newline\newline
All dies könnte darauf hindeuten, dass HTML5 wirklich die letzte versionierte
HTML-Spezifikation ist. Die Zukunft heißt also nicht "`HTML6"', sondern
einfach nur "`HTML"' oder "`HTML5 mit zusätzlichen Elementen"'. Das
würde die Entwicklung vorantreiben. Es muss nicht mehr ein ganzes
Paket an Neuerungen herausgebracht werden, auch einzelne Elemente
können eingeführt werden. Schließlich sind es die Browserhersteller, die
im Endeffekt HTML5 entwerfen. Dieses könnte eventuell die Existenz
der W3C in Zukunft überfküssig machen. Die beiden Spezifikationen sind
beinahe identisch und alles, was sich in der W3C-Version findet, steht
auch in der WHATWG-Version. Iam Hickson hat allerdings in einer E-Mail
kurz erwähnt, dass geplant ist, die WHATWG-Spezifikation in Zukunft ohne
engere Absprachen mit dem W3C zu erweitern. Das neue \em{<device>}
Element ist also nur ein Anfang.

\begin{quote}
	I've given up trying to keep a WHATWG copy of the
	HTML5 spec that matches what the W3C publish [...]
\end{quote}

Auch Buchautor Peter Kröner prophezeit, dass es eventuell nicht einmal zu
einer W3C-Version von HTML5 kommt.

\begin{quote}
	Die spannende Frage ist, ob es den "`One Big Cutoff"', sprich den
	Recommendation-Status für HTML5 zu einem Zeitpunkt, an dem
	es noch jemanden interessiert, zumindest für HTML5 geben wird.
	Ich glaube nicht.
\end{quote}

Dies begründet Kröner damit, dass die Ausarbeitung von HTML5 seit
dem Beginn der Entwicklung im Jahr 2004 bis heute (2013) neun
Jahre gedauert hat und inzwischen zum großen Teil implementiert
und benutzbar ist. Trotzdem wird aber von einem Recommendation-Status
gesprochen, der erst im Jahr 2022 angesetzt wird. Wenn im Jahr 2013
nun aber bereits von einem HTML6 \em{<device>} Element geredet wird,
könnte bereits im Jahr 2016 ein Standard namens HTML6 entwickelt und
großteilig in den Browsern implementiert sein und infolgedessen den
Stand von HTML5 im Jahr 2013 haben. Also ein Datum, was neun Jahre vor
dem Recommendation-Status von HTML5 liegt und im Jahr 2022 wahrscheinlich
niemanden mehr interessieren wird. Auch aus dieser Sicht scheint ein
versionsloses HTML-Modell mehr Sinn zu machen und ist laut
Kröner bereits auf dem Weg:

\begin{quote}
	Wenn die WHATWG obendrein plant, mit HTML6 ein neues,
	versionsloses Entwicklungsmodell einzuführen, betrifft dies
	bereits HTML5. Die Spezifikationen der WHATWG sind schon
	jetzt ein Mischmasch aus alten HTML-Features, Neuerungen,
	die sich auf einen breiten Konsens stützen hoch kontroversen
	Ideen wie HTML5-Microformats und dem \em{<device>}
	Element. Der unversionierte Ausbau ist also bereits im Gange."
\end{quote}