% =Einleitung
\chapter{Einleitung}
Wahrscheinlich gibt es keine andere Erfindung, abgesehen von der
Telekommunikation und des Fernsehens, die die Gesellschaft der Industrie-
und Schwellenländer entscheidender prägt als das Internet.
Das World Wide Web, der bekannteste Dienst des Internets, erlebt einen
stetigen Anstieg an Nutzerzahlen. Immer mehr und vor allem jüngere
Nutzergruppen lassen sich durch soziale Netzwerke in ihren Bann ziehen und
gerade diese Netzwerke haben dabei Einfluss auf die Art der Kommunikation von
Generationen.
\newline\newline
Für die Verbraucher beschäftigt sich ein kompletter Industriezweig damit,
ihnen das Leben zu vereinfachen, in dem im Internet sämtliche Informationen
zur Verfügung gestellt und Dienstleitungen wie diverse Einkaufsmöglichkeiten
verfügbar gemacht werden. Dabei zeigt die Tendenz, dass Nutzer ihre Daten
nicht mehr auf physikalischen Datenträgern speichern und transportieren,
sondern diese in der ``Cloud'' ablegen und somit unter der Vorraussetzung
eines Internetzugangs der Zugriff von überall ermöglicht wird.
\newline\newline
Ohne der Weiterentwicklung des Internets würde es Begriffe wie "`Soziale
Netzwerke"', "`Videokonferenz"', "`Onlinebanking"' oder "`Onlineversandhaus"'
nicht geben - Begriffe, die vor allem in den vergangenen Jahren stetig an
Bedeutung gewonnen haben. Wohin die Entwicklung des Internets dabei führen
wird, lässt sich kaum vorhersagen. Fest steht, dass das Internet ein
wesentlicher Teil im Leben der aktuellen Generation einnimmt und für
zukünftige Generationen einnehmen wird.
\newline\newline
Seit der Gründung des World Wide Web Consortiums (W3C) am 1. Dezember 1994
wird bemüht an der Umsetzung eines einheitlichen Standards und dem Festlegen
der dazu notwendigen Rahmenbedingungen gearbeitet. Seit Jahren wird weltweit
solch ein neuer Standard von Webentwicklern herbeigewünscht, um unnötigen
Programmieraufwand aufgrund unterschiedlicher Browservarianten und -versionen
zu beseitigen.
\newline\newline
Die vollständige Erarbeitung und Etablierung dieses Standards wird allerdings
noch einige Jahre in Anspruch nehmen. Dabei liegt die Problematik darin, dass
der neue Standard für die Zukunft gewappnet und die Bereitstellung älterer
Webseiten kompromisslos möglich sein muss. Dieser doppelte Anspruch verzögert
die Fertigstellung ungemein.
\vfill
% -Problemstellung
\section{Problemstellung}
Die Entwickler von Flash (früher Macromedia, jetzt Adobe) haben schon früh das
Potential von ihrem Flash Player erkannt, um plattformübergreifend Video- und
Audioinhalte im Web bereitzustellen. Ein großer Vorteil war, dass Inhalte für
Flash lediglich einmal kodiert werden mussten. Tausende von Seiten die sich für
Flash als Streamingplattform von Multimediainhalten entschieden haben,
bestätigten, bis heute, das Potential.
\newline\newline
Mit dem Erscheinen von Apples iPhone und iPod touch im Jahr 2007 und der
folgenden Entscheidung, Flash auf diesen Geräten nicht zu unterstützen, mussten
Webseitenbetreiber darauf reagieren. Viele boten Video/Audio-Streams an, die
direkt im mobilen Safari Browser wiedergegeben werden konnten. Durch die
Verwendung der H.264 Kodierung konnten die Inhalte auch über den Flash-Player
(sofern vom verwendeten Gerät unterstützt) abgespielt werden. Dadurch mussten
Inhalte weiterhin nur einmal kodiert werden, um mit möglichst vielen
Plattformen kompatibel zu sein.
\newline\newline
Die Entwickler der HTML5 Spezifikation (und u.a. Apple) sind der Meinung, dass
Browser Audio und Video nativ unterstützen sollten ohne dabei auf etwaige
Plugins zurückgreifen zu müssen. Dies hat den Vorteil, dass es keine
herstellerspezifischen Einschränkungen (wie es bei Flash der Fall ist) für den
Entwickler gibt, was er mit den Inhalten anstellt, nachdem sie in die Website
eingebettet wurden. Mittels CSS und JavaScript ist es jederzeit direkt möglich
dargestellte HTML Elemente zu manipulieren.
\newline\newline
Um ein Bild für eine Webanwendung zu erstellen wird üblicherweise eine
Grafiksoftware genutzt und anschließend in die Anwendung eingebettet. Für
Animationen wird vorwiegend Flash verwendet. Mit dem in HTML5 verfügbaren
Canvas Element können Entwickler Bilder und Animationen direkt im Browser
mittels JavaScript generieren. Mit dem Canvas Element ist es möglich, einfache
bis komplexe Formen, Graphen und auch Diagramme zu erstellen, ohne auf diverse
Bibliotheken, Flash oder ein anderes Plugin zurückgreifen zu müssen.
\newline\newline
Mit der Veröffentlichung und der Implementierung von HTML5 Features in die
neuesten Brower eröffneten sich neue Möglichkeiten um Animationen, Spiele und
Entertainment-Applikationen für das Web zu erstellen. Aufgrund vieler
auftretender Fragen ist sich kein Entwickler wirklich klar, ob HTML5 den
Platzhirschen Flash verdrängen kann:
\begin{itemize}
	\item Wo liegen die Vorteile und Nachteile beider Technologien?
	\item Ist HTML5 die Zukunft und löst Flash ab?
	\item Welche Technologie ist einfacher zu erlernen?
	\item Was kann HTML5 was Flash nicht kann (und visa-versa)?
	\item Lohnt es sich noch Flash zu erlernen?
	\item Gibt es Entwicklungsumgebungen, die das Programmieren erleichtern?
\end{itemize}
% -Motivation
\section{Motivation}
Das Internet stellt den weltweit größten Netzverbund und eine Zusammenfassung
von verschiedensten Dienstleistungen dar. Die dabei am häufigsten genutzte
Dienstleistung - das World Wide Web - hat einen sehr großen Anteil an der in
den 90er Jahren enstandenen und stetig wachsenden Beliebtheit des Internets.
Seit dem hat sich das Internet vom reinen Wissenschaftsnetz zu einem
kommerziell genutzten Netz, mit einer Vielzahl an Diensten und multimedialen
Anwendungen, entwickelt.
\newline\newline
Für viele Nutzer ist das Internet fast nicht mehr aus ihrem Leben wegzudenken.
Zu verlockend sind die Möglichkeiten der flexibleren Gestaltung des
Privatlebens, Zeitersparnisse, Kontaktpflege mit Familie und Bekannten,
Arbereitserleichterung und vielem mehr. Die Möglichkeiten, die das Internet
bietet, sind fast grenzenlos.
\newline\newline
Für eine Menge Menschen bilden soziale Plattformen, wie Facebook oder Google+,
die Grundlage ihrer sozialen Interaktion und dementsprechend hoch sind die
Ansprüche. Generell steigen mit der fortgeschrittenen Entwicklung des
Internets und dessen Technologien die Erwartungen und Ansprüche der Nutzer.
\newline\newline
Aus diesem Grund muss schon vor dem Beginn eines Projektes sehr gut
abgewogen werden, wie es umgesetzt werden soll.
Eine entscheidende Rolle spielt dabei die zu nutzende Technologie aus der
Sicht der möglichen Nutzer und deren Systemvorraussetzungen (Plattform, CPU-
Leistung, Browser-Hersteller und -version). Andererseits stellen auch
Entwicklungs- und wartungsaufwand eine nicht zu unterschätzende Größe dar.

% -Zielsetzung
\section{Zielsetzung}
Das Ziel dieser Arbeit ist es, zwei Technologien zu vergleichen, die in einem
direkten Konkurrenzkampf stehen.
\newline\newline
Adobe's Flash bietet einem Entwickler die Möglichkeit, mit einmaligen
Entwicklungsaufwand viele Nutzer zu ereichen. Einzige Vorraussetzung ist ein
installiertes Flash-Player-Plugin auf dem Computer des Seitenaufrufers.
Allerdings wird genau dieses Manko bei der Entwicklergemeinde als größtes
Hindernis angesehen. Wieso dem Nutzer die Installation eines Plugins
aufzwingen, wenn ähnliche Funktionalitäten auch ohne angeboten werden können?
\newline\newline
Hingegen verlangt die Verwendung bestehender Web-Standards dem Entwickler die
Kenntnisse mehrerer Programmiersprachen und deren Verknüpfung ab, um annähernd
vergleichbare Ergebnisse zu erzielen. Dieser Umstand und die Abhängigkeit
von verschiedensten Browsern und -versionen macht es zu einer komplexen
Aufgabe ein Produkt zu entwickeln, dass in allen Browsern dieselbe
Funktionalität zur Verfügung stellt. HTML5 steht als Weiterentwicklung der
bestehenden HTML 4.01 Spezifikation in den Startlöchern um das Web
zukunftssicher zu machen. Es soll besagten Aufwand minimieren und gleichzeitig
Möglichkeiten bieten, die Flash seit geraumer Zeit gewährleistet.
\newline\newline
Deshalb gilt es in den Vergleichen herauszufinden, ob die kommenden
Webstandards HTML5 und CSS3, im speziellen das Canvas Element, in Zukunft das
proprietäre Webformat Flash samt ActionScript ablösen können. Mittels der
Evaluierung der Ergebnisse sollen Richtlinien erstellt werden, anhand derer
Web-Entwickler leicht ermitteln können, in welchen Fällen die eine Technologie
der anderen vorzuziehen ist und aus welchen Gründen.

% -Aufbau der Arbeit
\section{Aufbau der Arbeit}
Der Aufbau dieser Arbeit kann grob in vier Bereiche unterteilt werden.
Zu Beginn bietet der Theorieteil einen Überblick und eine Einführung in die
verwendeten Technologien, die für das Verständnis der Inhalte dieser Arbeit
notwendig sind. Anschließend folgt ein theoretischer Vergleich beider
Technologien anhand verschiedener Kriterien. Im dritten Teil folgt die
praktische Analyse der jeweiligen Technologien, die Beschreibungen der
umgesetzten Prototypen und deren Implementierung. Abschließend folgt eine
Diskussion und Zusammenfassung der erhaltenen Ergebnisse sowie ein Ausblick
auf die mögliche Zukunft des Webs und dessen Technologien.