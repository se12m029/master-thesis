% =Einleitung
\chapter{Einleitung}
% -Problemstellung
\section{Problemstellung}
Die Entwickler von Flash (früher Macromedia, jetzt Adobe) haben schon früh das
Potential von ihrem Flash Player erkannt, um plattformübergreifend Video- und
Audioinhalte im Web bereitzustellen. Ein großer Vorteil war, dass Inhalte für
Flash lediglich einmal kodiert werden mussten. Tausende von Seiten die sich für
Flash als Streamingplattform von Multimediainhalten entschieden haben,
bestätigten, bis heute, das Potential.
\newline\newline
Mit dem Erscheinen von Apples iPhone und iPod touch im Jahr 2007 und der
folgenden Entscheidung Flash auf diesen Geräten nicht zu unterstützen, mussten
Webseitenbetreiber darauf reagieren. Viele boten Video/Audio-Streams an, die
direkt im mobilen Safari Browser wiedergegeben werden konnten. Durch die
Verwendung der H.264 Kodierung konnten die Inhalte auch über den Flash-Player
(sofern vom verwendeten Gerät unterstützt) abgespielt werden. Dadurch mussten
Inhalte weiterhin nur einmal kodiert werden um mit möglichst vielen Plattformen
kompatibel zu sein.
\newline\newline
Die Entwickler der HTML5 Spezifikation (und u.a. Apple) sind der Meinung, dass
Browser Audio und Video nativ unterstützen sollten ohne dabei auf etwaige
Plugins zurückgreifen zu müssen. Dies hat den Vorteil, dass es keine
herstellerspezifischen Einschränkungen (wie es bei Flash der Fall ist) für den
Entwickler gibt, was er mit den Inhalten anstellt, nachdem sie in die Website
eingebettet wurden. Mittels CSS und JavaScript ist es jederzeit direkt möglich
dargestellte HTML Elemente zu manipulieren.
\newline\newline
Um ein Bild für eine Webanwendung zu erstellen wird üblicherweise eine
Grafiksoftware genutzt und anschließend in die Anwendung eingebettet. Für
Animationen wird vorwiegend Flash verwendet. Mit dem in HTML5 verfügbaren Canvas
Element können Entwickler Bilder und Animationen direkt im Browser mittels
JavaScript generieren. Mit Canvas ist es möglich einfache bis komplexe Formen,
Graphen und auch Diagramme zu erstellen, ohne auf diverse Bibliotheken, Flash
oder ein anderes Plugin zurückgreifen zu müssen.
\newline\newline
Mit der Veröffentlichung und der Implementierung von HTML5 Features in die
neuesten Brower eröffneten sich neue Möglichkeiten um Animationen, Spiele und
Entertainment-Applikationen für das Web zu erstellen. Aufgrund vieler
auftretender Fragen ist sich kein Entwickler wirklich klar ob HTML5 den
Platzhirschen Flash verdrängen kann:
\begin{itemize}
	\item Wo liegen die Vorteile und Nachteile beider Technologien?
	\item Ist HTML5 die Zukunft und löst Flash ab?
	\item Welche Technologie ist einfacher zu erlernen?
	\item Was kann HTML5 was Flash nicht kann (und visa-versa)?
	\item Lohnt es sich noch Flash zu erlernen?
	\item Gibt es Entwicklungsumgebungen die das Programmieren erleichtern?
\end{itemize}
% -Motivation
\section{Motivation}
--------MOTIVATION-------
% -Zielsetzung
\section{Zielsetzung}
Das Ziel dieser Arbeit ist es herauszufinden, ob die kommenden Webstandards
HTML5 und CSS3, im speziellen das Canvas Element, in Zukunft das proprietäre
Webformat Flash samt ActionScript ablösen können. Mittels der Ergebnisse sollen
Richtlinien erstellt werden, anhand derer Web-Entwickler leicht ermitteln
können, in welchen Fällen die eine Technologie der anderen vorzuziehen ist und
aus welchen Gründen.

% -Aufbau der Arbeit
\section{Aufbau der Arbeit}
Kapitel 2 bietet einen Überblick und eine Einführung in die verwendeten
Technologien, die für das Verständnis der Inhalte dieser Arbeit notwendig sind.
\newline\newline
Kapitel 3 
\newline\newline
Kapitel 4