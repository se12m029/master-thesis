\section*{Kurzfassung}\thispagestyle{empty}
Die Neuerungen von HTML5 bieten Entwicklern neue Möglichkeiten. Mit Hilfe des
Canvas-Elements und der Verwendung der Skriptsprache JavaScript kann der
Browser für Multimedia-Anwendungen und Spiele genutzt und damit die
weitere Internet-Entwicklung unabhängig von Adobe Flash gemacht werden.
In Zukunft könnte das bedeuten, dass Flash-Anwendungen obsolet werden.
\newline\newline
Diese Arbeit befasst sich mit der Gegenüberstellung und dem Vergleich der
beiden Kerntechnologien HTML5 und Adobe Flash. Dabei soll aufgezeigt werden,
wie sich der Einsatz von HTML5 und Adobe Flash aktuell gestaltet und welche
der beiden Technologien ein größeres Zukunftspotenzial besitzt. Anhand von
praktischen Umsetzungen sollen die Vorteile und Nachteile beider Konkurrenten
aufgezeigt und ein grober Leitfaden gegeben werden, für welche Projekte
welche Technologie bevorzugt eingesetzt werden kann.
\newline\newline
Nachdem zu Beginn die technischen Grundlagen vermittelt werden, folgt ein
Vergleich zwischen den beiden Technologien HTML5 und Adobe Flash.
Im Anschluss werden Analysen anhand von praktischen Umsetzungen im Bereich
Multimedia- und Spiele-Entwicklung durchgeführt. Abschließend wird das
Ergebnis diskutiert und ein Blick in die Zukunft geworfen.
\\ \vfill
% Bitte 3-5 deutsche Schlagwörter eingeben, die die Arbeit charakterisieren:
\paragraph*{Schlagwörter:} Web-Entwicklung, Web Standards, HTML5,
Adobe Flash, CSS, JavaScript, ActionScript