% =FLASH
\chapter{Adobe Flash}

\section{Was ist Flash?}
Während die Popularität von HTML5 immer weiter steigt, werden immer mehr
Vergleiche mit "`Flash"' durchgeführt. Hierbei handelt es sich um ein Produkt
der Firma Adobe, die sich auch unter anderem für Software wie Photoshop oder
Illustrator verantwortlich zeigen. Allerdings bezieht sich das Wort Flash in
den meisten Fällen nicht auf die gleichnamige Entwicklungsumgebung der Firma,
sondern auf den Flash Player, der das Abspielen von proprietären SWF-Dateien
im Browser ermöglicht. Anfänglich war Flash ein Programm um Illustrationen und
einfache Animationen zu erstellen und wurde 1997 von der Firma Macromedia,
welche im Jahr 2005 von Adobe aufgekauft wurde, entwickelt.

\section{Die Geschichte von Flash und ActionScript}
Seit der Veröffentlichung von Flash im Jahr 1996 wurde Flash und ActionScript
gemeinsam weiter entwickelt. Mittlerweile stellt die Kombination aus Design-
und Animations-Tools in Flash und den interaktiven Möglichkeiten von
ActionScript eine der mächtigsten, vielseitigsten und beliebtesten
Entwicklungsumgebung dar. Allerdings war ActionScript zu Beginn eine sehr
bescheidene Programmiersprache.
\newline\newline
Die ersten drei Versionen von Flash boten noch keine Programmiertools an und
interaktive Abläufe wurden durch Drag-and-Drop Optionen als Actions festgelegt.
Lediglich die Navigation durch die Zeitleiste und das Erstellen von Links waren
über diese Actions möglich.
\newline\newline
Flash 4 war die erste Version, die die Möglichkeit bot, Code mittels einer
einfachen Skriptsprache zu schreiben, welche inoffiziell bereits ActionScript
genannt wurde. Mit Flash 5 entwickelte sich ActionScript weiter und wurde zur
offiziellen Skriptsprache. Mit jeder folgenden Version von Flash wurden auch
die Möglichkeiten von ActionScript erweitert. ActionScript ermöglichte die
interaktive Kontrolle von Text, Animation, Sound, Video, Data und mehr. 2003
wurde ActionScript 2.0 vorgestellt. Die Möglichkeiten dieses Produktes waren
vergleichbar mit objekt-orientierten Programmiersprachen wie Java oder C\#.
\newline\newline
Professionelle Programmierer interessierten sich immer mehr für ActionScript
als Entwicklungstool, jedoch zeigte sich, dass trotz der vergleichbaren
Möglichkeiten ActionScript in der Performance noch nicht mit seiner Konkurrenz
mithalten konnte. Der Grund dafür war, dass jede Version von ActionScript auf
der vorherigen aufbaute. Der Flash Player war ursprünglich nicht für
aufwändige Anwendungen und komplexe Spiele konzipiert, dennoch begannen
Entwickler den Flash Player dafür zu nutzen. Damit wurde es klar, dass eine
neue Version von ActionScript von Beginn an neu entwickelt werden musste um
den Erwartungen gerecht zu werden.
\newline\newline
2006 wurde von Adobe ActionScript 3.0 vorgestellt, welches bedeutend mehr neue
Funktionalitäten und eine starke Performancesteigerung mit sich brachte. Flash
CS3 war die erste Version von Flash, die ActionScript 3.0 unterstützte. Flash
CS4 fügte neue Funktionen zu ActionScript 3.0 hinzu, unter anderem neue 3D
Möglichkeiten, Kontrollfunktionen für Animationen und ActionScript Klassen für
Adobe AIR. Flash CS5 und Flash CS6 erweiterten die Funktionalitäten unter
anderem mit der Unterstützung von Controller und anderen Geräten, inklusive
Multitouch-Funktionen und Touch-Screen Geräten.
\newline\newline
Die aktuelle Version des Adobe Flash Players (Version 11) wurde im
Oktober 2011 veröffentlicht.
% http://helpx.adobe.com/en/flash-player/release-note/fp_119_air_39_release_notes.html
% http://www.adobe.com/support/documentation/en/flashplayer/releasenotes.html

%alexander benz - begin

\section{Features}
Adobe Flash bietet für Entwickler eine Fülle an Funktionen und Features.
An dieser Stelle soll ein Überblick über die bedeutsamsten Funktionalitäten
gegeben werden.

\subsection{Grafiken und Animation}
\begin{itemize}
	\item{2D Zeichentools}
	\item{Animationstools}
	\item{3D Unterstützung}
	\item{3D Transformationen}
	\item{SVG (Scaleable Vector Graphics)}
	\item{Skalierbare Inhalte}
	\item{Text-Engine}
	\item{Filtereffekte (z.B. Weichzeichnen)}
	\item{Präsentationen}
\end{itemize}
Adobe Flash bietet Entwicklern eine große Auswahl an Werkzeugen zur
Erstellung von Grafiken und Animationen. Die umfangreiche
Entwicklungsumgebung ermöglicht mittels integrierter Tools wie den
Zeichentools, der Zeitleiste oder der inversen Kinematik die einfache
Erstellung von Grafiken und Animationen, ohne dabei ActionScript Code verwenden
zu müssen. Auch Texte lassen sich innerhalb der Entwicklungsumgebung
bearbeiten und modifizieren und mühelos in die fertige Anwendung integrieren.
Vordefinierte Effekte können auf eine Anwendung bequem per Knopfdruck
angewendet werden. Allerdings ist es auch möglich alle vordefinierten
Anwendungen mittels ActionScript anzusprechen, selbst zu programmieren oder
auch mit weiteren Funktionalitäten zu erweitern. Durch die Nutzung von
ActionScript wird auch die Programmierung von komplexeren Anwendungen
ermöglicht. Adobe legt großen Wert auf die bisherigen Möglichkeiten im 3D-
Bereich und möchte diese noch weiter ausbauen.

\subsection{Schnittstellen (APIs)}
\begin{itemize}
	\item{Zugriff auf das Filesystem}
	\item{Web Sockets}
	\item{Geolocation}
	\item{Datenaustausch}
\end{itemize}
Adobe Flash bietet eine Vielzahl an Schnittstellen an, mit denen Systeme und
Daten angesprochen werden können. Mit dem Zugriff auf das Filesystem des
Computers ist es möglich, Daten zu laden, zu bearbeiten und zu speichern. Der
Zugriff auf die Hardware des Computers wird ebenfalls durch Schnittstellen
ermöglicht. Dadurch kann zum Beispiel das Mikrofon, die Kamera oder die
Sensoren zur Standortbestimmung per Geolocation des Endgeräts genutzt werden.
Für den Austausch von Daten zwischen Anwendungen oder dem Computersystem bietet
Adobe Flash den Entwicklern weitere Schnittstellen für XML und andere
Dateiformate an.

\subsection{Multimedia}
\begin{itemize}
	\item{Videounterstützung}
	\item{Audiounterstützung}
	\item{Streaming}
\end{itemize}
Adobe Flash bot als eine der ersten Technologien die Möglichkeit, Video- und
Audio-Elemente in Anwendungen zu integrieren. Die Unterstützung von
verschiedenen Kodierungsverfahren und Optionen ermöglichen die einfache und
vollständige Kontrolle über die Qualität und Datenmenge. Als große Stärke
von Adobe Flash zeichnet sich die einfachen Realisierung von direkten
Live-Streams ab.

\subsection{Sonstiges}
\begin{itemize}
	\item{
	Kompatibilität und Unterstützung auf unterschiedlichen Plattformen und
	Endgeräten
	}
\end{itemize}
Adobe Flash unterstützt alle gängigen Desktop-Betriebssysteme wie Apples
MacOS, Microsoft Windows und Linux. Lediglich das mobile Betriebssystem
Android kann eine vereinfachte Version des Adobe Flash Players nutzen und
damit Inhalte vollständig wiedergeben. Für die nicht unterstützten mobilen
Betriebssysteme wie z.B. iOS können Flash Inhalte in HTML5 umgewandelt werden.
Allerdings ist das Ergebnis aufgrund der begrenzten technischen Möglichkeiten
nicht ident mit der originalen Flash Version.

\subsection{Flash Spezifisch}
\begin{itemize}
	\item{ActionScript}
	\item{Eigene Entwicklungsumgebungen (Adobe Flash, Flex)}
\end{itemize}
Die in Adobe Flash integrierte objektorientierte Programmiersprache
ActionScript ermöglicht die Realisierung umfangreicher und komplexer
Anwendungen. Adobe bietet mit Adobe Flash eine umfangreiche aber teure
Entwicklungsumgebung mit grafischer Oberfläche und integrierten Tools sowie
Syntaxhervorhebungen und Programmierhilfen für ActionScript Code.
Mit Flex gibt es auch eine kostenfreie Entwicklungsumgebung mit Tools und
Hilfen für die Entwicklung von Anwendungen mittels ActionScript, allerdings
ohne grafische Oberfläche und Werkzeuge zur Erstellung und Bearbeitung
von Grafiken und Animationen.

\section{Aktueller Einsatz}
% Adobe 2010 99% - noch vor dem Wandel zu HTML5
Trotz propritärer Technologien hat Adobe es geschafft, ihren Flash Player auf
99\% Prozent aller Systeme unterzubringen. Gründe dafür waren die Fähigkeiten
von Flash, mit denen Ziele erreicht werden konnten, die mit Hilfe von offenen
Standards nicht möglich waren. In den Anfangstagen von Flash wurde es vor
allem für Introanimationen, welche auf Startseiten eingesetzt wurden,
interaktive Navigationen, die auf die Aktionen des Users mit z.B. Animationen
reagierten, oder für die allseits bekannten Werbebanner genutzt. Die
Erstellung ähnlicher Funktionen mit standardkonformen Technologien wie
HTML 4.01, JavaScript oder CSS, war im Gegensatz zu Flash, nur schwer oder
überhaupt nicht realisierbar. Selbst heute werden komplette Webseiten, Spiele
und vor allem Werbebanner mittels Flash realisiert. Ein weiterer Vorteil von
Flash ist das einfache Erlernen der Programmiersprache und Bedienen der
Entwicklungsumgebung, wodurch auch Neueinsteiger sehr schnell zu ansehnlichen
Ergebnissen kommen.
\newline\newline
Für sich betrachtet vermitteln die Verbreitung des Adobe Flash Player und die
genannten Fakten den Eindruck, dass Flash die unangefochtene Nummer Eins unter
den Webtechnologien sein muss. Jedoch gilt es zu beachten, dass diese
Informationen aus einer Zeit stammen, in der der Wandel zu HTML5 gerade erst
begonnen hat, was zur Folge hat, dass aktuelle Zahlen anders aussehen würden.
Dennoch verdeutlichen die Verbreitung und die genannten Fakten was für eine
starke Plattform Adobe Flash bietet und in Zunkunft weiterhin bieten wird.
\newline\newline
HTML5 hat es schwer, Flash in sämtlichen Bereichen zu verdrängen. Einige
Anbieter, unter anderem YouTube, haben bereits mit der Umstellung ihrer
Inhalte begonnen. In Bereichen in denen die Features der HTML5 Spezifikation
bereits in einem fortgeschrittenen Entwicklungsstadium sind, können dafür
bereits genutzt werden. In anderen Bereichen wird Flash auch in Zukunft
unumgänglich bleiben, da es Features enthält, die mittels HTML5 nur
umständlich oder überhaupt nicht umsetzbar sind. Dies sind zum Beispiel die
Bereiche des gesicherten Online-Streamings oder die Erstellung von Rich
Internet Applications (RIA).
\newline\newline
Ein großer Vorteil von Flash stellt die hohe Verbreitung des Adobe Flash
Player dar. Wie es aus dem ersten Absatz des Kapitels ersichtlich ist,
kann dieser auf fast jedem Computer gefunden werden. Anders als HTML5 ist man
dabei nicht von den Browserherstellern abhängig, um neue Funktionen einzubauen
und zu unterstützen. Mit der Installation des Adobe Flash Player, sind
lediglich eigene Updates der Anwendungen für Neuerungen und Inhalte notwendig.
Bei Betriebssystemen ohne Unterstützungen des Flash Players sieht das wiederum
anders aus. Als Beispiel verzichtet Apple vollständig auf die Unterstützung
von Flash auf sämtlichen mobilen Endgeräten.
\newline\newline
Veröffentlicht als proprietäre Software, wurden seitdem große Teile von
ActionScript 3.0 durch Adobe selbst, Drittanbietern oder OpenSource Projekte
öffentlich zugänglich gemacht, wodurch es mittlerweile möglich ist,
unabhängig von Adobe Produkten Flash Inhalte zu erstellen. Mit
Entwicklungsumgebungen wie Flex und Add-Ons für Eclipse wird versucht, für
Entwickler immer bessere Tools zur Erstellung von Flash Anwendungen anzubieten.
\newline\newline
Trotz möglicher aufkommender Konkurrenten lässt sich,
aufgrund der hohen Nutzerzahl und der vielen Vorteile, mit Sicherheit sagen,
dass Flash weiterhin ein wichtiger Bestandteil des World Wide Webs bleiben
wird.

\section{ActionScript 3.0}
ActionScript ist die von Adobe für Flash entwickelte Programmiersprache.
Die aktuelle Version 3.0 ist eine vollwertige objektorientierte
Programmiersprache.
\newline\newline
Erstmals erschien die Sprache mit dem Flash Player 4 im Jahr 1999. Die
volle Unterstützung kam erst im Jahr 2000. Die erste Version beinhaltete
einfache Befehle, mit denen die Steuerung von Flash-Elementen möglich war.
Nutzer hatten damit erstmals die Möglichkeit frei über die Interaktionen
der Elemente zu entscheiden.
\newline\newline
Version 2.0 von ActionScript erschien im Jahr 2003 mit dem Flash Player 7
und erweiterte die Steuerungsmöglichkeiten von Flash Inhalten deutlich.
Auch die ersten Ansätze von objektorientierter Programmierung und Vererbung
von Klassen wurden implementiert. Im Vergleich zu anderen Skriptsprachen wie
JavaScript war der Funktionsmöglichkeiten von ActionScript zu diesem Zeitpunkt
bereits umfangreicher. So konnten Entwickler bereits komplexe multimediale
Inhalte erstellen, die vom Nutzer, dank hoher Interaktivität, welche mittels
ActionScript ermöglicht wurde, gesteuert werden konnten.
\newline\newline
2006 erschien die bis heute aktuelle Version von ActionScript: 3.0. Mit der
Version 3.0 wurden auch die meisten Neuerungen eingeführt. So unterstützt
ActionScript 3.0 die von anderen Programmiersprachen, wie z.B. Java oder C,
bekannten Paradigmen der klassenbasierten Objektorientierung und Typisierung
zur Laufzeit vollständig. Mittels einer virtuellen Maschine, in der
ActionScript Code nun ausgeführt wird, wird garantiert, dass der Code auf
jeder Plattform gleich ausgeführt wird. Im Hinblick auf mobile Plattformen
und deren starkem Wachstum ein zukunftssicherer Schritt.
\newline\newline
Entwicklern werden verschiedene Bibliotheken von Adobe mit den
Funktionalitäten von Flash angeboten, sodass auch außerhalb ohne der Nutzung
der offiziellen Entwicklungsumgebung Code generiert werden kann. So kann auch
mit wenig Mitteln komplexere Projekte auf der Basis von ActionScript
realisieren, die weit über den Einsatz im Flash Player als Browser Plug-In
hinaus gehen. So hat das Aufkommen von ActionScript 3.0 besonders in den
letzten Jahren dafür gesorgt, dass sich Flash im Markt der Rich Internet
Applications eine starke Position sichern konnte.

% alexander benz - end

%\section{Flash Verfügbarkeit auf mobilen Geräten}

%\section{Möglichkeiten}
%\subsection{Spiele}
%\subsection{Animationen}
%\subsection{Entertainment}

%\section{Die Einflüsse von Flash auf das Web}
%\subsection{Die Wurzeln und Entwicklung von Flash}
%\subsection{Stärken von Flash}
%\subsection{Wichtiger Beiträger für die Entwicklung des Webs}
\section{Apples Argumente gegen Flash}
Eine oft verwendete Aussage und immer wiederkehrendes Streitthema in vielen
Diskussionen ist sicherlich, ob HTML5 langfristig Flash als
Webentwicklungswerkzeug ablöst. Apple spielt hierbei eine große Rolle. Das
Unternehmen verzichtete seit der Einführung ihrer iDevices (iPhone, iPod
touch, iPad) auf die Nutzung von Flash. Vor allem die fehlende Möglichkeit,
Flash auf dem iPad zu verwenden, führte vermehrt zu Kritik. Der User sei
eingeschränkt und ohne Flash handelt es sich nicht um das "`echte Internet"',
welches Apple mit ihren Produkten anpreist. Der verstorbene Steve Jobs, CEO
von Apple, hat daraufhin einen offenen Brief verfasst, der den Nutzern
erklärt, warum die Firma auf Flash verzichtet hat und auch weiterhin darauf
verzichten wird. Steve Jobs Hauptargumente waren und gelten noch bis heute:
\begin{itemize}
	\item[1]{
		Flash ist ein 100 Prozent proprietäres System von Adobe, welches trotz der
		allgemeinen Verfügbarkeit des Flash Players die Zügel in der Hand hat.
		Adobe könnte ganz alleine über die Richtung entscheiden, in die Flash
		geht oder die Preisgestaltung ändern. Apple möchte das Internetstandards
		offen sein sollen, selbst wenn es auf die eigenen Geräte zutrifft.
	}
	\item[3]{
		Flashprodukte sind laut einer Analyse des Softwarehauses Symantec mit
		am anfälligsten für Sicherheitsprobleme. Sie seien einer der Hauptgründe
		für Abstürze auf dem Mac, auch die Performance auf Mobiltelefonen würde
		zu wünschen übrig lassen.
	}
	\item[4]{
		Flash beansprucht die Akkulaufzeit auf mobilen Geräten, da das Dekodieren
		von Videomaterial über die Software laufen muss. H.264 würde über die
		Hardware dekodiert werden.
	}
	\item[5]{
		Flash wurde nicht für Touchscreens geschaffen. Viele Webseiten müssten
		ihre Seite komplett neu konzipieren und entwickeln. Entwickler sollten
		aber dann doch auf modernere Technologien wie HTML5, CSS3 und
		Javascript zurückgreifen.
	}
\end{itemize}
Ein kommerzieller Hersteller wirft somit einem anderen kommerziellen Hersteller
vor, zu proprietär zu sein - das mag durchaus merkwürdig erscheinen. Vor allem
weil die Kunden von Apple auch im eigenen System eingezäunt und der Kontakt zu
anderen Systemen über Schnittstellen so gering wie möglich gehalten wird und
bei Adobe nur von der theoretischen, aber unwahrscheinlichen Möglichkeite
ausgegangen wird, dass diese die Vormachtstellung ihres Browserplugins
missbrauchen könnten.
\newline\newline
Auch die Performanceprobleme scheinen bei Googles Smartphone Betriebssystem
Android seit der Version 2.2 gelöst worden zu sein. Denn seitdem läuft
Flash stabil und schnell, ohne den Akku zu sehr zu beanspruchen.
\newline\newline
Die Behauptung das Flash nicht für Touchscreen ausgelegt worden ist, ist an
und für sich wahr, das gleiche gilt allerdings auch für HTML. Auch hier gibt
es eben so viele mausbasierende Aktionen, wie z.B. Rollover Animationen.