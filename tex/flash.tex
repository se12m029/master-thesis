% =FLASH
\chapter{Adobe Flash}

\section{Was ist Flash?}
Während die Popularität von HTML5 immer weiter steigt, werden immer mehr Vergleiche
mit "`Flash"' durchgeführt. Hierbei handelt es sich um ein Produkt der Firma Adobe, die 
unter anderem auch für Software wie Photoshop oder Illustrator verantwortlich ist. Allerdings
bezieht sich das Wort Flash in den meisten Fällen nicht auf die gleichnamige 
Entwicklungsumgebung der Firma, sondern der Flash Player, der das Abspielen von 
proprietären SWF-Dateien im Browser ermöglicht. Anfänglich war Flash ein
Programm um Illustrationen und einfache Animationen zu erstellen und wurde
1997 vonder Firma Macromedia, welche im Jahr 2005 von Adobe aufgekauft wurde,
entwickelt.

\section{Die Geschichte von Flash und ActionScript}
Seit der Veröffentlichung von Flash in 1996 wurde Flash und ActionScript gemeinsam
weiter entwickelt. Mittlerweile stellt die Kombination aus Design- und Animations-Tools 
in Flash und den interaktiven Möglichkeiten von ActionScript eine der mächtigsten, 
vielseitigsten und beliebtesten Entwicklungsumgebung dar. Allerdings war ActionScript 
zu beginn eine sehr bescheidene Programmiersprache.
\newline\newline
Die ersten drei Versionen von Flash boten noch keine Programmiertools an und 
interaktive Abläufe wurden durch Drag-and-Drop Optionen als Actions festgelegt. 
Lediglich die Navigation durch die Zeitleiste und das Erstellen von Links war 
über diese Actions möglich.
\newline\newline
Flash 4 war die erste Version, die die Möglichkeit bot Code mittels einer einfachen
Skriptsprache zu schreiben, welche inoffiziell bereits ActionScript genannt wurde.
Mit Flash 5 entwickelte sich ActionScript weiter und wurde zur offiziellen Skriptsprache.
Mit jeder folgenden Version von Flash wurden auch die Möglichkeiten von ActionScript erweitert.
ActionScript ermöglichte die interaktive Kontrolle von Text, Animation, Sound, Video, Data und
mehr. 2003 wurde ActionScript 2.0 vorgestellt, und dessen Möglichkeiten waren schon
vergleichbar mit objekt-orientierten Programmiersprachen wie Java oder C\#.
\newline\newline
Professionelle Programmierer interessierten sich immer mehr für ActionScript als 
Entwicklungstool, jedoch zeigte sich, dass trotz der vergleichbaren Möglichkeiten 
ActionScript in der Performance noch nicht mit seiner Konkurrenz mithalten konnte.
Der Grund dafür war, dass jede Version von ActionScript auf der vorherigen aufbaute.
Der Flash Player war ursprünglich nicht für aufwändige Anwendungen und komplexe Spiele
konzipiert, dennoch begannen Entwickler den Flash Player dafür zu nutzen.
Damit wurde es klar, dass eine neue Version von ActionScript von Beginn an neu entwickelt
werden musste um den Erwartungen gerecht zu werden.
\newline\newline
2006 wurde von Adobe ActionScript 3.0 vorgestellt, welches bedeutend mehr neue 
Funktionalitäten und eine starke Performancesteigerung mit sich brachte. Flash CS3 
war die erste Version von Flash die ActionScript 3.0 unterstützte. Flash CS4 
fügte neue Funktionen zu ActionScript 3.0 hinzu, unter anderem neue 3D Möglichkeiten,
Kontrollfunktionen für Animationen und ActionScript Klassen für Adobe AIR. Flash CS5 und
Flash CS6 erweiterten die Funktionalitäten unter anderem mit der Unterstützung von Controller 
und anderen Geräten, inklusive Multitouch-Funktionen und Touch-Screen Geräten.

\section{Flash Verfügbarkeit auf Computer}
Trotz propritärer Technologien hat Adobe es geschafft, ihren Flash Player auf 99\%
Prozent aller Systeme unterzubringen. Gründe dafür waren die Fähigkeiten von Flash
mit denen Ziele erreicht werden konnten, die mit Hilfe von offenen Standards nicht
möglich waren. In den Anfangstagen von Flash wurde es vor allem für Introanimationen,
welche auf Startseiten eingesetzt wurden, interaktive Navigationen, die auf die Aktionen
des Users mit z.B. Animationen reagierten, oder für die allseits bekannten Werbebanner.
Im Gegensatz zu Flash war die Erstellung ähnlicher Funktionen mit standardkonformen 
Technologien wie HTML 4.01, JavaScript oder CSS nur schwer oder überhaupt nicht 
realisierbar. Selbst heute werden für komplette Webseiten, Spiele und vor allem
Werbebanner mittels Flash realisiert. Ein weiterer Vorteil von Flash ist, dass es so einfach
zu erlernen und bedienen ist, das auch Neueinsteiger sehr schnell zu ansehnlichen 
Ergebnissen kommen.

\section{Flash Verfügbarkeit auf mobilen Geräten}

\section{Möglichkeiten}
\subsection{Spiele}
\subsection{Animationen}
\subsection{Entertainment}

\section{Die Einflüsse von Flash auf das Web}
\subsection{Die Wurzeln und Entwicklung von Flash}
\subsection{Stärken von Flash}
\subsection{Wichtiger Beiträger für die Entwicklung des Webs}
\subsection{Apples Argumente gegen Flash}
Ein oft verwendete Aussage und immer wiederkehrendes Streitthema vieler Diskussionen 
ist sicherlich, ob HTML5 langfristig Flash als Webentwicklungswerkzeug ablöst. Apple
spielt hierbei eine große Rolle. Das Unternehmen verzichtete seit der Einführung ihrer 
iDevices (iPhone, iPod touch, iPad) auf die Nutzung von Flash. Vor allem die fehlende Möglichkeit
Flash auf dem iPad zu verwenden führte vermehrt zu Kritik. Der User sei eingeschränkt und
ohne Flash handelt es sich nicht um das "`echte Internet"', welches Apple mit ihren Produkten
anpreist. Der verstorbene Steve Jobs, CEO von Apple, hat daraufhin einen offenen Brief 
verfasst, der den Nutzern erklärt, warum die Firma auf Flash verzichtet hat und auch
weiterhin darauf verzichten wird. Steve Jobs Hauptargumente waren und gelten noch bis
heute:
\begin{itemize}
	\item[1]{
		Flash ist ein 100 Prozent proprietäres System von Adobe, welches trotz der
		allgemeinen Verfügbarkeit des Flash Players die Zügel in der Hand hat.
		Adobe könnte ganz alleine über die Richtung entscheiden, in die Flash
		geht oder die Preisgestaltung ändern. Apple möchte das Internetstandards
		offen sein sollen, selbst wenn es auf die eigenen Geräte zutrifft.
	}
	\item[3]{
		Flashprodukte sind laut einer Analyse des Softwarehauses Symantec mit
		am anfälligsten für Sicherheitsprobleme. Sie seien einer der Hauptgründe
		für Abstürze auf dem Mac, auch die Performance auf Mobiltelefonen würde
		zu wünschen übrig lassen.
	}
	\item[4]{
		Flash beansprucht die Akkulaufzeit auf mobilen Geräten, da das Dekodieren
		von Videomaterial über die Software laufen muss. H.264 würde über die
		Hardware dekodiert werden.
	}
	\item[5]{
		Flash wurde nicht für Touchscreens geschaffen. Viele Webseiten müssten
		ihre Seite komplett neu konzipieren und entwickeln. Entwickler sollten
		aber dann doch auf modernere Technologien wie HTML5, CSS3 und
		Javascript zurückgreifen.
	}
\end{itemize}
Ein kommerzieller Hersteller wirf nun einem anderen kommerziellen Hersteller vor zu
proprietär zu sein - das mag durchaus merkwürdig erscheinen. Vor allem weil die 
Kunden von Apple auch im eigenen System eingezäunt und der Kontakt zu anderen 
Systemen über Schnittstellen so gering wie möglich gehalten wird un bei Adobe nur 
von der theoretischen, aber unwahrscheinlichen Möglichkeite ausgegangen wird, dass
diese die Vormachtstellung ihres Browserplugins missbrauchen könnten.
\newline\newline
Auch die Performanceprobleme scheinen bei Googles Smartphone Betriebssystem
Android seit der Version 2.2 gelöst worden zu sein. Denn seitdem läuft
Flash stabil und schnell, ohne den Akku zu sehr zu beanspruchen.
\newline\newline
Die Behauptung das Flash nicht für Touchscreen ausgelegt wirden ist, ist an und für
sich wahr, das gleiche gilt allerdings auch für HTML. Auch hier gibt es eben so viele
mausbasierende Aktionen, wie z.B. Rollover Animationen.