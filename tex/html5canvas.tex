%# HTML5 Canvas
\chapter{HTML5 Canvas}

% Überblick Canvas Element (Was ist es?)
% Wie funktioniert es (Vergleich Canvas/Flash)
% Überblick Canvas API (Was kann es?)
% Beispiele (Wie wird es gemacht?)
% Wo wird es schon verwendet?

\section{Einleitung}
Das Canvas Element bildet eines der größten Teile der HTML5 Spezifikation.
Es bietet eine API für das Zeichnen von 2D Objekten, wie Linien, Füllungen,
Bilder oder Text. Ein Vergleich mit MS Paint ist sehr gut um sich die
Möglichkeiten mit dem Canvas Element vorstellen zu können. Tatsächlich
gibt es bereits mehrere Implementationen, basierend auf dem Canvas Element,
die Applikationen wie MS Paint vollständig ersetzen können. Es geht bereits so
weit, dass laufend neue Zeichenapplikationen entstehen, die sich darum
bemühen, als vollwertige Vektor-Zeichenprogramm, ähnlich wie Adobe
Illustrator, angesehen zu werden. Das bemerkenswerte daran ist, dass es
auch einige Anwendungen gibt, die ebenfalls auf mobilen Endgeräten, wie
iPhones oder Android Smartphones, funktionstüchtig sind.

% Pics, Canvas MS Paint, Sketchpad, Harmony

\section{Geschichte}
Das Konzept des Canvas Elements wurde durch Apple eingeführt um in Mac OS X
Webkit Widgets für das Dashboard zu erstellen. Vor der Einführung des Canvas
Elements konnten Zeichen Schnittstellen im Browser nur dann genutzt werden,
wenn Plugins wie Flash von Adobe, Scaleable Vector Graphics (SVG), die
Vector Markup Language (VML), welche nur in Internet Explorer vorhanden war,
oder diverse JavaScript Bibliotheken eingesetzt wurden. % (Mary Nyamor S.25)
\newline
Das Canvas Element ist eine Umgebung für das Erstellen von dynamischen Bildern.
Das Element selbst ist dabei sehr einfach zu verwenden. Lediglich die
Dimensionen des Canvas und eine id, über die das Canvas anschließend über
JavaScript referenziert werden kann, müssen festgelegt werden.
%
\lstset{style=HTML5}
\begin{lstlisting}
  <canvas id="myCanvas" width="360" height="240">
  </canvas>
\end{lstlisting}
%
Alles was sich zwischen den sich öffnenden und schließenden <canvas> Tags
befindet, wird nur Browsern angezeigt, die das Canvas Element nicht
unterstützen.
%
\begin{lstlisting}
  <canvas id="myCanvas" width="360" height="240">
    <p>Canvas is not supported by your browser.</p>
  </canvas>
\end{lstlisting}
%
Um auf dem Canvas zeichnen zu können muss JavaScript verwendet werden. Dabei
muss zunächst das entsprechende Canvas über seine id ausgewählt und
anschließend der gewünschte Kontext definiert werden. Die Auswahl des Kontexts
beeinflusst die Funktionen der genutzen API:
%
\lstset{style=JavaScript}
\begin{lstlisting}
  var canvas = document.getElementById("myCanvas");
  var context = canvas.getContext("2d");
\end{lstlisting}
%
Beim Verfassen dieser Arbeit standen nur der 2D oder der WebGL (3D) Kontext zur
Verfügung. Mit der Auswahl des gewünschten Kontexts ist das Canvas Element
bereit um für das Zeichnen von Bilder genutzt zu werden. Die 2D API verfügt
über alle Funktionen die bei einem herkömmlichen, teils kostenpflichtigen,
Grafikprogramm wie zum Beispiel Adobe Illustrator erwartet werden: Es können
Linien, Konturen, gefüllte Flächen, Verläufe, Schatten, Formen und Bézier
Kurven gezeichnet werden. Der wesentliche Unterschied zwischen dem Canvas
Element und einem Grafikprogramm besteht allerdings darin, dass das Canvas
Element kein grafisches User Interface, zum Erstellen von Bildern, zur
Verfügung stellt, sondern jeder Pixel per JavaScript programmiert werden muss.

\subsection{Mit Code zeichnen}
Um die Farbe einer Kontur bzw. einer Linie zu definieren ist folgender Code
notwendig:
%
\begin{lstlisting}
  context.strokeStyle = "\#990000";
\end{lstlisting}
%
Alles was nun folgend auf dem Canvas gezeichnet wird, hätte eine rote Kontur.
Um ein Rechteck mit Kontur und ohne Füllung auf das Canvas zu Zeichnen gibt es
die strokeRect Methode und ist folgendermaßen definiert:
%
\begin{lstlisting}
  strokeRect(left, top, width, height);
\end{lstlisting}
%
Um nun ein rotes Rechteck zeichnen zu können, das 20 Pixel vom linken Rand und
30 Pixel vom oberen Rand des Canvas entfernt ist, 100 Pixel breit und 50 Pixel
hoch ist, ist folgender Code notwendig:
%
\begin{lstlisting}
  context.strokeRect(20, 30, 100, 50);
\end{lstlisting}
%
Dabei handelt es sich noch um ein sehr einfaches Beispiel um das Canvas zum
Zeichnen zu nutzen. Der 2D Kontext bietet eine sehr umfangreiche Auswahl an
vordefinierten Methoden wie fillStyle, zum Festlegen von Füllungen, fillRect,
um ein gefülltes Rechteck zu zeichnen, lineWidth, zum Festlegen der
Linienstärke, shadowColor, zum Festlegen der Schattenfarbe, und viele mehr.
Theoretisch betrachtet kann jedes Bild das mit einem Programm wie Adobe
Illustrator erstellt werden kann, ebenfalls mit dem Canvas Element realisiert
werden. Praktisch sieht das anders aus. Ein Versuch würde sich als
äußerst arbeitsaufwendig und in einer Unmenge an JavaScript äussern.
Abgesehen davon, wurde das Canvas Element nicht zur Reproduktion von
aufwendigen Pixel- oder Vektorgrafiken entwickelt.

\subsection{Canvas: Was bringts?}
Im vorherigen Abschnitt wurde ein Beispiel, die Reproduktion von Bildern,
genannt, wofür das Canvas Element nicht verwendet werden sollte. Die wirkliche
Macht des Canvas Elements liegt in der Möglichkeit, dass dessen Inhalt
jederzeit, aufgrund von zum Beipiel Nutzereingaben, verändert werden kann.
Gerade das Reagieren auf Nutzereingaben ermöglicht die Erstellung von
Anwendungen und Spiele die ursprünglich eine Technologie wie Adobe Flash
benötigten.
\newline
Eines der ersten Vorzeigebeispiele für die Möglichkeiten des Canvas Elements
wurde von Mozilla Labs entwickelt. Bespin: Ein Code Editor der innerhalb des
Browserfensters läuft. % (https://bespin.mozilla.com) + bild
Eine sehr eindrucksvolle und nützliche Applikation, allerdings von vielen
Entwicklern als perfektes Beispiel betrachtet, wofür das Canvas Element nicht
genutzt werden sollte.


