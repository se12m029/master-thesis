% =Überblick
\chapter{Überblick}
% -HTML5 Geschichte
\section{Die Geschichte von HTML5}
HTML ist die allgegenwärtige Auszeichnungssprache des World Wide Webs. Durch den geschickten Einsatz der paar Elemente (fortan: Tags) die die Sprache unterstützt, konnten erstaunlich viele unterschiedliche Netzwerke von verlinkten Dokumente erstellt werden. Von bekannten Seiten wie Amazon, Ebay und Wikipedia bis hin zu personalisierten Blogs und Webseiten die sich auf Katzenbilder spezialisiert haben.
HTML5 ist die aktuellste Version dieser Auszeichnungssprache. Obwohl diese
Version die bisher umfangreichsten Änderungen mit sich bringt, ist es nicht die
erste Aktualisierung von HTML.
Sir Tim Berners-Lee zeichnet sich für die Entwicklung von HTML und damit dem
Beginn des Internet verantwortlich. 1991 veröffentlichte er ein Dokument mit dem
Titel ``HTML Tags'', in dem er weniger als zwei Dutzend Elemente, die für das
Schreiben von Webseiten genutzt werden konnten, vorschlug.
Die Verwendung von Tags (Wörter umgeben von eckigen Klammern: z.B. <html>) war
nicht Berners-Lees eigene Errungenschaft. Tags wurden bereits in der SGML
(Standard Generalized Markup Language) verwendet.
Anstatt einen komplett neuen Standard zu erfinden, erkannte Berners-Lee die
Vorteile, bereits existierende Standards weiterzuentwickeln - ein Trend der auch
in der Entwicklung der neuen HTML5 Spezifikation erkennbar ist.

\subsection{Von der IETF zur W3C: Der Weg zu HTML4}
Die erste offizielle Spezifikation war HTML 2.0, veröffentlicht durch die IETF,
die ``Internet Engineering Task Force''. Viele der neuen Features dieser
Spezifikation basierten dabei auf bereits veröffentlichte Implementierungen. So
bot der im Jahr 1994 marktführende Web Browser Mosaic Autoren von Webseiten
bereits die Möglichkeit Bilder in Dokumente mittels eines <img>-Tags
einzubinden. Das <img>-Tag wurde in die HTML 2.0 Spezifikation inkludiert.
Die IETF wurde allmählich durch die W3C, dem World Wide Web Consortium, ersetzt.
Folgende Aktualisierungen des HTML Standards wurden auf \url{http://www.w3.org}
veröffentlicht. In der zweiten Hälfte der Neunziger wurde der Standard mehrmals
überarbeitet bis 1999 HTML 4.01 veröffentlicht wurde.

\subsection{XHTML 1: Die Vermischung von HTML und XML}
Das nächste Update von HTML 4.01 trug den Namen XHTML 1.0. Das X steht dabei für
``eXtensible''. Die Spezifikation von XHTML 1.0 war ident zu der von HTML 4.01,
somit wurden keine neuen Elemente und Attribute eingeführt. Die Spezifikationen
unterschieden sich nur durch die zu verwendende Syntax zum Schreiben von
Dokumenten. Während Autoren von HTML Dokumenten kaum Einschränkungen im
Schreibstil von Elementen und Attributen hatten, setzte XHTML es voraus, dass
die Regeln von XML, einer strikteren Auszeichnungssprache auf der viele
Technologien des W3C basierten, eingehalten werden.
Aufgrund der strikteren Regeln einigten sich Autoren auf einen gängigen
Schreibstil, der auch unter dem HTML 4.01 Standard Verwendung fand. Während Tags
früher in Kleinbuchstaben, Großbuchstaben oder einer Mischung aus beidem
geschrieben werden konnten, verlangte ein valides XHTML Dokument, dass alle
Elemente und Attribute klein geschrieben werden.
Während XHTML 1.0 noch auf HTML basierte und lediglich die strikteren Regeln von
XML verwendete, war die XHTML 1.1 Spezifikation reines XML. Dieser Umstand
führte zu schwerwiegenden Problemen. XHTML 1.1 Dokumente konnten nicht mehr
unter dem Mime-Type \textit{text/html} definiert werden. Der bis dato
bekannteste Webbrowser Internet Explorer konnte jedoch Dokumente die mit einem
XML Mime-Type publiziert wurden, nicht darstellen.

\subsection{XHTML 2}
Das W3C war mit der vierten Version von HTML der Meinung, dass der HTML basierte
Ansatz seinen Zenith erreicht hat und setzten für die zukünftigen Version
vollständig auf XML. Trotz der fast identen Namen von XHTML 1 und XHTML 2,
konnten die Unterschiede zwischen den beiden Spezifikationen nicht größer sein.
Anders als XHTML 1, war XHTML 2 nicht abwärtskompatibel mit bereits
existierenden Webinhalten oder vorangegangenen HTML Versionen. Es sollte ein
vollkommen neuer Standard werden und es zeigte sich, dass dieser Weg nicht mit
Erfolg gekrönt sein würde.

\subsection{Die Spaltung: WHATWG}
Innerhalb des W3C bildete sich eine Gruppierung die gegen die Ansätze
rebellierten. Namenhafte Vertreter von Opera, Apple und Mozilla waren mit der
eingeschlagenen Entwicklungrichtung des W3Cs nicht zufrieden. Ihnen war es
wichtiger mehr Aufmerksamkeit auf Formate zur Entwicklung von Webapplikationen
zu richten.
Bei einem Workshop in 2004 schlug Ian Hickson, der zu der Zeit bei Opera
arbeitete, eine Weiterentwicklung von HTML vor, mit der es möglich sein soll
Anwendungen zu entwickeln. Der Vorschlag wurde durch die W3C abgelehnt.
Unzufrieden mit der Entscheidung bildeten die Rebellen eine eigene Gruppe: Die
Web Hypertext Application Technology Working Group, oder kurz WHATWG.

\subsection{Der Weg von Web Apps 1.0 zu HTML5}
Von Beginn an operierte die WHATWG anders als das W3C. Während das W3C einen
kosensorientierten Ansatz anwendet, Themen werden vorgetragen, es wird
diskutiert und abgestimmt, wird auch bei der WHATWG diskutiert und abgestimmt,
allerdings liegt die finale Entscheidung, was in die Spezifikation komm und was
nicht, beim Editor. Der Editor ist Ian Hickson.
Der W3C Ansatz klingt demokratisch und fair, allerdings führt dieser auch dazu,
dass der Prozess stark verlangsamt wird. Bei der WHATWG hat jeder die
Möglichkeit mitzuwirken, da die letzte Entscheidung aber beim Editor liegt,
entwickelt sich alles schneller.
Schon zu Beginn der Gründung von WHATWG wurden die Projekte in zwei große
Spezifikationen aufgeteilt: Web Forms 2.0 und Web Apps 1.0. Beide
Spezifikationen sollen die bisherige HTML Spezifikation erweitern. Mit
fortschreitender Entwicklung wurden alle Spezifikationen in eine einzige
implementiert und in HTML5 umbenannt.

\subsection{Die Wiedervereinigung}
Während HTML5 von der WHATWG weiter entwickelt wurde, setzte das W3C die
Entwicklung der XHTML 2 Spezifikation fort. Erst im Oktober 2006, schrieb Sir
Tim Berners-Lee einen Blog-Post in dem er zugab, dass der Versuch, das Internet
von HTML auf XML zu übertragen, nicht funktionieren wird.
Nur wenige Monate später entschied sich die W3C eine neue HTML Working Group zu
bilden. Anstatt komplett von Anfang an zu beginnen, nutzen sie glücklicherweise
die bisherige Arbeit der WHATWG als Basis für zukünftige HTML Spezifikationen.
Dieses hin- und her führte zu einer schwer überschaubaren Situation. Das W3C
arbeitete parallel an den beiden unterschiedlichen, nicht kompatiblen
Spezifikationen: XHTML 2 und HTML 5 (mit Leerzeichen). Währenddessen arbeitete
die WHATWG, an der Spezifikation für HTML5 (ohne Leerzeichen) auf der die Arbeit
der W3C aufbaut.

\subsection{HTML5: 2012 und 2022}
Heute ist der aktuelle Stand der HTML5 Spezifikation nicht mehr so
undurchsichtig wie er früher war, allerdings gibt es noch immer offene Fragen.

Es gibt noch immer zwei Gruppen die an HTML5 arbeiten.
Die wohl wichtigste Frage für Webentwickler ist "`Wann können wir es nutzen?"'.
In einem Interview gab Ian Hickson an, dass HTML5 frühestens 2022 den ``proposed
recommendation'' Status erreichen wird. Klingt nach einer langen Wartezeit,
allerdings bedeutet ``proposed recommendation'', dass die HTML5 Spezifikation
zwei mal komplett implementiert werden muss. Als Vergleich: HTML 4 existiert nun
seit über einem Jahrzehnt und hat noch nicht die gesetzten Features erreicht.
Wenn man den Umfang der Spezifikation betrachtet, klingt dieses Datum hoch
gesteckt. Browserhersteller sind nach wie vor nicht dafür bekannt, dass
existierende Standards so schnell wie möglich implementiert werden. Der Internet
Explorer benötigte mehr als ein Jahrzehnt um das \textit{abbr}-Element richtig
darstellen zu können.

Für Webentwickler war das Jahr 2012 wesentlich wichtiger. 2012 erreichte die
HTML5 Spezifikation den ``candidate recommendation'' Status, der gleichbedeutend
ist mit "`fertig und abgeschlossen"'.
Allerdings reicht das alleine leider auch nicht aus. Wirklich entscheident ist
es wann Webbrowser den neuen Standard unterstützen. Schon die Veröffentlichung
des CSS 2.1 Standards zeigte, dass man nicht auf die Fertigstellung der
Spezifikationen warten sollte, sondern, wenn möglich, die Features nutzen sollte
sobald es möglich ist. Das selbe gilt auch für HTML5. Sobald Webbrowser
bestimmte Features der Spezifikation unterstützen können diese auch jederzeit
verwendet werden.
Man darf nicht vergessen, dass HTML5 keine komplett neu entwickelte Sprache ist.
Im Sinne der HTML Spezifikation ist es eher eine Evolution als eine Revolution.
Da HTML5 auf den früheren Versionen aufsetzt und irgendeine Version des HTML
Standards zur Erstellung von Webseiten genutzt wird, wird bereits HTML5 genutzt.


% \section{Geschichte}
% \subsection{Der Begründer des World Wide Web}
% Die Geschichte des Internets geht zurück bis in die frühen neuziger als Tim
%Berners-Lee das World Wide Web entwickelte mit der Hyper Text Markup Language
%(HTML) als Auszeichnungssprache. Während die Elemente (auch Tags genannt) zur
%Auszeichnung von Inhalten in strukturierte Teile (Paragraphen, Überschriften,
%Listen, etc.) dabei auf der Standard Generalized Markup Language (SGML), eine
%international anerkannte Methode um Texte zu strukturieren, basierte, war die
%Idee des Hypertext Links von Tim Berners-Lee.
%Ein Jahr später veröffentlichte Berners-Lee die erste Version seines Browsers.
%Während dieser Zeit waren die Möglichkeiten von HTML stark eingeschränkt und
%Entwickler konnten lediglich einfache Textinhalte im Web veröffentlichen. Mit
%dem Erscheinen von HTML+, welche von Dave Ragget von Hewlett-Packards Labs
%entwickelt wurde, gab es die Möglichkeit mittels IMG Element auch Bilder in
%Webseiten zu inkludieren

%Versuche um weitere Verbesserungen zur HTML Spezifikation hinzuzufügen, führte
%zur Veröffentlichung einer neuen Version, HTML 2.0. Sie basierte vollständig auf
%der alten Version wurde jedoch mit weiteren Funktionen und der Document Type
%Definition (DTD) erweitert. Die Funktionen von HTML 2.0 wurden zum Standard für
%alle Webbrowser. Der Wunsch nach Personalisierung, gestalterischer Freiheit und
%einem zeitgemäßen Look and Feel führte zu HTML 3.0 und der Einführung der
%Cascading Style Sheets.
%
%Probleme entstanden erst als Netscape sich entschied eigene Tags und Attribute
%in ihre Browser zu implementieren. Entwickler mussten Tags speziell für Netscape
%anpassen. Netscape sah jedoch ein, dass ihr Versuch nicht zielführend ist und
%gab ihe Erweiterungen auf. Mit dem steigenden Interesse am Internet
%veröffentlichte Microsoft die erste Version des Internet Explorers.
%
%Mit dem schnellen Fortschreiten der Entwicklung des HTML Standards wurde HTML
%4.0 veröffentlicht. Diese Version enthielt weitere Verbesserungen und
%Möglichkeiten in der Entwicklung von Webseiten. Es gab mehr Möglichkeiten um
%Multimedia in Webseiten einzusetzen, nutzen von Skiptsprachen, verbesserte
%Druckeigenschaften, Style Sheets und Optimierungen für Menschen mit
%Behinderungen.

%IMAGE Timeline of Web Technologies
% http://www.instantshift.com/2012/07/20/the-evolution-of-html5-infographic/
% publishing multimedia on the web.pdf

\section{Rich Media}
\subsection{Audio}
MP3 stellt das allgegenwärtige Format zur Kodierung von Audio-Dateien dar. Um sich diese
Audio-Dateien auch anhören zu können waren propritäre Technologien notwendig. Damit
wurde der Flash Player allgegenwärtig.
\newline\newline
Mit HTML5 bietet sich eine neue Technologie an die versucht den Platzhirschen Flash zu
verdrängen.
Das Einbinden einer Audio-Datei in ein HTML5-Dokument ist äußerst simpel:
\begin{verbatim}
<audio src="sample.mp3">
</audio>
\end{verbatim}
Das Audio Element bietet mehrer Attribute zum Steuern der Audioausgabe an.
Das Attribut ``autoplay'' sorgt dafür, dass die Audio-Datei nach vollständigem Laden
sofort gestartet wird. ``loop'' hingegen, kümmert sich darum, dass das die Audio-Datei
in einer Endlosschleife abgespielt wird.
\begin{verbatim}
<audio src="sample.mp3" autoplay loop>
</audio>
\end{verbatim}
Eine Besonderheit dieser Attribute ist, dass sie im Gegensatz zu den herkömmlichen HTML-Attributen,
keinen Wert zugewiesen bekommen. Das liegt daran, dass es sich bei diesen Attributen bereits um
``Boolean''-Attribute handelt. Das zuweisen eines Wertes, z.B. autoplay="no", wird an der Ausführung
des Attributes nichts ändern - entweder die Attribute werden angegeben und ausgeführt oder
eben nicht.

\subsubsection{Audio API}
Das Boolean-Attribut controls sorgt dafür, dass der Browser die nativen Kontrollmöglichkeiten
anzeigt und den Usern zu Verfügung stellt. Somit ist es sehr schnell möglich die Audiowiedergabe
zu Starten/Stoppen, die Position und die Lautstärke zu verändern.
%Image Audioplayer mit Controls
Über JavaScript ist es möglich die Audio API zuzugreifen und somit Methoden für Play, Pause und
Eigenschaften wie Lautstärke zu nutzen. Ein einfaches Beispiel mit Button Elementen und
Inline Event Handler:
\begin{verbatim}
<audio id="player" src="sample.mp3">
</audio>
<div>
<button onclick="document.getElementById('player').play()">
Play
</button>
<button onclick="document.getElementById('player').pause()">
Pause
</button>
<button onclick="document.getElementById('player').volume += 0.1">
Volume Up
</button>
<button onclick="document.getElementById('player').volume -= 0.1">
Volume Down
</button>
</div>
\end{verbatim}

\subsubsection{Audioformate}
Obwohl das Audio Element einen sehr guten Eindruck macht gibt es einen Wehrmutstropfen,
und dieser liegt nicht in der Spezifikation des Elements. Das Problem ist die breite Fragmentierung
von Audiocodecs. Leider gibt es bezüglich der zu verwendeten Codecs unterschiedliche Meinungen
unter den Browserherstellern. Während heutzutage das MP3 Format allgegenwärtig ist,
ist es nach wie vor kein offenes Format. Die Folge ist, dass MP3 Dateien von Applikationen nur
dann dekodiert werden können, wenn für die entsprechenden Patentrechte bezahlt wird.
Für Giganten wie Apple oder Adobe stellt das kein gröberes Problem dar, allerdings erschwert es die
Arbeit von kleineren Unternehmen und Open-Source Organisationen. Infolgedessen gibt
Apples Safari ohne Probleme MP3 Datei wieder während Mozillas Firefox daran scheitert.
\newline\newline
Natürlich existieren weitere Audioformate. Der Vobis Codec - auch bekannt unter der Datei-Endung
.ogg - ist zum Beispiel nicht patentiert. Firefox unterstützt den Codec aber Safari nicht.
\newline\newline
Glücklicherweise ist es nicht notwendig eine fixe Entscheidung bei der Auswahl des Codecs zu treffen.
Anstatt das src Attribut im sich öffnenden <audio> Tag zu nutzen, können mehrere Dateiformate
über source Elemente festgelegt werden:
\begin{verbatim}
<audio controls>
<source src="sample.ogg">
<source src="sample.mp3">
</audio>
\end{verbatim}
Ein Browser der Ogg Vorbis Datei wiedergeben kann wird sich für die weiteren source Elemente
nicht mehr interessieren. Ein Browser der MP3 wiedergeben kann aber Ogg Vorbis nicht wird
einfach das erste Source Element überspringen und die Datei im zweiten wiedergeben.
Zusätzliche Hilfe bei der Entscheidung bietet die Deklarierung des entsprechenden
Mime Types der Audio Datei:
\begin{verbatim}
<audio controls>
<source src="sample.ogg" type="audio/ogg">
<source src="sample.mp3" type="audio/mpeg">
</audio>
\end{verbatim}
Um die Möglichkeiten vom Audio Element vollständig ausnutzen zu können wird angeraten
MP3 und Ogg Vorbis als Codecs zu verwenden.

\subsubsection{Fallback Lösungen}
Auch wenn das festlegen von mehreren source Elementen sehr nützlich ist, muss
bedacht werden, dass Browser existieren die das Audio Element nicht unterstützen.
Internet Explorer und Konsorten müssen auf die altmodische Art auf ein Flashumsetzung
zurückgreifen. Glücklicherweise unterstützt das Audie Element die Nutzung von Flash.
Alles was zwischen den sich öffnenden und schließenden <audio> Tags befindet und
kein source Element ist wird nur dem Browser, der das audio Element nicht unterstützt,
angezeigt:
\begin{verbatim}
<audio controls>
<source src="sample.ogg" type="audio/ogg">
<source src="sample.mp3" type="audio/mpeg">
<object type="application/x-shockwave-flash"
<param name="movie" value="player.swf?soundFile=sample.mp3">
</object>
</audio>
\end{verbatim}
Das object Element bietet zusätzlich eine Möglichkeit an um Fallback Inhalte anzubieten.
So kann zum Beispiel im schlimmsten Fall ein gewöhnlicher Downloadlink angezeigt werden.
\begin{verbatim}
<audio controls>
<source src="sample.ogg" type="audio/ogg">
<source src="sample.mp3" type="audio/mpeg">
<object type="application/x-shockwave-flash"
<param name="movie" value="player.swf?soundFile=sample.mp3">
<a href="sample.mp3">Audio Datei herunterladen</a>
</object>
</audio>
\end{verbatim}
Mit diesem Code werden bereits vier Fallback Ebenen angeboten:
\begin{itemize}
\item{Der Browser unterstützt das Audio Element und den Ogg Vorbis Codec.}
\item{Der Browser unterstützt das Audio Element und den MP3 Codec.}
\item{Der Browser unterstützt das Audio Element nicht, hat aber das Flash Plug-in installiert.}
\item{Der Browser unterstützt das Audio Element nicht und hat kein Flash Plug-in installiert.}
\end{itemize}

\subsection{Video}
Mit dem anstieg der verfügbaren Bandbreite stieg auch das Interesse an Video Inhalten an.
Das Flash-Plugin ist derzeit noch die erste Wahl wenn Videos im Web angeboten werden sollen.
Mit HTML5 könnte sich das ändern.
\newline\newline
Das Video-Element funktioniert genauso wie das Audio-Element. Es unterstützt die
selben optionalen ``autoplay'', ``loop'' und ``preload'' Attribute. Der Speicherort des Videos
kann entweder mittels ``src'' Attribut im Video-Element oder mittels ``source'' Elementen, die
sich verschachtelt zwischen den sich öffnenden und schließenden <video> Tags befinden,
festgelegt werden. Zur Darstellung eines geeigneten User Interfaces kann entweder mittels
``controls'' Attribut dem der Browser die Darstellung übernehmen lassen oder es wird mit
entsprechenden HTML-Elementen, CSS Befehlen und JavaScript ein benutzerdefiniertes erstellt.
\newline\newline
Einer der wesentlichen Unterschiede zwischen dem Audio und Video Element ist, dass Videos
natürlicherweise einen fixen Bereich der Webseite einnehmen werden. Um diesen
Bereich festzulegen müssen im Video Element die entsprechenden Dimensionen definiert werden:
\begin{verbatim}
<video src="sample.mp4" controls width="360" height="240">
</video
\end{verbatim}
Um beim Laden des Videos kann mittels ``poster'' Attribute dem Browser mitgeteilt werden,
währenddessen ein representatives Bild anzuzeigen:
\begin{verbatim}
<video src="sample.mp4" controls width="360" height="240"
poster="sampleimage.jpg">
</video
\end{verbatim}
% Image Placeholder and Dimensions
Der Kampf der rivalisierenden Videoformate stellt sich noch extremer als bei den Audio Formaten dar.
Die am stärksten vertretensten Formate sind das patentierte MP4 und das freie Theora Video. Wie beim
Audio Element muss das Video in mehreren Kodierungen verfügbar und Notfalls eine Fallback Lösung
vorhanden sein.
\begin{verbatim}
<video controls width="360" height="240"
poster="sampleimage.jpg">
<source src="sample.ogv" type="video/ogg">
<source src="sample.mp4" type="video/mp4">
<object type="application/x-shockwave-flash" width="360" height="240"
data="player.swf?file=movie.mp4">
<a href="movie.mp4">Video herunterladen</a>
</object>
</video
\end{verbatim}
Die Autoren der HTML5 Spezifikation hofften auf eine Standardisierung des Videoformates. Allerdings
konnten sich die Browserhersteller, bis heute, nicht auf ein einziges Format einigen.

\subsubsection{Native Unterstützung von Videos}
Die Fähigkeit Videos nativ in Webseiten einzubinden stellt womöglich eine der aufregensten Erweiterung
von HTML dar, seit der Einführung des ``img'' Elements. Giganten wie Google zögern nicht lange und
zeigen bereits ihren Enthusiasmus mit einer auf HTML5 basierenden YouTube-Version:
\url{http://youtube.com/HTML5}
Eines der Probleme von Plug-ins zur Darstellung von Webinhalten ist, dass der Inhalt des Plug-ins
von dem restlichen Inhalt der Webseite geschützt ist (``sandboxed''). Nativen Rich Media Elementen
in HTML haben zur Folge, dass sie ohne Probleme mit den anderen Web-Technologien,
CSS und JavaScript, zusammen arbeitet.
\newline\newline
Das Video Element ist somit nicht nur programmierbar sondern auch stylebar.
% Image Skin & Style
Ein Plug-in bietet derartige Möglichkeiten nicht an.

\subsection{Canvas}

Das Canvas Element ist eine Umgebung zur Erstellung von dynamischen Bildern.
Das Element ist genauso einfach wie das Audio oder Video Element zu verwenden.
Als Attribute werden lediglich Breiten- und Höhenangaben des Canvas angeboten:
\begin{verbatim}
<canvas id="canvas" width="360" height="240">
</canvas>
\end{verbatim}
Alles was sich zwischen den sich öffnenden und schließenden <canvas> Tags
befindet, wird nur Browsern angezeigt, die das canvas Element nicht
unterstützen.
\begin{lstlisting}[language=html]
<canvas id="canvas" width="360" height="240">
<p>Canvas Element wird von ihrem Browser nicht unterstuetzt.</p>
</canvas>
\end{lstlisting}

JavaScript wird verwendet um das Canvas produktiv nutzen zu können.
Um mit dem Canvas arbeite zu können muss immer das entsprechende
Element über ihre ID ausgewählt und der Kontext festgelegt werden.
Kontext bedeutet in diesem Fall welche API genutzt werden soll:
\begin{verbatim}
var canvas = document.getElementById('canvas');
var context = canvas.getContext('2d');
\end{verbatim}
Aktuell stehen nur der 2D und WebGL Kontexts zur Verfügung.
Mit der Auswahl des Kontextes ist das Canvas Element bereit
um für das Zeichnen von Bildern genutzt zu werden.
Die 2D API verfügt über so ziemlich alle Tools die man auch bei einem
Grafik Programm wie Illustrator erwartet: Es können Linien, Konturen, gefüllte
Flächen, Verläufe, Schatten, Formen und Bézier Kurven gezeichnet werden.
Der wesentliche Unterschied besteht allerdings darin, dass kein
grafischen User Interface genutzt wird, sondern alles mit JavaScript
definiert werden muss.

\subsubsection{Mit Code zeichnen}
Um die Farbe einer Kontur bzw. einer Linie zu definieren ist folgender
Code notwendig:
\begin{verbatim}
context.strokeStyle = "\#990000";
\end{verbatim}
Alles was nun auf dem Canvas gezeichnet wird, hat eine rote Kontur.

Die Syntax für die strokeRect Methode sieht dabei folgendermaßen aus:
\begin{verbatim}
strokeRect(left, top, width, height);
\end{verbatim}
Um nun ein rotes Rechteck zeichnen, das 20 Pixel vom linken Rand und
30 Pixel vom oberen Rand des Canvas entfernt ist und 100 Pixel breit und
50 Pixel hoch ist, ist folgender Code notwendig:
\begin{lstlisting}
context.strokeRect(20, 30, 100, 50);
\end{lstlisting}
% Image Rectangle Canvas
Dabei handelt es sich noch um ein sehr einfaches Beispiel. Die 2D API
bietet eine sehr umfangreiche Auswahl an Methoden wie
fillStyle, fillRect, lineWidth, shadowColor und viele mehr.
\newline\newline


\section{Verfügbarkeit}
Seit der offiziellen Präsentation der ersten Entwürfe der erneuerten
Webtechnologie, ist die Begeisterung in der IT Branche groß. Diese Begeisterung
ist besonders an dem Enthusiasmus der Web-Giganten Apple, Google und Mozilla und
der stetigen Implementierung von HTML5 und CSS3 in ihre Browser erkennbar.

\subsection{HTML5, CSS3 und JavaScript: Das Web von morgen}
Graph

\subsection{Verfügbarkeit auf Computer}
Ob und wie weit die Implementierung der neuen Webfeatures in Browsern
fortgeschritten ist, hängt vollständig von den jeweiligen Browserherstellern ab.
Um die Verfügbarkeit auf Computern festzustellen, muss der Marktanteil der
unterschiedlichen Browser mit einbezogen werden.

IMAGE Browser Jahr

Es ist notwendig die kommenden Trends zu bedenken, da die weitere
Implementierung von HTML5 stark von der Popularität des jeweiligen Web-Browsers
abhängt. Unbekanntere Browserhersteßller werden naturgemäß mehr Zeit für die zur
Verfügungstellung von HTML5-Features benötigen als bekannte und beliebte Größen
wie Google oder Mozilla.

IMAGE Browser Statistik Monat zu Monat

Die angeführten Statistiken zeigen auf, dass nicht nur der gegenwärtige
Prozentsatz an Nutzern sondern auch der Trendverlauf darauf schließen lässt,
dass der Internet Explorer tatsächlich nicht mehr zu den marktführenden Browsern
zählt, wodurch auch dessen Bedeutung in der Web-Entwicklung schwindet.
Aufgrund der sich reduzierenden Nutzeranzahl und dem Umstand, dass der Internet
Explorer der Browser mit der am geringsten fortgeschrittensten Implementierung
des HTML5 Spezifikation am Markt ist, ermöglicht vielen Web-Entwicklern, schon
vor der Fertigstellung der Spezifikation, die Nutzung einiger neuen Features.

Microsofts Marktstrategie ihr eigenes Betriebssystem (Windows) ausschließlich
mit dem eigenen Browser auszuliefern ändert kaum etwas an der Statistik. Viele
Benutzer installieren daher zum Beispiel Mozilla Firefox oder Google Chrome
selbstständig.

\subsection{Verfügbarkeit auf mobilen Geräten}

\section{Möglichkeiten}

Folglich werden einige neue Elemente der HTML5 Spezifikation genauer
vorgestellt. Diese wurden auch in der Umsetzung der praktischen Ausarbeitungen
verwendet.


\subsection{Spiele}
\subsection{Animationen}
Mit HTML5 Canvas und den CSS3 Neuerungen gibt es für Entwickler neue APIs um
Grafiken im Web zu zeichnen und diese auch zu animieren. Die APIs bieten
einfache anzuwendente Funktionen an um vordefinierte Formen zu zeichnen, Bilder
zu importieren, die Darstellung von bereits gezeichneten Bildern zu verändern
oder diese auch zu animieren. Im Gegensatz dazu bietet Flash eine komplette IDE
um schnell komplexe Formen zu erstellen, diese anzuzeigen, deren Verhalten zu
steuern und zu animieren.

Ein Charakter lässt sich mit einer Zeichenapplikation um ein vielfaches
einfacher gestalten als mit mehreren Zeilen Code. Flash bietet eine
vollständiges Tool zur Erstellung und Animierung komplexer Grafiken und ist
damit wesentlich effizienter zu nutzen, solange es kein equivalentes Tool für
HTML5 und CSS3 gibt.

Dieses Kapitel soll aufzeigen, ob HTML5 und CSS3 effizient für die Erstellung
von Animationen verwendet werden kann. Anhand bestehender Animationen ...
Anschließend werden die Fragen behandelt wie eine Animation mittels Code
erstellt wird und ob sich der Zeitaufwand für Entwickler auszahlt. Abschließend
wird analysiert ob Webbrowser die Animationen flüssig wiedergeben können.

\subsubsection{CSS3 Spiderman}
Ein kurzer Animationsfilm der mittels HTML5, CSS3 und jQuery erstellt wurde. Die
Animationen sind komplex. So bewegt sich der Hintergrund, es gibt mehrere
Szenen, unterschiedliche Betrachtungswinkel, Bewegungen und Gesichtsausdrücke.
Der Film beinhaltet auch Musik mittels dem HTML5 Audio Element.

Bisher gibt es viele Experimente die mittels CSS3 umgesetzt wurden. Im Vergleich
gibt es nur wenige die auf das Canvas Element und JavaScript setzen.

CSS3 bietet einfache Transformationen (Rotation, Translation, Skalierung) und
reichen für simple Animationen. Flash bietet zusätzlich Formtransformationen,
als Beispiel die allmähliche Veränderung eines Kreises in ein Quadrat. Bisher
ist das mit CSS3 nicht möglich.

CSS3 zeigt im Vergleich zu HTML5 ein größeres Potential um Animationen zu
erstellen. Allerdings ist es schwierig mittels Code komplexere Animationen zu
erstellen. Solange es kein Tool für HTML5 und CSS3 Animationen existiert wird
Flash weiterhin die führende Software in diesem Bereich bleiben.

Performancetechnisch wurde diese Animation ohne jegliche Probleme flüssig in
jedem modernen Browser angezeigt.

\subsection{Entertainment}

\subsubsection{Audio \& Video}
Flash ist bis heut noch das bekannteste propritäte Plugin um Audio- und
Videoinhalte im Internet zur Verfügung zu stellen. Jedoch bietet nun HTML5
eigene Features die diesem Bereich gewidmet und sind ernst zu nehmende
Alternativen zu Flash. Es werden die Audio und Video Spezifikation von HTML5
genauer betrachtet.

\subsubsection{Audio}
Das Audio Element steht bisher in allen aktuellen Webbrowsern zur Verfügung und
ist ähnlich einfach zu nutzen wie das Image Element. Folgender Code zeigt wie
ein Audiofile in eine Webseite eingebunden werden kann:
\begin{verbatim}
<audio src="audio.ogg" controls>
<p>Your browser does not support the audio element.</p>
</audio>
\end{verbatim}
Leider gibt es bezüglich der zu verwendeten Codecs unterschiedliche Meinungen
unter den Browserherstellern. Um die Möglichkeiten vom Audio Element vollständig
ausnutzen zu können wird angeraten MP3 und Ogg Vorbis als Codecs zu verwenden.

\subsubsection{Video}
Ein Video kann mittels folgendem Code auf einer Webseite angezeigt werden:
\begin{verbatim}
<video src="movie.ogg" width="640" height="360" controls>
<p>Your browser does not support the video element.</p>
</video>
\end{verbatim}
Auch beim Video Element ist die Frage nach dem Codec nicht einfach zu
beantworten. Das Video Element kann mit allen modernen Webbrowsern genutzt
werden, allerdings gibt es andere Probleme

\begin{itemize}
\item Apple Geräte können den Ogg Theora Codec aufgrund von Hardware Problemen
nicht wiedergeben
\item Opera und Firefox unterstützen den H.264 Codec aufgrund von Lizenz
Problemen nicht.
\end{itemize}

Erst vor kurzem brachte Google eine mögliche Lösung ist Spiel: Die Nutzung von
WebM. Dabei handelt es sich um einen lizenzfreie Videocodec. Firefox, Opera,
Chrome und auch IE haben bereits bestätigt, dass dieser Codec in Zukunft
unterstützt wird.
Einige der bekanntesten Video Streaming Anbieter sind bereits dabei HTML5 für
Videoinhalte zu nutzen, darunter unter anderem YouTube und Vimeo.

In Kombination mit einem Canvas Element können die Inhalte des Video Elements
auf verschiedenste Art manipuliert werden. Unter anderem könnte das Bild in
mehere Teile zerschnitten, Explosionen eingefügt und Filter angewendet werden.
Allerdings, benötigen komplexe Effekte und Manipulationen eine
Hardwarebeschleunigung. Flash nutzt zur effizienten Darstellung von Animationen
und Videos die Grafikkarte des Anwenders. Die Möglichkeiten von HTML5 sind noch
nicht soweit entwickelt und bietet keine native Möglichkeit um die Hardware des
Anwenders zu nutzen.

Um auch die Video Element effizient nutzen zu können, sollten die Quelldateien
mit mindestens zwei verschiedene Codecs zur Verfügung gestellt werden.

\section{Vor- und Nachteile}
Jeffrey Zeldman war schon im Jahr 2010 der Meinung, dass HTML, CSS und
JavaScript in Zukunft die Entwicklung von Rich Media Applikationen vorantreiben
wird. Einige der Vorteile werden hier aufgezählt.

\subsection{Plattformübergreifende Web-Technologien}
Web-Entwickler mussten vor der Umsetzung einer Webanwendung oder Webseite
entscheiden für welche Webbrowser entwickelt werden soll. Dieser Umstand
entstand aufgrund der ungeeigneten Konzeption früherer HTML und CSS Versionen.
Mit den neuen Webstandards (HTML5, CSS3 und JavaScript) ist es nun möglich
plattformübergreifende Anwendungen zu erstellen, die nicht nur auf allen
Desktopbrowsern sonder auch auf mobilen Browsern genutzt werden können.

\subsection{Optimiertes Mobile-Web}
Bei der Entwicklung von Anwendungen für mobile Geräte spielt die begrenzte
Bandbreite des Internets eine entscheidente Rolle. Mittels CSS3 können viele
Bilder durch Gradienten, Schatten und andere Effekte ersetzt werden.
Überdimensionierte Hintergrund- und Füllbilder werden somit überflüssig und
führen zu einem geringerem Verbrauch der zur Verfügung stehenden Bandbreite.
Einige HTML5 Features wie Local oder Session Storage erlauben es Daten
clientseitig zu speichern und Applikationen auch ohne Internet verfügbar zu
machen. Zusätzlich müssen Daten die bereits runtergeladen wurden nicht noch
einmal gealden werden.

\subsection{Rich Media ohne Plugins}
Die Audio, Video und Canvas APIs von HTML5 erlauben es Rich Media ohne
Verwendung von Plugins im Internet zu nutzen. Propritäte Technologien die in
diesem Bereich hauptsächlich verwendet werden haben mit HTML5 einen ernst zu
nehmenden Konkurrenten bekommen.

\subsection{Rich Media Applikationen}
Der Großteil der neuen HTML5 Features sind speziell für die Erstellung von Rich
Media Applikationen vorgesehen.

% =HTML5 Syntax
\section{Die Syntax von HTML5}

% =HTML5 Rich Media
\section{Rich Media}

% =HTML5 Web Forms 2.0
\section{Web Forms 2.0}

% =HTML5 Semantik
\section{Semantik}

% =HTML5 Schon Heute
\section{HTML5 schon heute}