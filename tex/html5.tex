%# HTML5
\chapter{HTML5}

\section{Was ist HTML5?}

%## HTML5 Geschichte
\section{Die Geschichte von HTML}
HTML, ``Hypertext Markup Language'', ist die allgegenwärtige
Auszeichnungssprache des World Wide Webs. Durch den geschickten Einsatz der
paar Elemente (fortan: Tags, Wörter umgeben von eckigen Klammern z.B. <html>)
die die Sprache unterstützt, konnten erstaunlich viele unterschiedliche
Netzwerke von verlinkten Dokumente erstellt werden. Von bekannten Seiten wie
Amazon, Ebay und Wikipedia bis hin zu personalisierten Blogs oder Webseiten,
die sich zum Beispiel auf Katzenbilder spezialisiert haben. HTML5 ist die
aktuellste Version dieser Auszeichnungssprache. Obwohl diese Version die
bisher umfangreichsten Änderungen mit sich bringt, ist es nicht die erste
Aktualisierung von HTML.
\newline\newline
Sir Tim Berners-Lee zeichnet sich für die Entwicklung von HTML und damit dem
Beginn des Internets verantwortlich. 1991 veröffentlichte er ein Dokument mit
dem Titel ``HTML Tags'', in dem er weniger als zwei Dutzend Elemente, die für
das Schreiben von Webseiten genutzt werden konnten, vorschlug.
Darunter auch einige Tags, die bis dato verwendet werden, wie zum Beispiel:
%
\begin{itemize}
  \item{Listen (<ul>, <ol>, <li>)}
  \item{Überschriften (<h1>, <h2>, ...)}
  \item{Paragraphen (<p>)}
  \item{Titel (<title>)}
  \item{Links (<a>)}
\end{itemize}
%
Die Verwendung von Tags war nicht Berners-Lees eigene Errungenschaft. Tags
wurden bereits in der SGML (Standard Generalized Markup Language) verwendet.
Anstatt einen komplett neuen Standard zu erfinden, erkannte Berners-Lee die
Vorteile, bereits existierende Standards weiterzuentwickeln - ein Trend der
auch in der Entwicklung der neuen HTML5 Spezifikation erkennbar ist.

\subsection{Von der IETF zur W3C: Der Weg zu HTML4}
Die erste offizielle Spezifikation war HTML 2.0, veröffentlicht durch die IETF,
die ``Internet Engineering Task Force''. Viele der neuen Features dieser
Spezifikation basierten dabei auf bereits veröffentlichte Implementierungen.
So bot der im Jahr 1994 marktführende Web Browser Mosaic Autoren von
Webseiten bereits die Möglichkeit, Bilder in Dokumente mittels eines <img>-Tags
einzubinden. Das <img>-Tag wurde in die HTML 2.0 Spezifikation inkludiert.
Die IETF wurde allmählich durch die W3C, dem ``World Wide Web Consortium'',
ersetzt. Folgende Aktualisierungen des HTML Standards wurden auf \url{
http://www.w3.org} veröffentlicht. In der zweiten Hälfte der Neunziger wurde
der Standard mehrmals überarbeitet bis 1999 HTML 4.01 veröffentlicht wurde.

\subsection{XHTML 1: Die Vermischung von HTML und XML}
Das nächste Update von HTML 4.01 trug den Namen XHTML 1.0. Das X steht dabei
für ``eXtensible''. Die Spezifikation von XHTML 1.0 war ident zu der von
HTML 4.01, somit wurden keine neuen Elemente und Attribute eingeführt. Die
Spezifikationen unterschieden sich nur durch die zu verwendende Syntax zum
Schreiben von Dokumenten. Während Autoren von HTML Dokumenten kaum
Einschränkungen im Schreibstil von Elementen und Attributen hatten, setzte
XHTML es voraus, dass die Regeln von XML, ``Extensible Markup Language'',
einer strikteren Auszeichnungssprache auf der viele Technologien des W3C
basierten, eingehalten werden. Aufgrund der strikteren Regeln einigten sich
Autoren auf einen gängigen Schreibstil, der auch unter dem HTML 4.01 Standard
Verwendung fand. Während Tags früher in Kleinbuchstaben, Großbuchstaben oder
einer Mischung aus beidem geschrieben werden konnten, verlangte ein valides
XHTML Dokument, dass alle Elemente und Attribute klein geschrieben werden.
Während XHTML 1.0 noch auf HTML basierte und lediglich die strikteren Regeln
von XML verwendete, war die XHTML 1.1 Spezifikation reines XML. Dieser Umstand
führte zu schwerwiegenden Problemen. XHTML 1.1 Dokumente konnten nicht mehr
unter dem Mime-Type \textit{text/html} definiert werden. Der bis dato
bekannteste Webbrowser Internet Explorer konnte jedoch Dokumente die mit einem
XML Mime-Type publiziert wurden, nicht darstellen.

\subsection{XHTML 2}
Das W3C war mit der vierten Version von HTML der Meinung, dass der HTML
basierte Ansatz seinen Zenith erreicht hat und setzte für die zukünftige
Version vollständig auf XML. Trotz der fast identen Namen von XHTML 1 und
XHTML 2, konnten die Unterschiede zwischen den beiden Spezifikationen nicht
größer sein. Anders als XHTML 1, war XHTML 2 nicht abwärtskompatibel mit
bereits existierenden Webinhalten oder vorangegangenen HTML Versionen. Es
sollte ein vollkommen neuer Standard werden und es zeigte sich, dass dieser
Weg nicht mit Erfolg gekrönt sein würde.

\subsection{Die Spaltung: WHATWG}
Innerhalb des W3C bildete sich eine Gruppierung die gegen die Ansätze
rebellierten. Namenhafte Vertreter von Opera, Apple und Mozilla waren mit der
eingeschlagenen Entwicklungrichtung des W3Cs nicht zufrieden. Ihnen war es
wichtiger, mehr Aufmerksamkeit auf Formate zur Entwicklung von Webapplikationen
zu richten. Bei einem Workshop in 2004 schlug Ian Hickson, der zu der Zeit bei
Opera arbeitete, eine Weiterentwicklung von HTML vor, mit der es möglich sein
soll, Anwendungen zu entwickeln. Der Vorschlag wurde durch die W3C abgelehnt.
Unzufrieden mit der Entscheidung bildeten die Rebellen eine eigene Gruppe:
Die ``Web Hypertext Application Technology Working Group'', oder kurz WHATWG.

\subsection{Der Weg von Web Apps 1.0 zu HTML5}
Von Beginn an operierte die WHATWG anders als das W3C. Während das W3C einen
kosensorientierten Ansatz anwendet, Themen werden vorgetragen, es wird
diskutiert und abgestimmt, wird auch bei der WHATWG diskutiert und abgestimmt,
allerdings liegt die finale Entscheidung, was in die Spezifikation komm und
was nicht, beim Editor. Der Editor ist Ian Hickson. Der W3C Ansatz klingt
demokratisch und fair, allerdings führt dieser auch dazu, dass der Prozess
stark verlangsamt wird. Bei der WHATWG hat jeder die Möglichkeit mitzuwirken,
da die letzte Entscheidung aber beim Editor liegt, entwickelt sich alles
schneller. Schon zu Beginn der Gründung von WHATWG wurden die Projekte in zwei
große Spezifikationen aufgeteilt: Web Forms 2.0 und Web Apps 1.0. Beide
Spezifikationen sollen die bisherige HTML Spezifikation erweitern. Mit
fortschreitender Entwicklung wurden alle Spezifikationen in eine einzige
implementiert und in HTML5 umbenannt.

\subsection{Die Wiedervereinigung}
Während HTML5 von der WHATWG weiter entwickelt wurde, setzte das W3C die
Entwicklung der XHTML 2 Spezifikation fort. Erst im Oktober 2006, schrieb Sir
Tim Berners-Lee einen Blog-Post, in dem er zugab, dass der Versuch, das
Internet von HTML auf XML zu übertragen, nicht funktionieren wird.
Nur wenige Monate später entschied sich das W3C eine neue HTML Working Group
zu bilden. Anstatt komplett von Anfang an zu beginnen, nutzen sie
glücklicherweise die bisherige Arbeit der WHATWG als Basis für zukünftige HTML
Spezifikationen. Dieser Umstand führte zu einer schwer überschaubaren
Situation. Das W3C arbeitete parallel an den beiden unterschiedlichen, nicht
kompatiblen Spezifikationen: XHTML 2 und HTML 5 (mit Leerzeichen).
Währenddessen arbeitete die WHATWG, an der Spezifikation für HTML5 (ohne
Leerzeichen) auf der die Arbeit der W3C aufbaut.

\subsection{HTML5: 2012 und 2022}
Heute ist der aktuelle Stand der HTML5 Spezifikation nicht mehr so
undurchsichtig wie er früher war, allerdings gibt es noch immer offene Fragen.
\newline\newline
Es gibt noch immer zwei Gruppen die an HTML5 arbeiten. Die wohl wichtigste
Frage für Webentwickler ist: Wann können wir es nutzen? In einem Interview gab
Ian Hickson an, dass die HTML5 Spezifikation frühestens 2022 den ``proposed
recommendation'' Status erreichen wird. Klingt nach einer langen Wartezeit,
allerdings bedeutet ``proposed recommendation'', dass die HTML5 Spezifikation
zwei mal komplett implementiert werden muss. Als Vergleich: HTML 4 existiert
nun seit über einem Jahrzehnt und hat noch nicht die gesetzten Features
erreicht. Wenn man nun den Umfang der Spezifikation betrachtet, klingt dieses
Datum hoch gesteckt. Browserhersteller sind nach wie vor nicht dafür bekannt,
dass existierende Standards so schnell wie möglich implementiert werden. Der
Internet Explorer benötigte mehr als ein Jahrzehnt um das \textit{abbr}-
Element richtig darstellen zu können.
\newline\newline
Für Webentwickler war das Jahr 2012 wesentlich wichtiger. 2012 erreichte die
HTML5 Spezifikation den ``candidate recommendation'' Status, der
gleichbedeutend ist mit "`fertig und abgeschlossen"'. Allerdings reicht das
alleine leider auch nicht aus. Wirklich entscheidend ist die Frage: Ab wann
unterstützen Webbrowser den neuen Standard? Schon die Veröffentlichung des
CSS 2.1 Standards zeigte, dass man nicht auf die Fertigstellung der
Spezifikationen warten sollte, sondern, wenn möglich, die Features nutzen
sollte sobald es möglich ist. Das selbe gilt auch für HTML5. Sobald Webbrowser
bestimmte Features der Spezifikation unterstützen, können diese auch jederzeit
verwendet werden. Man darf nicht vergessen, dass HTML5 keine komplett neu
entwickelte Sprache ist. Im Sinne der HTML Spezifikation ist es eher eine
Evolution als eine Revolution. Da HTML5 auf den früheren Versionen aufsetzt
und immer irgendeine Version des HTML Standards zur Erstellung von Webseiten
genutzt wird, wird bereits HTML5 genutzt.

\section{Features}
Mit HTML5 wird Entwicklern eine Fülle an neuen Funktionen und Features
angeboten. An dieser Stelle soll ein Überblick über die Funktionalitäten
gegeben werden, die diese Technologie auszeichnet.

\subsection{Grafiken und Animation}
\begin{itemize}
  \item{2D Zeichentools}
  \item{Animationstools}
  \item{3D Unterstützung}
  \item{3D Transformationen}
  \item{SVG (Scaleable Vector Graphics)}
  \item{Skalierbare Inhalte}
  \item{Web-Fonts}
  \item{Filtereffekte (z.B. Weichzeichnen)}
  \item{Präsentationen}
\end{itemize}
Die Spezifikation von HTML5 enthält bereits umfangreiche Ansätze zum Erstellen
von Grafiken und Animationen. So können mit CSS3 und dem neuen Canvas-Element
einfache und auch komplexe Grafiken erstellt und, mit dem Einsatz von der von
dem Canvas-Element zur Verfügung gestellten Schnittstellen, über JavaScript
animiert werden. Dank CSS3 können auch erstmals ohne Abstriche Schriftarten
integriert und verwendet werden. Ein Feature das bis zu diesem Zeitpunkt nur
von Flash angeboten wurde. Mit der Integration des SVG-Formats bietet auch
HTML5 eine Möglichkeit um skalierbare Inhalte in Webseiten einzubinden.
Einfache Filtereffekte können nun unkompliziert mit CSS3-Regeln erstellt
werden. Für komplexe Effekte kann das Canvas-Element in Kombination mit
JavaScript verwendet werden.

\subsection{Schnittstellen (APIs)}
\begin{itemize}
  \item{Zugriff auf das Filesystem}
  \item{Sockets}
  \item{Geolocation}
  \item{Datenaustausch}
\end{itemize}
HTML5 bietet wie Adobe Flash eine Vielzahl an Schnittstellen, um mit anderen
Systemen zu interagieren. Beispielsweise kann auf das Filesystem des
Computers zugeriffen werden, um Daten zu speichern oder zu laden.
Mittlerweile gibt es eine Unmenge an JavaScript Frameworks aus denen ein
Entwickler zur Umsetzung von Projekten zurückgreifen kann. Das wohl
bekannteste Framework ist jQuery. Frameworks helfen Entwicklern sonst schwer
zu realisierende Features einfacher umzusetzen, so bietet jQuery mehrere
Schnittstellen mit der ein Zugriff auf verschiedene Hardwarekomponenten, wie
zum Beispiel die zur Standordbestimmung, möglich ist. Schon frühere Versionen
von HTML unterstützten den Austausch von Daten über verschiedene Formate
wie XML, JSON oder andere.

\subsection{Multimedia}
\begin{itemize}
  \item{Videounterstützung}
  \item{Audiounterstützung}
  \item{Spracheingaben}
\end{itemize}
Eine der größten Neuerungen und Grund für den Schlagabtausch zwischen HTML5
und Adobe Flash bildet die Integration von nativen Audio- und Videoelementen
und deren Unterstützung. Beide Elemente können dabei über seperate
Schnittstellen angesprochen und gesteuert werden. Damit bietet sich eine
Möglichkeit an Multimedia-Inhalte in Webseiten einzubinden, ohne auf den Adobe
Flash Player zurückgreifen zu müssen. Einziger Wehrmutstropfen ist die noch
zum Teil unvollständige Unterstützung durch bestehende Browser, jedoch sollten
zukünftige Aktualisierungen dieses Problem beseitigen. In Kombination mit
weiteren Neuerungen von HTML5 und JavaScript lassen sich interessante und
komplexe multimediale Anwendungen erstellen. So kann Aufgrund der neuen
Spracheingabeschnittstelle auf das Mikrofon des Computers zugegriffen und
Spracheingaben verarbeitet werden.

\subsection{Sonstiges}
\begin{itemize}
  \item{
    Kompatibilität und Unterstützung auf unterschiedlichen Plattformen und
    Endgeräten
  }
  \item{Ansätze einer Programmiersprache}
\end{itemize}
HTML5 Inhalte lassen sich auf jeder und für jede Plattform erstellen.
Die Kompatibilität herzustellen liegt dabei nicht in der Hand des Entwicklers,
sondern in denen der unterschiedlichen Browserhersteller. Durch die steigende
Anzahl an Schnittstellen und den vielen Neuerungen nimmt JavaScript immer mehr
die Form einer komplexen Programmiersprache an, mit deren Hilfe komplexe
Anwendungen realisiert werden können.

\subsection{HTML5 Spezifisch}
\begin{itemize}
  \item{Semantisches Markup}
  \item{Suchmaschinenoptimierung}
\end{itemize}
Semantisches Markup ist wohl das Feature das HTML im allgemeinen auszeichnet.
Dank der angewandten Semantik ist es möglich, Inhalte mit, für den Computer
verständliche, Bedeutung zu versehen. Ein Feature das sich bereits im
Bereich der Suchmaschinenoptimierung als essenziel erwiesen hat und mit der
steigenden Bedeutung von Suchmaschinen nicht zu unterschätzen ist. Einige
interessante Neuerungen von HTML5 umfassen die Erweiterungen der bestehenden
Tags mit noch aussagekräftigeren Elementen.

\section{Aktueller Einsatz}
Keine andere Technologie konnte bis jetzt für einen ähnlich großen Hype sorgen
wie HTML5. Von vielen wird die neue Spezifikation als die einzige Technologie
angesehen, die notwendig sein wird, um in Zunkunft professionelle,
anspruchsvolle und interaktive Webinhalte zu erstellen. Gegenwärtig werden
diese Erwartungen allerdings noch nicht vollständig erfüllt. Die aktuelle
Version, die veröffentlicht wurde, ist lediglich ein Entwurf für den finalen
Webstandard. Bis dieser Standard verabschiedet wird, dauert es noch bis ins
Jahr 2022. Das hat zur Folge, dass Entwickler mit Änderungen im
Funktionsumfang und der bestehenden Features rechnen können.
\newline\newline
Nichts desto trotz kann und soll der neue Standard für die Entwicklung
von Webseiten und -applikationen genutzt werden. Mit HTML5 wurden viele
Altlasten älterer HTML Versionen beseitigt und einige neue und zukunftsichere
Funktionen hinzugefügt. In einigen Bereichen wie Multimedia, Interaktivität
und Schnittstellen steht HTML5 dem bisherigen Platzhirsch Adobe Flash in
den erwähnten Bereichen kaum noch nach. Im direkten Vergleich zu Adobe Flash
löst HTML5 einige Aufgaben deutlich besser und verbraucht dabei auch noch
weniger Ressourcen und Bandbreite.
\newline\newline
Mit der kürzlichen Umstellung großer Videoportalen (beispielsweise Youtube)
von Adobe Flash auf HTML5 lässt sich eine eindeutige Bewegung feststellen.
Allerdings wird bis jetzt die HTML5 Variante lediglich als zusätzliche
Möglichkeit neben Flash angeboten. In Zukunft soll jedoch komplett auf die
Verwendung von Flash verzichtet werden. Was die Umstellung bei Webseiten
betrifft, geht diese wesentlich gemächlicher voran.
\newline\newline
Im Gegensatz zu Adobe Flash ist für die Verbreitung von HTML5 nicht nur
der Einsatz des Standards sondern auch die Unterstützung der Features durch
die verschiedenen Browser notwendig. Leider zeigt sich hierbei eine der
Schwächen von HTML5: Längst nicht alle Features werden von allen Browsern
unterstützt. Ganz im Gegenteil, interessante neue Funktionen werden
von vielen Browsern nur teilweise oder garnicht unterstützt.
Das führt unweigerlich zu sehr unterschiedlichen Darstellungen einer
Webseite auf mehreren Browsern und damit zu einem Mehraufwand und höheren
Kosten in deren Entwicklung und Testung.

\section{Technologieverbund}
Wenn von HTML5 gesprochen wird, ist in den häufigsten Fällen der Verbund
aus der Markupsprache HTML in der Version 5, der deklarativen Sprache
CSS in der Version 3 und der Skriptsprache JavaScript (auch bekannt
als ECMAScript) gemeint. Alle drei Technologien greifen ineinander und
bieten in ihrer Gesamtheit die neuen Features von HTML5.
% Bild mit Triangle HTML Struktur/Inhalt, CSS Präsentation/Darstellung, JS Logik

\subsection{HTML5}
HTML ermöglicht es, Inhalte im Web zu strukturieren und darzustellen.
Zusätzlich bietet HTML Funktionen um Texte, Bilder, Hyperlinks,
Videos, Audio und andere multimediale Inhalte semantisch auszuzeichnen
und damit für andere Systeme, zum Bespiel dem Computer, verständlich
zu machen. Jegliche Inhalte, die im Web dargestellt werden sollen, müssen
in die HTML-Struktur eingebunden werden, selbst solche, die zur Darstellung
Plug-Ins von anderen Anbietern wie beispielsweise Adobe Flash benötigen.
\newline\newline
Die gewünschten Inhalte werden dabei in verschiedene HTML-Elemente
eingebettet. Jedes dieser HTML-Elemente erfüllt einen bestimmten Zweck und
dieser reicht von Überschriften und Paragraphen bis hin zu Hyperlinks und
Bildern. Zusätzlich können diese Elemente durch verschiedene Attribute noch
genauer ausgezeichnet werden. Generische Elemente wie das div-Element,
ergänzen den begrenzten Elemente-Umfang, um nicht eindeutig zuordenbare
Inhalte auf einer Webseite zu platzieren. Über diesen Weg kann jede
Form von Inhalt erstellt werden.
\newline\newline
Für HTML5 wurden verschiedene Ziele festgelegt die es zu erreichen galt.
Diese Ziele dienten zusätzlich dem Zweck, den neuen Standard konkurrenzfähig
und zukunftssicher zu machen. Die Ziele waren:
%http://www.w3.org/TR/html-design-principles/
\begin{description}
  \item[Kompatibilität:]
  Bereits bestehende Inhalte müssen weiterhin verwendet werden können. Neue
  Funktionen des Standards dürfen die bestehenden Inhalte nicht verändern oder
  beeinflussen.
  \item[Verwendbarkeit:]
  Neue Funktionen sollen bestehende Probleme lösen und die Entwicklung
  einfacher machen.
  \item[Sicherheit:]
  Ein wichtiger Aspekt, der im Web immer mehr an Bedeutung gewinnt.
  \item[Konsistenz:]
  Sowohl in HTML als auch in XHTML sollen XML-Elemente genutzt werden können.
  Die Sprachen greifen dabei auf eine gemeinsame Dokumentation zurück.
  \item[Vereinfachung:]
  HTML soll deutlich vereinfacht und durch konkrete Fehlermeldungen schneller
  und einfacher korrigierbar werden.
  \item[Universalität:]
  Inhalte sollen in jeder Sprache und auf allen Endgeräten sowie Systemen
  darstellbar sein.
  \item[Zugänglichkeit:]
  Inhalte und Funktionen sollen jedem zugänglich sein. Dies entspricht dem
  OpenSource-Gedanken.
\end{description}
%
Bis dato wurden bereits einige Ziele erreicht. Bis zur offiziellen
Verabschiedung des Standards im Jahr 2022 werden mit Sicherheit alle Ziele
erreicht werden.

\subsection{CSS3}
Mittels ``Cascading Stylesheets'', oder kurz CSS, wird die Darstellung von
Elementen in einem HTML Dokument kontrolliert. So ist es möglich die
Positionierung und Attribute wie Farbe, Schattierung oder ähnliches, von
Elementen exakt zu definieren. CSS ist damit das Werkzeug der Wahl um
Designvorgaben möglichst genau umzusetzen.
\newline\newline
Die Trennung von HTML und CSS symbolisiert die Trennung von Inhalt und
Design. Dies ermöglicht eine unterschiedliche Darstellung der selben Inhalte
durch unterschiedliche Stylesheets für verschiedene Endgeräte oder den Druck,
ohne dabei den Inhalt selbst anpassen zu müssen.
\newline\newline
Der Name Cascading Stylesheets ist abgeleitet von den zur Verfügung stehenden
Vererbungsregeln, die das Format zulässt. Durch die Vererbungsregeln ist es
möglich, ein große Auswahl an Elementen gleich aussehen zu lassen und
dennoch jedes beliebige Element individuell anzupassen. Die von
HTML gelieferten Attribute ID und Class dienen zur Unterscheidung von
einmalig auftretenden Elementen (id) und ganzen Gruppierungen (class).
\newline\newline
Mit dem Level 3 des Cascading Stylesheet Standards kommen viele neue
Eigenschaften hinzu, die die Gestaltung von Elementen noch schneller und
einfacher machen. Mit CSS2 konnten vorwiegend einfache Darstellungsparameter
wie Farbe, Rahmen, Innenabstand und Aussenabstand sowie die genau
Positionierung im Dokumentenfluss definiert werden. CSS3 ermöglicht nun
etliche grafische Effekte wie Verläufe, Schatten oder abgerundete Ecken.
Zusätzlich zählen zu den wichtigsten neuen Funktionen von CSS3 die Erstellung
von einfachen Animationen, das Einbinden und Modifizieren von Schriftarten
sowie Transformationen und Skalierungen von Inhalten.
\newline\newline
Seit dem Jahr 2000 ist der CSS3 Standard in Entwicklung und die komplette
Unterstützung durch Browserhersteller schreitet immer weiter voran. Vor
allem durch das Aufkommen von HTML5 werden die Eigenschaften immer stärker in
den Mittelpunkt gerückt. Aus diesem Grund werden die Funktionen von CSS3 oft
dem HTML5 Standard zugeordnet.

\subsection{JavaScript}
Die Skriptsprache JavaScript übernimmt in dem Technologieverbund die logische
Ebene. Die Verwendung von HTML als eigenständige Technologie bietet nur die
statische Erstellung und Darstellung von Inhalten. Durch JavaScript lassen
sich HTML-Elemente allerdings dynamisch verändern. Ein Trend der sich bei
der Entwicklung von modernen Webseiten durchgesetzt hat und mittlerweile
nicht mehr weg zu denken ist. JavaScript wurde 1995 offiziell von Brendan Eich
eingeführt und ist aktuell in der Version 1.8.5 verfügbar. Allerdings
unterstützen ein Großteil lediglich die Version 1.5 vollständig.
\newline\newline
Die Skriptsprache ermöglicht die Manipulation von HTML-Elementen auf der
im Browser aufgerufenen Seite, das dynamische Erstellen neuer Elemente oder
das dynamische Nachladen von Inhalten mit Hilfe von XML und AJAX. Besonders
für einfache Effekte wie ein Dropdown-Menü oder das Einblenden von Tooltips
bietet sich der Einsatz von JavaScript an.
\newline\newline
Als dritte Säule der Webseitenerstellung kümmert sich JavaScript daher vor
allem um die Dynamik und Interaktivität von modernen Webseiten. Diese
Funktionalitäten werden heutzutage von Nutzern erwartet. Die steigenden
Erwartungen und die immer komplexer werdenden Webseiten fordern auch, dass
JavaScript diesen gewachsen ist. Diese Anforderungen und die fehlende
Unterstützung der aktuellsten Skript-Versionen machen die Entwicklung
unverhältnismäßig schwer, saubere Lösungen für an sich einfache Probleme zu
entwickeln. Für diesen Fall kommen die sogenannten JavaScript Frameworks ins
Spiel. Frameworks bieten Anhand von kompatibler JavaScript Versionen neue
Funktionalitäten. Besonders um einfach, sicher und effizient Animationen oder
Effekte zu realisieren oder Elemente zu manipulieren bietet sich die Nutzung
von Frameworks an. Einige der bekanntesten Frameworks sind dabei Prototype,
MooTools und das mit Abstand am häufigsten verwendete Framework jQuery.
\newline\newline
Nur durch das Zusammenspiel des Grundgerüsts HTML, des Designs CSS3 und der
Interaktivität durch JavaScript wird HTML5 zu dem Gesamtprodukt, welches in
Zukunft die Entwicklung des Webs prägen wird und zu einem ernst zu nehmenden
Konkurrenten für bereits etablierte Softwarelösungen wie Adobe Flash werden
lässt.

\section{Notwendigkeit von HTML5}
Die letzte Version von HTML wurde gut 10 Jahre lang eingesetzt. Mit dem
unglaublichen schnellen Wachstum im Bereich der mobilen Endgeräte, wie
Smartphones und Tablets, und den steigenden Erwartungen der Nutzer im Bezug
auf Inhalte und Funktionalitäten konnte die bestehende Version den gegebenen
Ansprüchen nicht mehr gerecht werden. Um sowohl die geforderten Anforderungen
zu erfüllen und zukunftssicher zu sein ist eine neue Version der Sprache
notwendig. HTML5 kann diese Version werden.
\newline\newline
Die bisher vergleichweise rasante Entwicklung von HTML5 wird sich daher auch
in Zukunft kaum verlangsamen. Anders als bei einer proprietären Software
beteiligen sich viele der führenden Unternehmen im Bereich Internet an der
Entwicklung des neuen Standards. Damit kann man von einer umfassenden
Unterstützung durch verschiedene Browser, Endgeräte und Schnittstellen
ausgehen.
\newline\newline
Um technische Neuerungen und moderne Webseiten zu gestalten mussten bisher
immer die Funktionen von Adobe Flash oder ähnlicher Software genutzt werden.
Die neuen Features von HTML5 sollen hingegen eine Möglichkeit bieten
komplett auf Software-Lösungen von Drittanbieteren zu verzichten und dabei
einen einheitlichen Standard für alle Geräte und Plattformen zu schaffen.
\newline\newline
Vor allem im Bereich der Multimediainhalte wie Videos, Audio, Animationen, 3D
und Interaktion musste Adobe Flash genutzt werden. Zu beachten ist dennoch,
dass Software von Drittanbietern neben technischen Vorteilen auch einige
Nachteile mit sich bringen. Einer der größten Nachteile stellt die kaum
oder nur mit sehr viel Aufwand verbundenen Suchmaschinen-Optimierung von
Flash-Inhalten dar. Ein Problem, dass in einer Zeit in der Online-Marketing
immer mehr an Bedeutung gewinnt, nicht zu unterschätzen ist. Weiters
werden laufend neue Sicherheitsproblemem bei den angebotenen Plug-Ins, die
notwendig sind um entsprechende Inhalte darzustellen, bekannt und nur langsam
oder auch unzureichend behoben. Aufgrund der vielen Nachteile wurde der Ruf
der Kritiker, von Adobe Flash und proprietärer Software, nach einer
Alternative immer lauter.
\newline\newline
Mit der Entscheidung von Apple Adoble Flash auf ihren mobilen Endgeräten
nicht mehr zu unterstützen, führte außerdem dazu, dass schnell Alternativen
entwickelt werden mussten. HTML5 stellt dabei eine optimale Lösung dar.