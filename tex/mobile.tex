% =MOBILE
\chapter{Mobile Geräte}

\section{Die Entwicklung des mobilen Internets}
Die Optimierung von Applikationen und Webseiten für mobile Endgeräte gewinnt
immer mehr an Wichtigkeit. Wollte ein Nutzer vor der aktuellen Smartphonegeneration 
(iPhone und Android) unkompliziert mobil im Internet surfen oder E-Mails schreiben 
wollte, kam er nicht um ein Blackberry von RIM herum. RIM legte bei ihren Telefon
den Hauptfokus auf die E-Mail Funktion. Das Surfen auf einem mobilen Telefon
war meist durch zu kleine Bildschirme, langsame Verbindungsgeschwindigkeiten,
umständlicher Bedienung, zu hohen Kosten und Einschränkungen in der Darstellung
und Nutzung von Webinhalten stark eingeschränkt. Die Einführung der 
GSM-Erweiterungen GPRS und EDGE sowie die noch schnelleren, auf dem 
Mobilfunkstandard basierenden UMTS-Datenübertragungsverfahren HSDPA und
HSUPA sorgte, insbesondere mit dem Erscheinen des iPhone von Apple, für einen
langsamen Umschwung in der mobilen Internetnutzung. Das Bankhaus Morgan
Stanley hat im April 2010 eine 87-seitige Präsentation zum Thema "`Internet Trends"'
veröffentlicht, welche zum einen dem mobilen Internet ein schnelleres Wachstum 
gegenüber dem desktop-basierten Internet verspricht, und zum anderen, dass im
Jahr 2014 dieses in der Nutzung überholen wird.

\section{iOS von Apple}
\section{Android von Google}