%# Implementierungen
\chapter{Implementierungen}

Um die in dieser Arbeit vorgestellten Technologien im praktischen Einsatz zu
veranschaulichen beschreibt dieses Kapitel die Implementierungen von drei
unterschiedlichen Applikationen. Diese Applikationen verwenden jeweils
verschiedene Features der HTML5 Spezifikation, wobei die Aufmerksamkeit
hauptsächlich an der Nutzung des Canvas Elements liegt, und Adobe Flash mit
ActionScript 3.0.

\section{Animation}

\subsection{Konzept}
\subsection{Umsetzung HTML5}
\subsection{Umsetzung Adobe Flash}

\section{Spiel}

\subsection{Konzept}
\subsection{Umsetzung HTML5}

  \subsubsection{Dokumente und Objekte}
  Bei der Umsetzung dieses Spiels werden HTML, CSS und JavaScript-Dateien
  benötigt. Zusätzlich werden Grafiken benötigt, die die Juwele im Spiel
  repräsentieren soll.

  1. Beginn
  Index.html - Benötigte CSS+JS Files festlegen.
  settings festlegen
  yepnope preload/loader

\subsection{Umsetzung Adobe Flash}

\section{Multimedia-Anwendung}

\subsection{Konzept}
\subsection{Umsetzung HTML5}
\subsection{Umsetzung Adobe Flash}


\section{HTML5 Frameworks for Animations, Games and more}
\begin{itemize}
  \item kineticjs (http://www.kineticjs.com) f
  \item fabricjs (http://fabricjs.com/) f
  \item paperjs (http://paperjs.org/) f
  \item createjs (http://www.createjs.com) f
  \item pixijs (http://www.pixijs.com/) f
  \item processingjs (http://processingjs.org/) f
  \item impactjs (http://impactjs.com/) €
  \item limejs (http://www.limejs.com/) f
  \item craftyjs (http://craftyjs.com/) f
  \item canvasengine (http://canvasengine.net/) f
  \item threejs (http://threejs.org/) f
  \item bhive (http://www.bhivecanvas.com/) f
  \item cakejs (https://code.google.com/p/cakejs/) f
  \item phaser.io (http://phaser.io/) f
  \item twojs(http://jonobr1.github.io/two.js/) f
\end{itemize}