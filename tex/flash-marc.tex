% Marc Frydetzki - Flash und HTML5 im Vergleich
In diesem Kapitel wird zunächst anhand der Geschichte von Flash die Entstehung
und Veröffentlichungen bis zum aktuellen Stand beschrieben. Im Anschluss
wird die Programmiersprache ActionScript, spezifische Dateiformate,
Einsatzbereiche und häufig verwendete Techniken vorgestellt. Anschließend
werden Vor- und Nachteile genannt und ein Fazit abgegeben.

\section{Die Geschichte von Flash}
Als Jonathan Gay und Charlie Jackson -- der Gründer von Silicon Beach
Software -- 1993 FutureWave Software gründeten, war ihnen sich nicht bewusst,
mit dieser Firma und dem 1994 erschienenen vektorbasierten Zeichenprogramm
SmartSketch den Grundstein für das einmal am meisten verbreitete Freeware
Programm der Welt gelegt zu haben.
SmartSketch wurde aus der Intention heraus entwickelt, den Arbeitsprozess für
Gestalter zu vereinfachen, indem die Möglichkeit geschaffen wurde, mittels
Grafiktablets Zeichnungen -- ähnlich dem Zeichnen auf Papier -- zu erstellen.
Jonathan Gay wollte eine Software entwickeln, die es dem Anwender erlaubt,
artgemäß dem Erstellen von Lego-Objekten Schritt für Schritt im
Gestaltungsprozess Fortschritte zu machen. Dieses Schachtelungsprinzip findet
sich noch heute in Flash wieder. Im Sommer 1995 erhielt FutureWave
zunehmend positive Resonanz zu SmartSketch sowie Vorschläge das Grafikprogramm
zu einem Animationsprogramm weiterzuentwickeln. Zur selben Zeit trat ein neues
Medium in das Auge der Öffentlichkeit, das Internet. Gay und seine Kollegen
sahen darin das Potential, dass ein Großteil der zukünftigen Internetnutzer
reges Interesse daran haben könnte, Grafiken und Animationen anzubieten bzw.
auszutauschen.
Daraufhin erweiterte Gay SmartSketch kurzerhand um Animationsmöglichkeiten.
Mittels Java wurde ein Browserplugin entwickelt, das die erstellten
Animationen, die im Splash Format (SPL) vorlagen, im Browser abspielen konnte.
Im Herbst 1995 erschien der Netscape Browser mit dem ``advanced programming
interface'', welche es erlaubte den Browser zu erweitern. Auf Grund des
Wiedererkennungswertes wurde SmartSketch erst in CelAnimator und später in
FutureSplash Animator umbenannt.
Der größte Erfolg stellte sich im August 1996 ein, als Microsoft für seine
MSN Web-Version ein fernsehähnliches Erlebnis schuf und das zur Umsetzung
eingesetzte Programm FutureSplash Animator war. Ein weiter großer Kunde
war Disney Online, welcher FutureSplash nutzte, um Animationen für den
Bezahldienst ``Disneys Daily Blast'' zu erstellen. Dadurch, dass Disney zum
selben Zeitpunkt den Macromedia Shockwave Player nutzte, kam über diesen
Umweg der Kontakt mit Macromedia zustande. Im Dezember 1996 kaufte Macromedia
FutureWave und aus FutureSplash Animator wurde Macromedia Flash 1.0. Im
Zuge der Produktumbenennung wurde aus dem Abspielformat Splash das SWF.
Seit dem Verkauf an Macromedia erschienen im Jahresrhythmus neue Versionen
von Flash. Nach und nach stieg der Funktionsumfang und dem Gestalter
wurden mehr Möglichkeiten zur Auslebung seiner Kreativität gegeben. Im Jahr
1999 erschien Version 4, die zum ersten Mal die so genannte Programmiersprache
ActionScript integrierte. Es konnten Variablen, bedingte Anweisungen, sowie
Schleifen zur Programmsteuerung eingesetzt werden. Dadurch war der Weg
geebnet für mehr als nur reine Animationsentwicklung. Eingabetextfelder
ermöglichten es komplexe Formulare zu erstellen und die eingegebenen
Daten mittels dynamischer Webseiten zu verarbeiten. Im Jahr 2000 erschien
Version 5 in der ActionScript stark verändert und an den ECMAScript-Standard
angepasst wurde. JavaScript-Entwicklern sollte dadurch ein schnellerer
Einstieg in Flash ermöglicht werden, da JavaScript auf diesem Standard
aufbaute. Eine weitere Neuerung war der Debugger, der den Entwicklungsprozess
und die damit verbunden häufige Fehlersuche in Flash-Projekten stark
vereinfachte. 2002 erschien Flash MX (Version 6) mit entscheidenden
Erweiterungen wie der Zeichnen-API, die es erlaubte dynamische Formen zu
erstellen. Zusätzlich war ein Videocodec integriert und die Unterstützung für
Unicode gegeben. Im Oktober 2003 kam Flash MX2004 auf den Markt und damit
ActionScript 2.0. Im Juni 2004 erscheint die aktualisierte Version Flash
MX 2004 7.2 in der Stabilitätsprobleme behoben und die Leistung stark
verbessert wurde. Mitte Juni stellt Macromedia die "`Flash Plattform"' vor,
die in erster Linie an Unternehmenskunden adressiert war. Am 8. August wurde
zusammen mit "`Studio 8"' Flash Professional 8 vorgestellt. Flash 8 enthielt
etliche Neuerungen, unter anderem wurde die Möglichkeit geschaffen
Rastergrafiken zu erstellen und zu manipulieren. Filter wie zum Beispiel
Gaußscher Weichzeichner, Schlagschatten oder Verzerrungen, Datei-Upload und
eine neue Text-Engine namens FlashType, Bitmap-Caching, einstellbares
Easing, ein neuer Videocodec mit Alphakanal-Unterstützung (Transparenz), ein
stand-alone Video-Encoder mit Stapelverarbeitung, sowie eine verbesserte
Programmieroberfläsche sind hier als die bedeutendsten zu nennen. Im selben
Jahr übernimmt Adobe Macromedia für 3,4 Milliarden US-Dollar. Adobe führt
die Bezeichnung Macromedia Flash bis zum nächsten Produktzyklus fort,
danach heißt das Produkt \emph{Adobe Flash}.
Im Juni 2006 erscheint der Adobe Flash Player 9 für Windows und Mac OS X mit
speziellen Anpassungen für eine bessere Integration in Adobe Flex 2. Eine
weitere Innovation stellt die Integration von ActionScript 3.0 dar. Des
Weiteren ist ein JustInTime-Compiler Bestandteil, welcher mit anderen
Verbesserungen einen großen Geschwindigkeitsvorteil bei der Scriptausführung
gegenüber den älteren ActionScript-Versionen mit sich bringt. Seit Mai 2007
ist Adobe Flash CS3 verfügbar, diese Version ermöglicht direkte Importe aus
Adobe Photoshop und Illustrator, was zu einer enormen Erleichterung und
Beschleunigung des Arbeitsprozesses führt. Im Jahr 2008 legt Adobe die
Spezifikation für Flash offen, was die komplette Indizierung durch
Suchmaschinen ermöglicht und somit ein großes Manko an SWF-Dateien beheben
soll. Mitte Oktober erscheint Adobe Flash CS4 und der Adobe Flash Player 10.
Am 7. Mai 2010 erscheint Adobe Flash CS5, das diverse Neuigkeiten zur
Prozessoptimierung bietet. Der Flash Builder ist nahtlos integriert und die
Code Hinweise, die so genannte IntelliSense, wurden stark verbessert. Ein
weiteres neues Feature sind die Code-Snippets, die häufig verwendete Code-
Bausteine zur Verfügung stellen. Videos können direkt ub der
Entwicklungsumgebung abgespielt werden, was der Interaktion mit Videos sehr
zu Gute kommt. Ein sehr hilfreiches Werkzeug ist die neue Text-Engine. Diese
erlaubt eine feinere Steuerung von Textelementen. Textfelder können per
Mausklick in Spalten eingeteilt und so miteinander verknüpft werden, dass
Text dabei ab einem bestimmten Punkt in ein anderes Textfeld überläuft. Ein
weiteres neues Feature ist die Unterstützung von XFL-Dateien, diese trennt
Code, Layout und Assets voneinander. Dadurch wird der Austausch von Daten,
z.B. Bildern, sehr vereinfacht.
% Absatz
Am 10. Juni 2010 erschien der Adobe Flash Player 10.1, der im wesentlichen
Umfang Stabilität, Performance und Ressourcennutzung optimierte, um somit
auch auf dem mobilen Endgerätemarkt bestehen zu können. Grundlegende 3D
Manipulationen und eine 3D-Zeichen API sind integriert worden. Eine große
Weiterentwicklung stellt das Auslagern von Prozessen auf die Grafikkarte dar.
Somit wird die CPU entlastet und die Wiedergabe von H.264 Videos (HD)
beschleunigt. Des Weiteren wurde die Speicherverwaltung überarbeitet und
Multitouch- sowie Lagesensoren-Unterstützung integriert. Letztere sind
vorwiegend in Smartphones integriert und ermöglichen neue Möglichkeiten der
Steuerung und Interaktion.
% Absatz
Mit der aktuellsten Flash Player-Version 10.3 stopft Adobe diverse kritische
Sicherheitslücken, verbessert die Audioqualität und ermöglicht das Löschen
von ``Local Storage Objects'' über das Browsermenü.
% sub
ActionScript
ActionScript, kurz AS, ist die Programmiersprache für die Adobe Player-
Laufzeitumgebung. Sie ist eine objekt-und ereignisorientierte
Programmiersprache, die speziell für die Webseitenanimation entwickelt wurde
und in direktem Zusammenhang mit der Entstehung von Flash zu nennen ist. Der
aktuellste Entwicklungsstand ist ActionScript 3.0 und stellt ein vielseitiges
Werkzeug zur Umsetzung interaktiver und vielseitiger Projekte dar.
Der ActionScript-Programmcode wird in das sogenannte Bytecode-Format kompiliert
und anschließend von der ActionScript Virtual Machine, kurz AVM, ausgeführt.
Ein sogenannter Compiler übernimmt die Umwandlung in Bytecode und bettet
diesen in eine SWF-Datei ein. Als Erfinder von ActionScript gilt Gary
Grossman. Er verließ die Firma zeitig um sich anderen Projekten zu widmen.
%sub
ActionScript 1.0
ActionScript wurde erstmals in Flash 4 veröffentlicht und entsprach einer
Sammlung von Aktionen die mittels "`zusammenklicken"' in einen logischen
Zusammenhang zu bringen waren. In dieser Version entspricht ActionScript
einer Mischung aus Perl und JavaScript. Für die Anwendung spezifische Werte
konnten bereits in Variablen gespeichert und mittels Kontrollstrukturen
gesteuert werden. In den Vorgängerversionen gab es auch Scripting
Unterstützung, aber diese basierte auf keinem grundlegenden Standard und
bezog sich nur auf einfache Zeitleistensteuerung. In der Flash 5 Version
handelte es sich bei ActionScript erstmals um eine vollwertige
Programmiersprache. Sie wurde an den ECMAScript 3 Standard angepasst, entsprach
diesem aber nicht zu 100\%. Zusätzlich wurden Events und Switch-Anweisungen
eingeführt, wodurch erstmals gezielt auf Nutzeraktionen reagiert werden konnte.
Eine weitere sehr nützliche Funktionalität waren \emph{Prototypes}, die eine
Vorstufe zu Klasse darstellen.
%sub
ActionScript 2.0
ActionScript 2.0 erschien im September 2003 zusammen mit dem Flash Player 7
und wurde mit Flash MX 2004 verfügbar. Die Entwicklergemeinde -- sie war von
Flash und ActionScript begeistert -- teilte den Flash-Entwicklern Änderungs-
und Erweiterungsvorschläge mit, wodurch ActionScript 2.0 für größere und
komplexere Projekte ausgelegt werden konnte. Unter anderem wurde eine
Laufzeutüberprüfung, klassenbasierte Syntax, die Schlüsselwörter \emph{class}
und \emph{extends} eingeführt und die Anpassung an den ECMAScript 4 Standard
vorgenommen. Als besonders gilt, dass ActionScript 2.0 in ActionScript 1.0
Bytecode abgespeichert wird, damit es auch mit dem Flash Player 6 kompatibel
ist, da dieser zu dem damaligen Zeitpunkt der am weitesten verbreitete war.
Die Auslegung auf objektorientierte Programmierung führte dazu, dass AS C++
und Java immer ähnlicher wurde. Die dadurch stark reduzierte Umstellungs-
bzw. Einarbeitungszeit machte die Programmiersprache einer größeren
Entwicklergemeinde zugänglich.
%sub
ActionScript 3.0
ActionScript 3.0 erschien im Juni 2006 zusammen mit dem Flash Player 9 und
wurde mit Flash 9/CS3 verfügbar. Diese Version basiert auf der ECMAScript 4
Sprachspezifikation, wurde vollkommen überarbeitet und ist für vollständige
objektorientierte Programmierung ausgelegt. AS 3.0 unterscheidet sich in
seiner Architektur und Konzeption weitestgehend von seinen Vorgängern. In
Folge dessen wurde die AVM neu konzipiert und für AS 3.0 in AVM2 umbenannt.
Als besondere Neuerung ist das Ereignismodell, das auf der DOM3-
Ereignisspezifikation basiert, zu beachten. Die verschiedenen visuellen
Elemente einer AS 3.0 Anwendung sind in der AnzeigeListe -- einer Art
Baumstruktur -- angeordnet. Jedes Element hat ein Vaterelement und eventuell
Kinderelemente. Das Ereignismodel spielt in der Nutzung der neuen einheitlichen
Klassen zum Laden von diversen Daten eine entscheidende Rolle.
Mit der URLLoader-Klasse lassen sich Text oder Binärdaten, z.B. XML-Dateien
oder PHP-Skripte, laden. Um Zustandsänderungen des Ladevorgangs abfangen zu
können, müssen relevante Ereignisse (Events) mit der Funktion
\emp{haddEventListener()} direkt an das Objekt der URLLoader-Klasse
"`angehängt"' werden. Während des Download-Vorgangs geben Benachrichtigungen
(Events) über den Fortschritt Auskunft. Die zwei Eigenschaften
\emph{bytesLoaded} und \emph{bytesTotal}, die als Eigenschaften des
\emph{ProgressEvent} abrufbar sind, bieten die Möglichkeit den Ladevorgang
prozentual darzustellen. Ist der Ladevorgang abgeschlossen, sind die geladenen
Daten über die Eigenschaft data des Events abrufbar.
Zum Laden von SWF-Dateien und Grafiken (JPG, PNG, GIF) ist die Loader-Klasse
vorgesehen, bei der die Event-Listener an die Eigenschaft
\emph{contentLoaderInfo}, ein Objekt das dem zu ladenden Datenobjekt
entspricht, angehängt werden.
%Absatz
Die neue URLStream-Klasse stellt noch während des Downloadvorgangs Zugriff auf
die Daten als unformatierte Binärdaten bereit, wodurch z.B. Informationen wie
der Dateityp und die Auflösung eines Bildes ermittelt werden können.
AS 3.0 unterstützt Pakete und Namespaces, wodurch besonders die Strukturierung
von Projekten verbessert und Namenskonflikte vermiedern werden.
Eine weitere sehr nützliche Neuerung stellt die XML-API dar, die auf der
E4X-Spezifikation beruht. XML-Daten werden nun als nativer Datentyp behandelt,
dies führt zu einer enormen Steigerung der Verarbeitungsgeschwindigkeit seitens
der AVM und zusätzlich zu einer einfacheren Verarbeitung und Nutzung von
XML-Daten.
Die Aufnahme der numerischen Grunddatentypen ``int'' als z.B. Schleifenzähler-
Variable und ``uint'' als z.B. Farbwert stellt im Verhältnis zu AS 2.0 nicht
nur für Programmierer ein nützliches Mittel dar. Die Vorteile der schnellen
Ganzzahlenarithmetik der CPU können damit genutzt werden und führen ihrerseits
zu einer kürzeren Verarbeitungszeit. In AS 3.0 wird ein strikte Typisierung
vorausgesetzt und die vom Programmierer gesetzten Datentypen bleiben zur
Laufzeit erhalten. Während des Kompilierens und zur Laufzeit wird eine
Datentypüberprüfung durchgeführt, wodurch die Datentypsicherheit des Systems
stark verbessert wird.
Alle textbezogenen Schnittstellen wurden im \emph{flash.text} Paket
zusammengefasst. Zusätzlich wurde die TextLineMetrics-Klasse hinzugefügt,
die sehr nützliche Informationen über eine einzelne Zeile oder ein Zeichen
liefert. Ab Flash CS5 und somit Flash Player Version 10 wurde die Text-Engine
durch das neue TLF-Text Format erweitert, wodurch Funktionalitäten geboten
werden, die Flash-Programmierern enorm viel Arbeit abnehmen. Es lassen sich
nahezu alle Text-Manipulationen durchführen, die in Layout-Werkzeugen schon
lange möglich sind. Zum Beispiel sind mehrspaltige Textfelder, Textfluss um
Objekte und Textfeld-Verknüpfungen, die "`überfließenden"' Text ermöglichen,
realisierbar. Die neue Sound-API bietet mittels SoundChannel- und SoundMixer-
Klasse umfangreiche Funktionen. Audio lässt sich nun synthetisieren,
manipulieren und abspielen. Darüber hinaus ist eine exakte Steuerung einzelner
und aller in einer Anwendung befindlicher Sounds möglich.
Im Bereich Video wird ab dem Flash Player 10.1 die Hardware beschleunigte
Dekodierung von H.264-Videos eingeführt. Erfüllt der Client die notwendigen
Voraussetzungen, wird die Dekodierung aud die Hardware des Clients
ausgelagert. Somit wird die CPU weniger beansprucht, was zu einer besseren
Performance des Flash Players und einer niedrigeren Systemauslastung führt.
Insbesondere auf mobilen Geräten wird dadurch der Akku geschont.

%sub
Dateiformate
%sub
.fla
Eine FLA Datei ist eine unkomprimierte Projektdatei und enthält alle
Ausgangsmaterialien und Animationen. Mit dem ensprechenden Flash
Autoringwerkzeug kann sie bearbeitet und zu einer kompiliert (veröffentlicht)
werden.
%sub
.swf
Eine SWF-Datei ist ein komprimiertes Flashprojekt (Bytecode) und kann von
einem Flash Player abgespielt werden. Bei der Bedeutung, was SWF ausgeschrieben
heißt, gibt es keine eindeutige Festlegung. Es gibt zwei Varianten, die in etwa
gleich viel Bedeutung finden. Zum einen \emph{ShockWaveFlash}, sie entspricht
der Bezeichnung, die Macromedia als 3D-Format der Director-Umgebung
vermarktet hatte. Die andere Variante ist \emph{SmallWebFormat}, was den
Eigenschaften des Formates, geringe Dateigröße und enorme Skalierbarkeit ohne
Qualitätsverlust, am ehesten entspricht. Die geringe Dateigröße und
Skalierbarkeit resultieren daraus, dass Grafiken, die in Flash erstellt wurden,
als Vektorgrafik abgespeichert werden.
%sub
.as
Eine ActionScript-Datei enthält ausschließlich ActionScript Anweisungen
und kann mit jedem Texteditor bearbeitet werden. Zum Beispiel kann jedes
Objekt in einem Flash-Projekt als Klasse in einer .as-Datei gespeichert werden.
Dieses Format ermöglicht eine bessere Strukturierung von Projekten, indem
Design und Quellcode getrennt gespeichert werden.
%sub
.mxml
MXML ist eine auf XML basierende Beschreibungssprache die von Adobe Flex und
dem Adobe Flash Builder verwendet wird, insbesondere lassen sich mit ihr
Komponenten beschreiben und mit Funktionalität versehen. Eingeführt wurde
MXML, um die Entwicklung von RIAs und den Einstieg für Softwareentwickler zu
vereinfachen.
%sub
Weitere Formate
.swd - Eine vom Flash Debugger erzeugte Datei die Debug-Informationen enthält.
Sie wird benötigt um Remote Debugging durchzuführen und ermöglicht Einblick
in die Struktur einer SWF
.asc - Eine Datei die Server-seitiges ActionScript enthält
.flv - Flash Video Format, ein Containerformat das ein Flashvideo enthält
.swc - Eine ZIP-Datei die mit dem Flash-Authoring-Tool erstellt wird und
Informationen über eine oder mehrere Flash-Komponenten enthält
.jsfl - Eine Datei die Anweisungen zur Erweiterung des Flash-Authoring-Tools
enthält
.flp - Eine XML-Datei die alle Informationen über ein Flashprojekt enthält.
.aso - Eine Datei die vom Flash Player erzeugt wird, um Local Shared Objects zu
speichern.