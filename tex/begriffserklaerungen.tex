% =Begriffserklaerungen
\chapter{Begriffserklärungen}

% Marc Frydetzki - Flash und HTML5 im Vergleich
Bevor in den nächsten Kapitel die zwei Technologien Flash und HTML5
vorgestellt und im Anschluss anhand einer prototypischen Anwendung
verglichen werden, wird in diesem Kapitel zunächst die Geschichte des
Internets vorgestellt. Beide Technologien nehmen einen großen Stellenwert
in den Diskussionen um dieses Medium ein, wodurch diese Informationen
grundlegend erforderlich sind. Im Anschluss werden die Arbeitsgruppen die
sich mit HTML5 beschäftigen und Begriffe der Thematik Web 2.0 erläutert.

\section{Die Geschichte des Internets}

Der Name Internet kommt von dem englischen Begriff ``Interconnected set of
networks'', was soviel bedeutet wie "`miteinander verbundene Netzwerke"'.
Das Internet ist heute ein weltumspannendes Netz von vielen einzelnen
Computernetzwerken.
Durch die Erfindung des elektrischen Stroms im Jahre 1730 war der
Grundstein für die elektrische Telegrafie gelegt, welche Samuel Thomas
von Soemmerring im Jahr 1809 erfand.
Die Erfindung der elektromagnetischen Induktion im Jahr 1832 durch Michael
Faradey führte zu den ersten Versuchen mit einem elektromagnetischen
Telegrafen, wodurch 1833 die erste telegrafische Nachrichtenübertragung
gelang.
Ab diesem Zeitpunkt befassten sich mehrere Personen mit der Erforschung und
Entwischlung von Apparaturen zur Übertragung von akustischen Signalen.
Schlussendlich war es Alexander Graham Bell - im Jahr 1876 -, der die
Fertigkeiten zusammen brachte, das Telefon über eine Versuchsapparatur
hinaus als Gesamtsystem zur Marktreife zu bringen.
Die Datenübertragung mittels Impulsen, die durch Kabel übertragen werden,
stellt bis heute das wichtigste Transportmittel von Informationen dar.
Aufgeschreckt durch \emph{Sputnik}, dem ersten von der UdSSR ins All
beförderten geostationären Satellit, gründete das Verteidigungsministerium
der USA die``Advanced Research Projects Agency''. Die ARPA hatte die Aufgabe
neue Technologien im Bereich der Datenübertragung und Kommunikation zu
entwickeln.
Das führte Ende der 1950er Jahre, durch J. C. R. Licklider und seinem
Forschungsteam, zur Entwicklung des ersten ``Time-Sharing-Systems'' der Welt.
Dabei war ein Zentralrechner mit sternförmig angeschlossenen Terminals
verbunden, wodurch mehrere Nutzer gleichzeitig dessen Rechenleistung nutzen
konnten. Damit wurde es ermöglicht große geografische Distanzen mit Terminals
eines Herstellers zu überwinden, allerdings war die Anzahl der Anschlüsse
begrenzt. Für militärische Zwecke eignete sich dieser Netzaufbau jedoch nicht,
weil eine Störung des Zentralrechner den Ausfall des gesamten Netzes bedeutete.
Ein weiterführender Ansatz war das dezentrale Netzwerk, mit dem sich Paul
Baran beschäftigte. Hierbei wurden mehrere Zentralrechner, die jeweils über
die sternförmig angeschlossenen Terminals verfügten, miteinander verbunden.
Fiel ein Zentralrechner aus, war das Netzwerk im Ganzen war geschwächt, aber
immer noch in Teilen nutzbar. Um bei einem Netzwerkschaden so wenige Terminals
wie möglich zu verlieren, entwickelte Pauk Barann das ``Distributed Network''
bei dem der Zentralrechner überflüssig wurde. Jedes Terminal, das mit dem
Netzwerk verbunden war, hat alle Funktionen an Board, die für die
Kommunikation im Netzwerk notwendig sind und galt somit als Computer. Ein
weiterer Vorteil mehrerer Verbindungen zu einem möglichen Zielcomputer
führte zur Entwicklung des ``Packet-Switching'', wobei eine Datei nicht mehr
als Ganzes übertragen, sondern in viele Datenpakete gleicher Größe zerteilt
und einzeln transportiert wird. Staus werden somit vermieden, denn pro Paket
kann die Route neu berechnet werden und muss bei einer fehlerhaften
Übertragung nicht noch einmal als Ganzes übertragen werden. Diese intelligente
Übertragung beschleunigte und entlastete das Netzwerk wesentlich. 1965 wurde
diese Art der Netzkommunikation zum ersten Mal eingesetzt. 1966 begann
die Planung zur Vernetzung aller über das Land verteilten Computerzentren
der ARPA nach Paul Barans ``Distributed Network''. Gegen Ende 1969 wurden
nacheinander die großen Computerzentren verbunden.
Die angestrebte Unabhängigkeit von Herstellern und Betriebssystem führte zur
Entwicklung des ``Interface Message Processor'', kurz IMP, durch Wesley Clark.
Der IMP ist ein Minicomputer mit einem einheitlichen Netzwerkprogramm, der an
jeden Großrechner angeschlossen wurde unf dür die Datenübermittlung zuständig
war. Durch diese Verbindung der großen Computerzentren der ARPA enstand das
ARPANET und zum ersten Mal waren Rechner verschiedenster Art miteinander
verbunden. Die ersten Anwendungen im ARPANET waren ``Telnet'' und ``FTP''
bevor Ray Tomlinson 1972 eine Software zum Verschicken und Empfangen
elektronischer Post veröffentlichte. Der E-Mail-Dienst war geboren und
führte zu einem sprunghaften Anstieg der Nutzerzahlen. Weitere Verbreitung
erfuhr das ARPANET durch Vorführung auf der International Conference on
Computer Communications und Weitergabe des Wissens unter anderem an die NASA,
die Air Force und Universitäten. Darauf folgte das ALOHANET, das
Forschungsstationen auf Hawaii vernetzte und sich aufgrund störungsanfälliger
Telefonleitungen zum PRNET (Packet Radio Network) weiterentwickelte, in dem
Daten per Radiowellen übertragen wurden. Als weitere Kommunikationsmöglichkeit
wurde 1973 das SATNET (Datelliten Network) entwickelt. Die Nachteile der
verschiedenen Datenübertragungsarten führten schließlich 1974 Vinton Cerf und
Bob Kahn zur Entwicklung eines umfassenden Netzwerkprotokolls, dass sich auf
ein einheitliches Datenformat und eine einheitliche Verbindungsmethode
beschränkte, dem Transmission Control Protocol / Internet Protocol, kurz
TCP/IP. 1983 wurde das TCP/IP zum Standard erklärt und kommt noch heute in der
Form zum Einsatz. Im selben Jahr entstand das MILNET (Military Network) in
das der komplette militärische Bereich des ARPANET verlagert wurde. Das
ARPANET diente ab dem Zeitpunkt nur noch rein zivilen Gruppen. Dieser
Schritt war notwendig geworden, weil sich immer mehr internationale
Netzwerke anschlossen und somit die Unabhängigkeit des Militärs in Gefahr
geriet. Bis 1985 wurden die bestehenden Netzwerke vorrangig von kleineren und
größeren Nutzergruppen zur Kommunikation und zum Austausch ihrer akademischen
Forschungsthemen genutzt.
1985 gab es einen entscheidende Änderung. Die Netzwerkzugänge des 1984
gegründeten JANET (Joint Academic Network) und des NSFNET (National Science
Foundation Network) wurden für alle Benutzer -- egal aus welchem
Forschungsbereich -- geöffnet. Dadurch erlangte das NSFNET immer mehr an
Beliebtheit und übernahm schließlich 1990 die Funktion des ARPANET. In
diesem Schritt wurde das ARPANET eingestellt. Im selben Jahr schlossen sich
die Netzwerke von Canada, Dänemark, Finnland, Frankreich, Island, Norwegen und
Schweden an das NSFNET an. 1991 folgten dann Deutschland, Japan, Niederlande
und das Vereinigte Königreich.
Der Dienst der in der heutigen Zeit von vielen Menschen mit dem Internet
gleichgesetzt wird ist das World Wide Web. Es ist 1990 aus einer
Weiterentwicklung eines etwas älteren Projekts von Tim Berners-Lee -- er war
in den 80er und 90er Jahren Informatiker am Hochenergieforschungszentrum
CERN -- hervorgegangen und führte zu einer rasant ansteigenden Beliebtheit
des Internet.

\section{Browser}
Ein Browser ist ein spezielles Programm, das zur Darstellung von Inhalten aus
dem Internet (World Wide Web) genutzt werden kann. Er übersetzt die im
Quellcode verwendeten Befehle in das entsprechende Aussehen. Browser sind
die Benutzeroberfläche für Webanwendungen, aber auch andere Daten und
Dokumente können betrachtet werden. Es gibt mehrere Hersteller, die die
Konventionen des W3C unterschiedlich umsetzen. Diese auseinander gehenden
Herangehensweisen führen dazu, dass Webseiten in verschiedenen Browsern
unterschiedlich dargestellt werden. In unterschiedlichen Versionen eines
Browsers können Konventionen unterschiedlich umgesetzt worden sein, was die
versionsübergreifende Entwicklung für Programmierer zu einer enormen
Herausforderung macht. Zu den Merkmalen eines guten Browsers zählen
Schnelligkeit, Leistungsbedarf auf dem Rechner des Anwenders und die
Umsetzung der neuesten Innovationen im World Wide Web.

\section{W3C}
Das W3C ist ein internationales Konsortiom, in dem Mitgliedsorganisationen, ein
fest angestelltes Team und die Öffentlichkeit gemeinsam daran arbeiten,
Web-Standards und Richtlinien zu entwickeln -- daher der Name World Wide Web
Consortium, kurz W3C. Gegründet wurde es am 1. Oktober 1994 von Tim
Berners-Lee, dem Erfinder des World Wide Web. Die Entstehung des W3C ist eng
mit der Entstehung des World Wide Web verbunden. Die große Gefahr von
Inkonsistenten in der Nutzung vorliegender Technologien könnte zu unwirksamen
Verknüpfungen führen und das galt es möglichst zu verhindern. Die
Grundherangehensweise an neue Technologien oder Teilaspekte besteht darin, den
kleinsten gemeinsamen Nenner zu finden und diesen zu einer Spezifikation zu
verarbeiten, so dass diese von allen Mitgliedsorganisationen unterstützt wird.

\section{WHATWG}
Die Web Hypertext Application Working Group, kurz WHATWG, ist eine
Arbeitsgruppe, die von mehreren Unternehmen, darunter unter anderem Mozilla
Foundation, Opera Software ASA und Apple Inc., betrieben wird und deren Ziel
die Entwicklung neuer Technologien zur einfacheren Erstellung von
Internetanwendungen ist. Die WHATWG wurde 2004 gegründet, weil aus Sicht der
Browser-Hersteller das W3C die Entwicklung von HTML -- zu Gunsten von XHTML
-- vernachlässigte und auf die Bedürfnisse von Programmierern zu wenig
einging. Als Unterschied zu den Bestrebungen der W3C muss festgehalten werden,
dass die WHATWG keinen festen Standard als Zielsetzung verfolgt. Vielmehr soll
HTML weiterentwickelt werden, ohne aber auf einen definierten Stand hin zu
arbeiten. Anders das W3C, dessen aktuelle Bestrebungen ganz klar auf HTML5
ausgelegt sind.

\section{Plattformunabhängigkeit}
Plattformunabhängigkeit bedeutet bei Software, dass sie ohne extra Aufwand und
ohne Hilfssoftware auf unterschiedlichen Systemen ausgeführt werden kann.
Genauer betrachtet ist darunter die Eigenschaft eines Programms zu verstehen,
auf unterschiedlichen Computersystemen mit Unterschieden in Architektur,
Prozessor, Compiler, Betriebssystem und weiteren Dienstprogrammen, die zur
Übersetzung oder Ausführung notwendig sind, lauffähig zu sein.

\section{Rich Internet Application}
Der Begriff ``Rich Internet Application'' kurz RIA (reichhaltige
Internetanwendung) wurde von Macromedia im Jahre 2002 erstmalig verwendet.
Damalige Internetangebote entwickelten sich zunehmend vom reinen statischen
Wiedergeben von Inhalten zu vom Nutzer manipulierbaren dynamischen Inhalten,
wodurch der Begriff \emph{Web 2.0} entstand. Der Begriff RIA
bezeichnet ein Konzept und basiert meist auf mehreren Web-Technologien
wie zum Beispiel HTML, CSS, JavaScript und AJAX oder Flash und PHP. Um als RIA
zu gelten, muss eine Anwendung über das Internet verfügbar sein. Allerdings
muss sie nicht zwangsläufig im Browser ausgeführt werden, sondern kann
auch als Desktopanwendung zum Einsatz kommen. RIAs im heutigen Sinn bieten dem
Anwender Möglichkeiten, die früher Desktopanwendungen zur Verfügungen stellten.
Die Individualisierung und Vereinfachung der Internetnutzung spielt eine
entscheidende Rolle bei der Entwicklung neuer Anwendungen. Unter Interaktion
versteht man beispielsweise das Ändern der Gestaltung, das Erzeugen und
Verwalten von Daten, Drag\&Drop-Funktionalitäten, sowie die Verwendung von
Tastenkürzeln.