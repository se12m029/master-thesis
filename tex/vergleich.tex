% =VERGLEICH

% Links
% http://venturebeat.com/2012/01/31/html5-versus-flash-infographic/
% http://www.periscopic.com/#/news/2011/05/our-research-into-flash-and-html5-which-one-is-right-for-your-project/
% http://www.actionscript.org/resources/blogs/53/A-good-comparison-between-HTML5-and-Flash.html
% http://blog.accusoft.com/posts/2012/october/html5-vs-flash-what-do-you-need-to-know-part-1.html
% http://blog.accusoft.com/posts/2013/january/html5-vs-flash-infographic.html
% http://blog.accusoft.com/posts/2012/october/html5-vs-flash-what-do-you-need-to-know-part-2.aspx
% http://readwrite.com/2010/03/09/does_html5_really_beat_flash_surprising_results_of_new_tests#awesm=~obelLBPH3UgwgC

\chapter{HTML5 Canvas und Adobe Flash: Vergleich und mögliche Auswirkungen}

\section{Vergleich von HTML5 Canvas und Adobe Flash}
Proprietär vs Standard
\subsection{Verfügbarkeit}
\subsection{Audio und Video}
\subsection{Animation}
\subsection{Spiele}
\subsection{Werbung}
\subsection{Web-Applikationen}

\section{Mögliche Auswirkungen}
\subsection{Der Wandel von dem allgegenwärtigen Flash zu den neuen Webstandards}
\subsection{Die Zukunft von HTML5 und Flash}
\subsection{Unternehmen und ihre Einstellung zu neuen Ideen}

\chapter{Auswirkungen}
\section{Browser}
\subsection{Chrome und Safari (WebKit)}
Chrome von Google und Safari von Apple bauen beide auf der HTML-Rendering-Engine
WebKit zur Darstellung von Webinhalten auf. Google Chrome konnte zunehmend mit
dem bis dato meist genutzen Webbrowser Internet Explorer konkurrieren, bis es
nach den Angaben des globalen Statistiksunternehmens StatCounter erstmals im Mai
2012, weltweit die Spitzenposition einnehmen konnte. Mit einem
durchschnittlichen Anteil von 35\% für Google Chrome und 7\% für Apple Safari
stellen diese die beliebtesten Webbrowser dar.
% http://gs.statcounter.com/#browser-ww-monthly-201204-201304-bar
Im Vergleich decken beide Browser den größten Teil der HTML5 Unterstützung ab.
% Vergleiche HTML5 Unterstützung - http://html5readiness.com/

Die WebKit Engine wurde auf der Grundlage des KDE-Projekts KHTML als Open Source
Projekt von Apple entwickelt. Apple ist sehr daran interessiert, dass HTML5 so
schnell wie möglich auf allen Geräten verfügbar ist, um den Benutzern ein
uneingeschränktes Internet anbieten zu können. Auch die Mobile Safari Variante
auf den iDivices basieren auf der WebKit-Engine.

\subsection{Firefox (Gecko)}
Der Open Source Webbrowser Firefox von Mozilla stellt im deutschsprachigen Raum
den meist genutzten Webbrowser dar. Weltweit betrachtet befindet sich Firefox
mit einem durchschnittlichen Anteil von 23\% hinter dem Internet Explorer von
Microsoft und Google Chrome.
Mozilla zeichnete sich schon früh als Unterstützer des HTML5 Standards ab. So
war es schon mit sehr frühen Versionen von Firefox möglich, das Video-Element zu
nutzen.

\subsection{Internet Explorer (Trident)}
Der wohl bekannteste Webbrowser Internet Explorer von Microsoft ist bereits in
der zehnten Version veröffentlicht worden. Schon mit dem Internet Explorer 9 war
Microsoft darauf bedacht die Geschwindigkeit ihres Browsers zu verbessern und es
wurde darauf verzichtet, eigene Standards durchzusetzen.
Erstmals orientierte man sich gänzlich an den Spezifikationen und unterstütze
bereits einige HTML5 Features.
So wurde das Audio/Video-Element unterstützt, wobei nur auf proprietäre Codecs
wie H.264 und MP3/AAC zurückgegriffen wurde. Seit dem Internet Explorer 10 wird
auch der freie Codec WebM unterstützt.

Microsoft hatte sich aus seinem größten Manko einen Vorteil verschafft: Dadurch,
das der Internet Explorer nur auf Windows Geräten läuft, kann er die ganze
Plattform und Hardware nutzen. Im Gegensatz zur Browsersoftware der Konkurrenz,
die auf den kleinsten gemeinsamen Nenner zurückgreifen müssen.
So kann Internet Explorers JavaScript Engine Chakra, die Leistung von Multicore
Prozessoren ausnutzen.

Mit dem Internet Explorer 10 hat Microsoft einiges aus der Vergangenheit
aufgeholt, allerdings sollten diese neuen Features nicht überbewertet werden,
denn die Einsatzfähigkeit von HTML5 zeichnet sich weniger durch die neuen
Browser, sonder viel mehr durch die Anzahl an veralteten Browser aus. Solange
die alten Versionen des Internet Explorer im Gebrauch sind können viele
Neuheiten von HTML5 nicht sinnvoll und ohne Fallbacklösung verwendet werden.