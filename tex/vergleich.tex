% =VERGLEICH

% Links
% http://venturebeat.com/2012/01/31/html5-versus-flash-infographic/
% http://www.periscopic.com/#/news/2011/05/our-research-into-flash-and-html5-which-one-is-right-for-your-project/
% http://www.actionscript.org/resources/blogs/53/A-good-comparison-between-HTML5-and-Flash.html
% http://blog.accusoft.com/posts/2012/october/html5-vs-flash-what-do-you-need-to-know-part-1.html
% http://blog.accusoft.com/posts/2013/january/html5-vs-flash-infographic.html
% http://blog.accusoft.com/posts/2012/october/html5-vs-flash-what-do-you-need-to-know-part-2.aspx
% http://readwrite.com/2010/03/09/does_html5_really_beat_flash_surprising_results_of_new_tests#awesm=~obelLBPH3UgwgC

\chapter{HTML5 Canvas und Adobe Flash: Vergleich und mögliche Auswirkungen}

\section{Vorraussetzungen}
Sowohl Nutzer als auch Entwickler benötigen um HTML5 und Adobe Flash einsetzen
zu können verschiedene Tools. Auf Entwicklerseite sind die Möglichkeiten sehr
vielfältig, sodass eine vollständige Auflistung den Rahmen dieser Arbeit
sprengen würde. Auf Nutzerseite ist die Auswahl der Tools ebenfalls breit
gefächert, allerdings ist die Anzahl der notwendigen Programme vergleichweise
geringer und kann daher genauer untersucht werden.
\newline\newline
Die Entwicklung von Flash Anwendungen setzt einen Editor vorraus, der in der
Lage ist ActionScript Code zu kompilieren. Dabei gibt es eine umfangreiche
Auswahl an verschiedenen Editoren, die nicht zwingend von Adobe kommen müssen.
Adobe bietet mit Adobe Flash CS6 eine Entwicklungsumgebung, welche sämtliche
Funktionen von Flash und ActionScript unterstützt und dem Entwickler neben dem
Editieren des ActionScript Codes eine grafische Oberfläche bietet, die es
vereinfacht, Animationen zu erstellen oder Grafiken einzubinden, allerdings
auch sehr kostspielig ist. Über die kostenlos erhältlichen Editoren können
zwar alle Funktionen von ActionScript genutzt werden, aber es werden keine
Tools zur Erstellung von Grafiken und Animationen angeboten.
\newline\newline
Im Vergleich dazu benötigen HTML5 Entwickler lediglich einen Editor, mit dem
textbasierte Inhalte erstellt werden können und einen Browser, der diese
Inhalte korrekt darstellt. Die Lösungen hierfür sind nahezu grenzenlos,
weshalb Entwickler aufgrund ihrer Präferenzen die für sie beste Lösung wählen
können. Der komplette Aufbau der Webseite wird im Editor in der entsprechenden
HTML5 Syntax geschrieben. Multimediale Inhalte lassen sich ebenfalls über das
Markup einbinden, und können zuvor mit den entsprechenden Programmen den
Wünschen angepasst werden. Adobe ist dabei eines der ersten Unternehmen, die
Softwarelösungen bieten, mit denen Webseiten und Animationen in HTML5 mittels
einer grafischen Oberfläche erstellt und bearbeitet werden können. In Zukunft
ist es möglich, dass andere Unternehmen diese Idee aufgreifen und ähnliche
Programme anbieten werden, sodass HTML5 im Bezug auf grafische
Entwicklungsumgebungen Adobe Flash in nichts nachstehen wird.
\newline\newline
Die technischen Vorraussetzungen für Nutzer von Flash-Inhalten sind relativ
einfach zu erfüllen. Der Nutzer müssen lediglich eine aktuelle Version des
Adobe Flash Players auf seinem System installiert haben, sei es als Browser
Plug-In oder als Standalone Version, und das verwendete System muss die
Hardware-Mindestanforderungen von Adobe erfüllen. Da diese
Mindestanforderungen äusserst gering sind, sollten alle aktuellen Systeme,
unabhängig ob mobil oder Desktop, in der Lage sein Flash-Inhalte abzuspielen.
Da der Adobe Flash Player auf jedem gängigen System verwendet werden kann,
ist auch die Entwicklung über mehrere Plattformen in der Regel kein Problem.
\newline\newline
Für die Nutzung von HTML5 Inhalten wird lediglich ein Browser vorrausgesetzt,
der diese korrekt darstellen kann. Der Entwickler hat dafür zu
sorgen, dass jeder Browser die Seite soweit korrekt darstellt, sodass der
Nutzer auf die Inhalte zugreifen kann. Das Problem hierbei ist, dass der
Browsermarkt sehr stark fragmentiert ist und diese sowohl von Hersteller zu
Hersteller als auch von Version zu Version eine unterschiedlich
fortgeschrittene Unterstützung von HTML5 anbietet. Die Hardware-
Mindestanforderungen sind dabei an den jeweiligen Browser gebunden. Allerdings
sind auch diese sehr gering wodurch auch schon auf mobilen Systemen Browser
integriert sind die eine sehr fortgeschrittene Unterstützung von HTML5
gewährleisten.

\section{Betriebssysteme}
HTML5 ist ein offener Standard. Das bedeutet, dass die Technologie nicht
von einem großen Softwarehersteller entwickelt wird und somit für die
Öffentlichkeit frei zugänglich ist. Das hat zur Folge, dass der Grad der
Unterstützung von HTML5 abhängig von der Einbindung in den Browser der
jeweiligen Browserhersteller ist. Jedes System mit einer aktuellen Version
eines Browsers kann damit die Funktionen von HTML5 verwenden und darstellen.
\newline\newline
Adobe Flash stellt das extreme Gegenteil zu HTML5 dar. Die Technologie ist
eine von Adobe entwickelte proprietäre Software und setzt vorraus, dass für
Browser ein Plug-In und für das System der Adobe Flash Player installiert sein
muss, um Flash-Inhalte darstellen zu können. Damit hat Adobe die Macht über die
Entscheidung der zu unterstützenden Plattformen. Sollte also Adobe sich dazu
entscheiden, eine bestimmte Plattform nicht mehr zu unterstützen, haben die
Nutzer dieser Plattform keine Möglichkeit mehr Flash-Inhalte zu konsumieren.
Apple hat sich aus diesem Grund dazu entschieden, auf ihren mobilen Engeräten
komplett auf die Unterstützung von Adobe Flash zu verzichten und ihre
Aufmerksamkeit auf offene Standard wie HTML5, CSS3 und JavaScript zu richten.
\newline\newline
Die gängigsten Desktop-Betriebssysteme (Windows, MacOS und Linux)
unterstützen sowohl HTML5, mittels diverser Browser, und Adobe Flash, mittels
Plug-In oder Player, und können damit Inhalte beider Technologien darstellen.
Bei mobilen Betriebssysteme sind immer vorinstallierte Browser vorhanden,
die einen Großteil der Features von HTML5 unterstützen. Für die Unterstützung
von Adobe Flash müssen zusätzliche Applikationen installiert werden.
\newline\newline
Um Flash-Inhalte auf Apples mobilen Endgeräte nutzen zu können, bietet Adobe
eine Softwarelösung, wodurch Flash Inhalte in HTML5 Inhalte umgewandelt werden.
Allerdings gilt es dabei zu beachten, dass die umgewandelten Inhalte nur
selten den originalen Inhalten entsprechen, aufgrund der fehlenden Funktionen
seitens HTML5 oder der hohen Komplexität der Inhalte. Adobe hat bereits
angekündigt, die Entwicklung des Adobe Flash Players für mobile Plattformen
einzustellen und sich auf Desktop Plattformen zu konzentrieren. Mit dieser
Entscheidung seitens Adobe in Zukunft der mobile Bereich sicherlich HTML5
gehören.

\section{Browserunterstützung}
Wie im letzten Abschnitt bereits beschrieben, ist die Darstellung von
HTML5-Inhalten abhängig von dem genutzten Browser. Unter den Browsern
gibt es eine breite Auswahl, die alle eine unterschiedlich fortgeschrittene
Unterstützung der Funktionen und Features von HTML5 bieten. Das führt
automatisch zu dem Problem, dass eine Webseite in unterschiedlichen
Browsern unterschiedlich dargestellt wird. Wie weit die Darstellung
dabei vom Original abweicht hängt von dem jeweiligen Browser ab.
Das stellt eines der größten Probleme im Bereich der Webentwicklung
mit den offenen Standards dar. Ein Entwickler muss sich darum bemühen,
dass eine möglichst große Anzahl an Nutzern die angebotene Seite/Applikation
optimal nutzen kann.
\newline\newline
An dieser Stelle soll daher ein Überblick darüber gegeben werden, welche
Browser welche Features und Funktionen von HTML5 unterstützen und der
Vergleich aufgestellt werden, ob diese mit Adobe Flash auch
erstellt/dargestellt werden können.
%caniuse grafik

\section{Leistungstests}
Bei der Auswahl einer Technologie ist neben den gebotenen Funktionen, dem damit
verbundenen Entwicklungs- und Lernaufwand und den Kosten auch die
Leistungsfähigkeit auf verschiedenen Betriebssystemen und Endgeräten
entscheidend. Speziell die Anzeigerate bei Videos und Animationen sowie die
Belastung des Prozessors spielen eine tragende Rolle, da die Anzeigerate
möglichst konstant und hoch und zeitgleich die Belastung des Prozessors
möglichst gering sein soll.
\newline\newline
Um diese Faktoren austesten zu können werden im Internet verschiedene
Tests angeboten, die eine bestimmte Situation simulieren und die
entsprechenden Ergebnisse liefern. Diese Tests, auch bekannt unter dem Namen
``Benchmarks'', decken dabei einen möglichst großen Testbereich ab. So reichen
diese von der Darstellung von Grafiken über das Berechnen von Partikeln
bis hin zu einem Belastungstest des Prozessors.
\newline\newline
Die Benchmark Ergebnisse können dabei lediglich als Richtwert angesehen
werden, da diese abhängig von dem beim Test verwendeten Betriebssystem und der
verwendeten Software (Browserhersteller und Version bzw. Version des Flash
Players) stark voneinander abweichen können.
\newline\newline
In den folgenden Abschnitten werden verschiedene Benchmarks verwendet, um
die Leistung von HTML5 und Adobe Flash möglichst genau zu vergleichen.
Bei den Benchmarks wird darauf geachtet, dass Funktionen verwendet werden,
die sowohl HTML5 als auch Adobe Flash zur Verfügung stellen damit für beide
Technologien die gleichen Bedingungen bestehen.

\subsubsection{Testumgebung}

Um sämtliche am Markt verfügbaren Browser zu testen, werden zwei
unterschiedliche Testsysteme verwendet. Die Testsysteme werden wie folgt
spezifiziert:
\begin{itemize}
  \item Gerät: Macbook Air (13 Zoll, Mitte 2012)
  \item Betriebssystem: OS X Version 10.9.1 (Mavericks)
  \item Prozessor: 2 GHz Intel Core i7
  \item Speicher: 8 GB 1600 MHz DDR3
  \item Grafikkarte: Intel HD Graphics 4000 1024 MB
  \item Festplatte: Flash-Speicher (SSD)
\end{itemize}
und
\begin{itemize}
  \item Gerät: HP Compaq nc6400 (13 Zoll)
  \item Betriebssystem: Windows 7 Professional, Service Pack 1
  \item Prozessor: 2 GHz Intel Core Duo 2 T7200
  \item Speicher: 2 GB
  \item Grafikkarte: ATI Mobility Radeon X1300 256 MB
  \item Festplatte: HDD
\end{itemize}

\newline\newline
Die Auswahl der Browser wird wie folgt spezifiziert:
\begin{itemize}
  \item Firefox 26.0
  \item Chrome 32.0.1700.102
  \item Opera 19.0.1326.47
  \item Safari 7.0.1 (nur OS X)
  \item Internet Explorer 11 (nur Windows)
\end{itemize}

\newline\newline
Die getestete Version vom Adobe Flash Player ist:
\begin{itemize}
  \item Adobe Flash Player 12.0.0.38
\end{itemize}

\subsection{GUIMark 2: The rise of HTML5}
% http://www.craftymind.com/guimark2/
GUIMark 2 ist ein kostenloser Benchmark entwickelt von Sean Christmann für
HTML5 und Adobe Flash. Christmann ist davon überzeugt, dass bei einer vollen
Ausschöpfung der Rendering Pipelines einfacher entschieden werden kann, welche
Technologie am Besten für interaktiven Inhalt im Web genutzt werden soll.
Laut Christmann tendieren Entwickler hauptsächlich dazu die Nutzbarkeit der
Technologien zu vergleichen, während in Wirklichkeit die meiste
Prozessorleistung für interne Rendering Schnittstellen genutzt wird.
\newline\newline
Der Benchmark ist in drei unterschiedliche Tests unterteilt: Vektor-, Bitmap-
und Textdarstellung. Die Testfälle sind dabei möglichst realen
Alltagsszenarien nachempfunden. Zusätzlich sind einige Testfälle vorhanden die
ebenfalls für mobile Endgeräte entwickelt wurden, um die Aussagen von Steve
Jobs, im Bezug auf die mobile Leistungsfähigkeit von HTML5 und Adobe Flash, zu
hinterfragen.
\newline\newline
Bei den folgenden Tests wird die Anzeigerate in Bilder pro Sekunde (fps,
Frames per Second) gemessen und miteinander verglichen.

\subsubsection{Vector Charting Test}
Dieser Benchmark wurde konzipiert um die Funktionen der Vektor APIs mittels
einem animierten Diagramm auszulasten. Das Diagramm wird dabei durch den
massiven Gebrauch von Linien und aufwendigen Transparenzfüllungen erstellt.
% Statistik
\begin{table}[H]
\begin{center}
\begin{tabular}{|p{3cm}| >{\centering\arraybackslash}p{3cm} | >{\centering\arraybackslash}p{3cm}|}

  \hline & HTML5 & Flash \\ \hline
  OSX 10.9.1 & & \\ \hline
  Firefox & 6,86 fps & 43,85 fps \\ \hline
  Chrome & 30,62 fps & 48,38 fps \\ \hline
  Opera & 29,86 fps & 46,98 fps \\ \hline
  Safari & \textbf{55.76 fps} & \textbf{49.96 fps} \\ \hline
  & i.D. 30,7 fps & i.D. 47,3 fps \\ \hline
  Windows 7 SP1 & & \\ \hline
  Firefox & 10,05 fps & 23,96 fps \\ \hline
  Chrome & 12,42 fps & \textbf{59,89 fps} \\ \hline
  Opera &  \textbf{13,52 fps} & 15,26 fps \\ \hline
  IE & 10,24 fps & 22,43 fps \\ \hline
  & i.D. 11,6 fps & i.D. 30,39 fps \\ \hline

\end{tabular}
\end{center}
\caption{Testergebnisse - Vector Charting Test}
\end{table}
Auf dem Macbook Air zeigt sich, dass Apples Browser Safari mit rund 55fps die
beste HTML5 Leistung bringt und die Leistung von Flash überall gleich konstant
bleibt.
% 1 Pixel Stroke Results
\begin{table}[H]
\begin{center}
\begin{tabular}{|p{3cm}| >{\centering\arraybackslash}p{3cm} |}

  \hline & HTML5 \\ \hline
  OSX 10.9.1 & \\ \hline
  Firefox & 44,91 fps \\ \hline
  Chrome & 54,43 fps \\ \hline
  Opera & 54,68 fps \\ \hline
  Safari & \textbf{56.92 fps} \\ \hline
  Windows 7 SP1 &  \\ \hline
  Firefox & 10.54 fps \\ \hline
  Chrome & 27.36 fps \\ \hline
  Opera & \textbf{30.38 fps} \\ \hline
  IE & 10.51 fps \\ \hline

\end{tabular}
\end{center}
\caption{Testergebnisse - Vector Charting Test 1 Pixel Stroke}
\end{table}
% Bild Test

\subsubsection{Bitmap Gaming Test}
Der Bitmap Test simuliert ein Spiel auf dem Prinzip des Towerdefense.
Dabei wird eine Vielzahl an Bitmaps pro Bild unterschiedlich animiert. Mit
jedem fertigen Bild muss dieses auch wieder gelöscht werden um alle neuen
Zustände im neuen Bild darstellen zu können. Um einen Tiefeneffekt zu erzeugen
werden Bitmaps auf unterschiedlichen Ebenen positioniert (Z Depth Ordering)
und um die Bitmaps zu skalieren wird Anti-alising verwendet.
% Statistik
\begin{table}[H]
\begin{center}
\begin{tabular}{|p{3cm}| >{\centering\arraybackslash}p{3cm} | >{\centering\arraybackslash}p{3cm}|}

  \hline & HTML5 & Flash \\ \hline
  OSX 10.9.1 & & \\ \hline
  Firefox & 20,9 fps & 36,95 fps \\ \hline
  Chrome & 55,61 fps & 35,02 fps \\ \hline
  Opera & 55,86 fps & 36,28 fps \\ \hline
  Safari & \textbf{57.03 fps} & \textbf{39.33 fps} \\ \hline
  & i.D. 47,35 fps & i.D. 36,9 fps \\ \hline
  Windows 7 SP1 & & \\ \hline
  Firefox & 7,34 fps & \textbf{20,09 fps} \\ \hline
  Chrome & 14,68 fps & 17,55 fps \\ \hline
  Opera & \textbf{18,71 fps} & 18,79 fps \\ \hline
  IE & 10,66 fps & 18,09 fps \\ \hline
  & i.D. 12,85 fps & i.D. 18,63 fps \\ \hline

\end{tabular}
\end{center}
\caption{Testergebnisse - Bitmap Gaming Test}
\end{table}
% Bild Test

\subsubsection{Text Column Test}
Bei diesem Test wird die Text Layout und Rendering Engine von HTML und Flash
ausgelastet. Es werden Custom Fonts, die mit CSS3 eingeführt wurden,
eingebunden und Multibyte Character Strings dargestellt. Anders als die
vorherigen Benchmarks stellt dieser Testfall kein reales Testszenario dar,
sondern dient der Analyse, wie viel Zeit benötigt wird, um eine textlastige
Webseite vollständig zu laden. Christmann nennt diesen Test den "`Eisberg"'-
Test, da ungefähr 80\% der Prozessorbelastung ausserhalb des Renderviews (des
sichtbaren Bereichs) entstehen. Dennoch soll der Test realistische Ergebnisse
liefern, da beim Scrollen berechnet werden muss, wieviele Zeilen des Textes nun
in bzw. aus dem Renderview bewegt werden müssen. Bei HTML Webseiten geschieht
dies immer, sobald eine Seite geöffnet wird und die Auflösung des genutzen
Monitors zu klein ist, um diese komplett anzeigen zu können.
% Statistik
\begin{table}[H]
\begin{center}
\begin{tabular}{|p{3cm}| >{\centering\arraybackslash}p{3cm} | >{\centering\arraybackslash}p{3cm}|}

  \hline & HTML5 & Flash \\ \hline
  OSX 10.9.1 & & \\ \hline
  Firefox & 28,09 fps & 32,3 fps \\ \hline
  Chrome & \textbf{31,85 fps} & \textbf{35,74 fps} \\ \hline
  Opera & 28,54 fps & 30,08 fps \\ \hline
  Safari & \textbf{31,85 fps} & 34,8 fps \\ \hline
  & i.D. 30,08 fps & i.D. 33,23 fps \\ \hline
  Windows 7 PS1 & & \\ \hline
  Firefox & \textbf{16,15 fps} & 11,53 fps \\ \hline
  Chrome & 15,29 fps & \textbf{18,74 fps} \\ \hline
  Opera & 11,39 fps & 11,38 fps \\ \hline
  IE & 10,6 fps & 11,08 fps \\ \hline
  & i.D. 13,36 fps & i.D. 13,18 fps \\ \hline

\end{tabular}
\end{center}
\caption{Testergebnisse - Text Column Test}
\end{table}
% Bild Test

\subsection{GUIMark 3: Mobile Showdown}

\subsection{The Man in Blue Animation Benchmark}

\subsection{Animation}

\subsection{Video}

\section{Besonderheiten}

% http://www.oesterreich.gv.at/site/5566/default.aspx#a4
% http://www.barrierefreies-webdesign.de/wcag2/
\subsection{Barrierefreiheit}
Barrierefreiheit im Internet bedeutet, Webseiten so zu gestalten, dass
jeder Mensch unabhängig von seinen körperlichen Einschränkungen die
Inhalte der Webseite nutzen kann. Dabei stellt besonders die Aufbereitung der
Inhalte für Menschen mit Sehbehinderungen eine Schwierigkeit dar.
Nichtsdestotrotz soll diese und auch andere Behinderungen bei der
Aufbereitung von Inhalten für das Internet berücksichtigt werden.
\newline\newline
Das W3C hat bereits im Jahre 1999 die erste Empfehlung für barrierefreies
Internet ausgesprochen. Für das Erstellen barrierefreier Webseiten werden,
seit 2008, Richtlinien in den Web Content Accessibility Guidlines (WCAG 2.0)
festgehalten und der Öffentlichkeit zugänglich gemacht. Die WCAG 2.0 bilden
nun einen flexibleren und testbareren, neuen Standard für barrierefreies
Webdesign. Dieser deckt Zugänglichkeitsanforderungen für alle
Arten von Web-Inhalten (Text, Bilder, Audio oder Video) sowie Web-
Applikationen ab und definiert technologieunabhängige Richtlinien und
Erfolgskriterien.
\newline\newline
Dabei sind die WCAG 2.0 pyramidenartig aufgebaut und umfassen vier Ebenen:
\begin{enumerate}
  \item 4 Prinzipien
  \item 12 Richtlinien
  \item 61 Erfolgskriterien
  \item unzählige Techniken
\end{enumerate}
Die ersten drei Ebenen stellen einen Maßstab und das Fundament der Richtlinien
dar. Im Gegensatz umfasst die vierte Ebene ergänzende Dokumente, die nicht
normativ sind und regelmäßig aktualisiert werden.
Im folgenden werden die Prinzipen und Richtlinien genauer betrachtet:
\begin{enumerate}
  \item Prinzip: Wahrnehmbar - Informationen und Bestandteile der
  Benutzerschnittstellen müssen den Benutzer so präsentiert werden, dass
  diese sie wahrnehmen können.
  \begin{itemize}
    \item Richtlinie 1.1 Textalternativen: Stellen Sie Textalternativen für
    alle Nicht-Text-Inhalte zur Verfügung, so dass in andere vom Benutzer
    benötigte Formen geändert werden können, wie zum Beispiel Großschrift,
    Braille, Symbole oder einfachere Sprache.
    \item Richtlinie 1.2 Zeitbasierte Medien: Stellen Sie Alternativen für
    zeitbasierte Medien zur Verfügung.
    \item Richtlinie 1.3 Anpassbar: Erstellen Sie Inhalte, die auf
    verschiedene Arten dargestellt werden können (z.B. einfacheres Layout),
    ohne dass Informationen oder Struktur verloren gehen.
    \item Richtlinie 1.4 Unterscheidbar: Machen Sie es Benutzern leichter,
    Inhalt zu sehen und zu hören einschließlich der Trennung von Vorder- und
    Hintergrund.
  \end{itemize}
  \item Prinzip: Bedienbar - Bestandteile der Benutzerschnittstelle und
  Navigation müssen bedienbar sein.
  \begin{itemize}
    \item Richtlinie 2.1 Per Tastatur zugänglich: Sorgen Sie dafür, dass alle
    Funktionalitäten per Tastatur zugänglich sind.
    \item Richtlinie 2.2 Ausreichend Zeit: Geben Sie den Benutzern ausreichend
    Zeit, Inhalte zu lesen und zu benutzen.
    \item Richtlinie 2.3 Anfälle: Gestalten Sie Inhalte nicht auf Arten, von
    denen bekannt ist, dass sie zu Anfällen führen.
    \item Richtlinie 2.4 Navigierbar: Stellen Sie Mittel zur Verfügung, um
    Benutzer dabei zu unterstützen zu navigieren, Inhalte zu finden und zu
    bestimmen, wo sie sich befinden.
  \end{itemize}
  \item Prinzip: Verständlich - Informationen und Bedienung der
  Benutzerschnittstellen müssen verständlich sein.
  \begin{itemize}
    \item Richtlinie 3.1 Lesbar: Machen Sie Inhalte lesbar und verständlich.
    \item Richtlinie 3.2 Vorhersehbar: Sorgen Sie dafür, dass Webseiten
    vorhersehbar aussehen und funktionieren.
    \item Richtlinie 3.3 Hilfestellung bei der Eingabe: Helfen Sie den
    Benutzern dabei, Fehler zu vermeiden und zu korrigieren.
  \end{itemize}
  \item Prinzip: Robust - Inhalte müssen robust genug sein, damit sie
  zuverlässig von einer großen Auswahl an Benutzeragenten einschließlich
  assistierender Techniken interpretiert werden können.
  \begin{itemize}
    \item Richtlinie 4.1 Kompatibel: Maximieren Sie die Kompatibilität mit
    aktuellen und zukünftigen Benutzeragenten, einschließlich assistierender
    Techniken.
  \end{itemize}
\end{enumerate}
Im Vergleich HTML5 und Adobe Flash ist es verständlich, dass die
Webtechnologie HTML einen Vorteil daraus zieht, dass die aufgezeigten
Richtlinien vor allem für das Erstellen von klassischen Webseiten gegeben
wurden. Allerdings bietet auch Adobe Flash verschiedene Möglichkeiten, um
diese Richtlinien erfüllen zu können. Diese sind jedoch mit deutlich höherem
Aufwand verbunden, als es bei HTML der Fall ist. Dieses Problem ist nicht nur
auf Adobe Flash beschränkt, sondern umfasst auch andere im Web verwendete
Technologien und Programmiersprachen, die bei der Umsetzung der Richtlinien für
Barrierefreiheit Schwierigkeiten bereiten. Das hat zur Folge, dass zum
Beispiel blinde Menschen große Schwierigkeiten bei der Nutzung von Nicht-HTML
Inhalten haben, da diese nicht von z.B. Screenreadern verarbeitet werden
können.
\newline\newline
Mit den aktuellen Versionen des Adobe Flash Players und der ActionScript
Version 3 bietet Adobe Entwicklern neue Möglichkeiten, um dem Nutzer
alternative Inhalte anzubieten. Grundsätzlich sind die Möglichkeiten dabei
ähnlich denen von HTML5. Damit wären barrierefreie Webseiten beziehungsweise
Anwendungen durchaus auch mit Adobe Flash umsetzbar. Trotzdem bleibt die
Umsetzung der Richtlinien des W3C ein schwieriges Unterfangen und ist deutlich
aufwändiger und somit auch teurer, sodass häufig darauf verzichtet wird.
\newline\newline
Es zeigt sich aus dem Vergleich HTML5 und Adobe Flash und bei der Betrachtung
der durch die W3C empfohlenen Richtlinien, dass Barrierefreiheit im Internet
nicht eine Frage der eingesetzten Technologie/n sondern der vorhandenen
Ressourcen (Zeit, Geld, Mitarbeiter, \dots) ist.

% Beispiele für Barrierefreien Code HTML5/Flash

\subsection{Suchmaschinenoptimierung}
Der Bereich der Suchmaschinenoptimierung hat sich in den letzten Jahren
enorm entwickelt. Kaum eine Webseite, unabhängig ob privat oder kommerziell
genutzt, kommt ohne sie aus. Das liegt vorwiegend daran, dass Suchmaschinen
wie Google, Yahoo oder Bing dem Nutzer die Suche nach interessanten und
relevanten Inhalte vereinfachen. Die Suchmaschinenoptimierung spielt dabei
insofern eine gewichtige Rolle, dass für viele Nutzer nur die Suchergebnisse
auf der ersten Seite relevant sind. Oft wird der Suchstring angepasst bevor
überhaupt Ergebnisse auf der zweiten oder dritten Ergebnisseite in Betracht
gezogen werden. Somit ist es für den Betreiber einer Webseite wichtig, dass
seine Webseite möglichst weit vorne im Suchmaschinenindex auftaucht.
\newline\newline
Inhalte die mit Adobe Flash erstellt wurden, haben einen großen
Nachteil: Crawler und Spider (die für die Indizierung und Bewertung der
Webseiten zuständig sind) können diese nicht oder nur sehr schwer lesen und
daher auch nicht indizieren. Die Indizierung ist allerdings notwendig, um
überhaupt auf einer Suchergebnisseite gelistet zu werden. Demzufolge ist es
nachvollziehbar, dass Webseiten und Inhalte, die auf Flash basieren, nicht in
das Ranking von Suchmaschinen mit aufgenommen werden. Die
Suchmaschinenbetreiber suchen immer wieder nach einer Lösung zu diesem
Problem. Die jüngsten Ergebnisse sind das Auslesen von Texten und Links aus
Flash-Daten oder externen Quellen wie XML-Dateien, die die Inhalte für die
Flash-Seiten beinhalten. Trotz allem ist die Indizierung bei Flash bei weitem
nicht so effektiv wie bei HTML. Bei einer multimedialen Flash-Anwendung ist
es zum Beispiel nicht möglich Bilder, Video- oder Audioinhalte und deren
Bedeutung zu extrahieren und die Zuordnung von Unterseiten zur Navigation der
Webseite ist nur äußerst umständlich über das sogenannte Deeplinking möglich.
Wobei dieses Problem auch bei anderen Webtechnologien, wie zum Beispiel AJAX,
mit dessen Hilfe Inhalte in die aktuell geöffnete Webseite nachgeladen werden
können, bestehen.
\newline\newline
Bei der Verwendung von Adobe Flash und der Suchmaschinenoptimierung bietet
Google einige Ratschläge, die helfen sollen: Suchmaschinen-Robotern sollen
stehts Zugriff auf die selben Inhalte wie die Nutzer haben. Das bedeutet,
dass keine zusätzlichen Inhalte aussschließlich für Crawler und Spider zur
Verfügung gestellt werden sollen. Die Angabe von Informationen die nur von
Suchmaschinen-Robotern gelesen werden können nennt man Cloaking. Grob
umschrieben gibt man den Robotern dabei vor etwas zu sein was man nicht ist.
Weitere Empfehlungen von Google sind, auf die Erstellung von vollständig
auf Flash basierenden Webseiten zu verzichten und stattdessen gezielt
einzelne Flash-Inhalte, zum Beispiel Videos, in eine HTML Seite zu
integrieren. Eine Alternative dazu wäre die gewünschten Inhalte sowohl mit
HTML5 als auch mit Adobe Flash zu erstellen, womit die Roboter die
HTML-Version indizieren können und die Nutzer die eventuell besser
aufbereitete Flash-Version besuchen können.
\newline\newline
Das ein Suchmaschinenbetreiber seinen Nutzern Ratschläge bei der Erstellung
von Flash-Inhalten bereitstellt, zeigt wie umständlich es sein kann, diese für
Suchmaschinen zu optimieren. Für eine möglichst erfolgreiche Optimierung
der Suchergebnisse sollte stets ein externes Datenformat wie HTML oder
XML eingesetzt werden. Damit ist es wesentlich einfacher, ein optimales
Suchmaschinenergebnis zu erhalten. Auch multimediale Inhalte können
durch einfache Mittel mit Bedeutung versehen werden, die wiederum für die
Crawler und Spider einfach verständlich sind. Unter anderem sind die weiteren
Vorteile bei der Nutzung von HTML die Möglichkeit, Hyperlinks unterschiedlich
stark zu bewerten, Meta-Attribute wie Stichwörter und/oder eine Beschreibung
der Seite anzugeben und damit eine differenziertere Indizierung zu erreichen
und somit ein deutlich besseres Ergebnis bei Suchmaschinen zu erreichen.
\newline\newline
Ist es für den Betreiber einer Webseite wichtig, im Ranking von Suchmaschinen
weit vorne zu liegen, führt fast kein Weg an HTML vorbei. Adobe Flash Inhalte
können, wie sich gezeigt hat, auch indiziert werden, allerdings sind deren
Relevanz für Suchmaschinen nie so hoch wie die von HTML Inhalten.
Mit der neuen Version von HTML stehen nun noch mehr Möglichkeiten zum Einbinden
von multimedialen und interaktiven Inhalten zur Verfügung, die wiederum
eine optimale Alternative für die Optimierung von Suchmaschinenergebnissen
genutzt werden können.

%Die neuen Möglichkeiten der neuen HTML-Version bieten dabei viele Vorteile bei
%der Suchmaschinenoptimierung wie zum Beispiel die Alternativen zur Einbindung
%von multimedialen und interaktiven Inhalten.

\subsection{Schnittstellen}
HTML5 und Adobe Flash sind mächtige Entwicklertools, welche die Realisierung
komplexer Anwendungen ermöglichen. Dennoch ist es mittlerweile im Internet
notwendig, dass Schnittstellen zu anderen Technologien entweder bereits
vorhanden oder zumindest realisierbar sind. Technologien, die oft in einem
Verbund verwendet werden, sind Datenbanken (SQL), Auszeichnungs- und
Austauschsprachen wie XML oder JSON oder Programmiersprachen wie PHP oder
JAVA.
\newline\newline
Da HTML5 eine Auszeichnungssprache und keine Programmiersprache ist, können
Inhalte und Daten aus anderen Technologien nur dargestellt werden. Mit
Hilfe von JavaScript lassen sich diese Daten allerdings auch verarbeiten.
Demnach ist im Fall von HTML5 nicht die Frage, ob die Technologie ausreichende
Schnittstellen bietet, sondern ob es genügend Möglichkeiten zur Integration
verschiedener andersartiger Technologien bietet und ob diese Technologien
auf JavaScript-Aufrufe reagieren können. Daher müssen die Daten der
eigesetzten Technologie so aufbereitet werden, dass sie mittels HTML
dargestellt werden können, zum Beispiel als Text, Bild oder sonstigen
multimedialen Inhalt.
\newline\newline
Adobe Flash bietet ebenfalls Möglichkeiten für den Austausch von Daten
zwischen verschiedenen Technologien an. Allerdings müssen, wie bei HTML5,
sämtliche Daten im Voraus aufbereitet werden um innerhalb von Flash
verarbeitet und dargestellt werden zu können. Mit der, in Flash integrierten
Programmiersprache ActionScript, lassen sich die Daten in vielfältigerer
Weise weiterverarbeiten, als es mit purem HTML5 möglich wäre. Dennoch kann
mit Flash nicht direkt auf andere Technologien zugegriffen werden.

%\section{Vergleich von HTML5 Canvas und Adobe Flash}
%Proprietär vs Standard
%\subsection{Verfügbarkeit}
%\subsection{Audio und Video}
%\subsection{Animation}
%\subsection{Spiele}
%\subsection{Werbung}
%\subsection{Web-Applikationen}

%\section{Mögliche Auswirkungen}
%\subsection{Der Wandel von dem allgegenwärtigen Flash zu den neuen Webstandards}
%\subsection{Die Zukunft von HTML5 und Flash}
%\subsection{Unternehmen und ihre Einstellung zu neuen Ideen}

%\chapter{Auswirkungen}
%\section{Browser}
%\subsection{Chrome und Safari (WebKit)}
%Chrome von Google und Safari von Apple bauen beide auf der HTML-Rendering-Engine
%WebKit zur Darstellung von Webinhalten auf. Google Chrome konnte zunehmend mit
%dem bis dato meist genutzen Webbrowser Internet Explorer konkurrieren, bis es
%nach den Angaben des globalen Statistiksunternehmens StatCounter erstmals im Mai
%2012, weltweit die Spitzenposition einnehmen konnte. Mit einem
%durchschnittlichen Anteil von 35\% für Google Chrome und 7\% für Apple Safari
%stellen diese die beliebtesten Webbrowser dar.
% http://gs.statcounter.com/#browser-ww-monthly-201204-201304-bar
%Im Vergleich decken beide Browser den größten Teil der HTML5 Unterstützung ab.
% Vergleiche HTML5 Unterstützung - http://html5readiness.com/

%Die WebKit Engine wurde auf der Grundlage des KDE-Projekts KHTML als Open Source
%Projekt von Apple entwickelt. Apple ist sehr daran interessiert, dass HTML5 so
%schnell wie möglich auf allen Geräten verfügbar ist, um den Benutzern ein
%uneingeschränktes Internet anbieten zu können. Auch die Mobile Safari Variante
%auf den iDivices basieren auf der WebKit-Engine.

%\subsection{Firefox (Gecko)}
%Der Open Source Webbrowser Firefox von Mozilla stellt im deutschsprachigen Raum
%den meist genutzten Webbrowser dar. Weltweit betrachtet befindet sich Firefox
%mit einem durchschnittlichen Anteil von 23\% hinter dem Internet Explorer von
%Microsoft und Google Chrome.
%Mozilla zeichnete sich schon früh als Unterstützer des HTML5 Standards ab. So
%war es schon mit sehr frühen Versionen von Firefox möglich, das Video-Element zu
%nutzen.

%\subsection{Internet Explorer (Trident)}
%Der wohl bekannteste Webbrowser Internet Explorer von Microsoft ist bereits in
%der zehnten Version veröffentlicht worden. Schon mit dem Internet Explorer 9 war
%Microsoft darauf bedacht die Geschwindigkeit ihres Browsers zu verbessern und es
%wurde darauf verzichtet, eigene Standards durchzusetzen.
%Erstmals orientierte man sich gänzlich an den Spezifikationen und unterstütze
%bereits einige HTML5 Features.
%So wurde das Audio/Video-Element unterstützt, wobei nur auf proprietäre Codecs
%wie H.264 und MP3/AAC zurückgegriffen wurde. Seit dem Internet Explorer 10 wird
%auch der freie Codec WebM unterstützt.

%Microsoft hatte sich aus seinem größten Manko einen Vorteil verschafft: Dadurch,
%das der Internet Explorer nur auf Windows Geräten läuft, kann er die ganze
%Plattform und Hardware nutzen. Im Gegensatz zur Browsersoftware der Konkurrenz,
%die auf den kleinsten gemeinsamen Nenner zurückgreifen müssen.
%So kann Internet Explorers JavaScript Engine Chakra, die Leistung von Multicore
%Prozessoren ausnutzen.

%Mit dem Internet Explorer 10 hat Microsoft einiges aus der Vergangenheit
%aufgeholt, allerdings sollten diese neuen Features nicht überbewertet werden,
%denn die Einsatzfähigkeit von HTML5 zeichnet sich weniger durch die neuen
%Browser, sonder viel mehr durch die Anzahl an veralteten Browser aus. Solange
%die alten Versionen des Internet Explorer im Gebrauch sind können viele
%Neuheiten von HTML5 nicht sinnvoll und ohne Fallbacklösung verwendet werden.